\documentclass[handout]{beamer} 
\title{ITCS 531: LA3 homework solutions}
\date{}
\author{Rob Egrot}

\usepackage{amsmath, bbold, bussproofs,graphicx}
\usepackage{mathrsfs}
\usepackage{amsthm}
\usepackage{amssymb}
\usepackage[all]{xy}
\usepackage{multirow}
\usepackage{tikz-cd}


\newtheorem{proposition}[theorem]{Proposition}
\newcommand{\bN}{\mathbb{N}}
\newcommand{\bZ}{\mathbb{Z}}
\newcommand{\bQ}{\mathbb{Q}}
\newcommand{\bR}{\mathbb{R}}
\newcommand{\bP}{\mathbb{P}}
\newcommand{\bC}{\mathbb{C}}
\newcommand{\bF}{\mathbb{F}}
\newcommand{\spa}{\mathrm{span}}
\newcommand{\sL}{\mathscr{L}}
\newcommand{\lequiv}{\models\text{\reflectbox{$\models$}}}
\newcommand{\cL}{\mathcal{L}}
\DeclareMathOperator{\ran}{\mathrm{ran}}

\addtobeamertemplate{navigation symbols}{}{%
    \usebeamerfont{footline}%
    \usebeamercolor[fg]{footline}%
    \hspace{1em}%
    \insertframenumber/\inserttotalframenumber
}
\setbeamertemplate{theorems}[numbered]
\begin{document}

\begin{frame}
\titlepage
\end{frame}

\begin{frame}
\frametitle{LA3 Q1(1)}
Let $b,c\in \bR$. Define $T:\bR^3\to \bR^2$ by 
\[T(x,y,z) = (2x-4y+3z+b, 6x +cxyz).\]
Prove that $T$ is linear if and only if $b=c=0$.
\begin{itemize}
\item If $b=c=0$ then $T(x,y,z) = (2x-4y+3z, 6x)$. 
\item We check the conditions of definition 3.1. 
\item First,
\begin{align*}
&T(x_1 + x_2,y_1+y_2,z_1+z_2)\\
=& (2(x_1+x_2) - 4(y_1+y_2) +3(z_1+z_2), 6(x_1+x_2))\\
=& (2x_1 - 4y_1 +3z_1, 6x_1) + (2x_2 - 4y_2 +3z_2, 6x_2)\\
=& T(x_1,y_1,z_1) + T(x_2,y_2,z_2).
\end{align*} 
\item Second,
\begin{align*}
T(\lambda x, \lambda y, \lambda z) &= (2\lambda x-4\lambda y+3 \lambda z, 6\lambda x)\\
&= \lambda (2 x-4 y+3 z, 6 x)\\
&= \lambda T(x,y,z).
\end{align*}
\end{itemize}
\end{frame}

\begin{frame}
\frametitle{LA3 Q1(2)}
Let $b,c\in \bR$. Define $T:\bR^3\to \bR^2$ by 
\[T(x,y,z) = (2x-4y+3z+b, 6x +cxyz).\]
Prove that $T$ is linear if and only if $b=c=0$.
\begin{itemize}
\item Conversely, if $T$ is linear then $2T(1,0,0) = T(2,0,0)$. 
\item So $2(2+b,6) = (4+b,12)$. 
\item I.e. $(4+2b,12) = (4+b,12)$, and so $b$ must be zero. 
\item Also, $T(1,1,1) = T(1,0,0) + T(0,1,1)$. 
\item So
\[(2-4+3,6 + c) = (2, 6) + (-4+3,0) = (2-4+3,6),\]
\item So $c=0$. 
\end{itemize}
\end{frame}

\begin{frame}
\frametitle{LA3 Q2}
Let $T\in \cL(V,W)$. Let $v_1,\ldots, v_n\in V$ and suppose that $(T(v_1),\ldots, T(v_n))$ is linearly independent in $W$. Prove that $(v_1,\ldots, v_n)$ is linearly independent in $V$.\vspace{0.5cm}
\begin{itemize}
\item Suppose $a_1v_1+\ldots+a_nv_n = 0$. \vspace{0.2cm}
\item Then, as $T$ is linear, we have
\[a_1T(v_1)+\ldots +a_nT(v_n)=T(a_1v_1+\ldots+a_nv_n) = T(0) = 0.\]
\item As $T(v_1),\ldots, T(v_n)$ is linearly independent, it follows that $a_1=\ldots=a_n=0$. \vspace{0.2cm}
\item So $(v_1,\ldots, v_n)$ is also linearly independent. 
\end{itemize}
\end{frame}

\begin{frame}
\frametitle{LA3 Q4(1)}
Let $S\in \cL(U,V)$ and let $T\in\cL(V,W)$. Let $(u_1,\ldots,u_n)$, $(v_1,\ldots,v_m)$ and $(w_1,\ldots, w_p)$ be bases for $U$, $V$ and $W$ respectively. Suppose that $B$ is the matrix of $T$ with respect to $(v_1,\ldots,v_m)$ and $(w_1,\ldots, w_p)$, and that $A$ is the matrix of $S$ with respect to $(u_1,\ldots,u_n)$ and $(v_1,\ldots,v_m)$. Then $BA$ is the matrix of $TS$ with respect to $(u_1,\ldots,u_n)$ and $(w_1,\ldots, w_p)$.

\end{frame}

\begin{frame}
\frametitle{LA3 Q4(2)}
\begin{itemize}
\item What should $TS$ do to the basis vector $u_i$ of $U$? \vspace{0.2cm}
\item As $A$ is the matrix of $S$, to find $Su_i$ we look at what $A$ does to the column vector that is zeroes except for 1 in the $i$th place. \vspace{0.2cm}
\item So the result is $a_{1i}v_1 + \ldots + a_{mi}v_m$. \vspace{0.2cm}
\item What does $T$ do to a basis vector $v_j$ of $V$? \vspace{0.2cm}
\item The matrix $B$ tells us that $T(v_j) = b_{1j}w_1+\ldots+ b_{pj}w_p$. \vspace{0.2cm}
\item So, 
\begin{align*}&TS(u_i)\\ 
=& T(a_{1i}v_1 + \ldots + a_{mi}v_m)\\
=& a_{1i}T(v_1)+\ldots + a_{mi}T(v_m)\\
=& a_{1i}(b_{11}w_1+\ldots+ b_{p1}w_p) + \ldots + a_{mi}(b_{1m}w_1+\ldots+ b_{pm}w_p). \end{align*}
\end{itemize}
\end{frame}

\begin{frame}
\frametitle{LA3 Q4(3)}
\begin{itemize}
\item We can rearrange this as
\begin{align*}&(a_{1i}b_{11} + \ldots + a_{mi}b_{1m})w_1\\
+&(a_{1i}b_{21} + \ldots + a_{mi}b_{2m})w_2\\ 
+& \ldots\\
+&(a_{1i}b_{p1}+\ldots +a_{mi}b_{pm})w_p.\end{align*}

\item But this is the $i$th column of the matrix $BA$: 
\[\begin{bmatrix}
a_{1i}b_{11} + \ldots + a_{mi}b_{1m}\\
a_{1i}b_{21} + \ldots + a_{mi}b_{2m}\\
\vdots\\
a_{1i}b_{p1}+\ldots +a_{mi}b_{pm}
\end{bmatrix}
\]

\item Since this is true for every basis vector $u_i$ of $U$, the transformation $TS$ is given by the matrix $BA$ as claimed.
\end{itemize}
\end{frame}

\begin{frame}
\frametitle{LA3 Q5}
Let $T\in\cL(V,W)$, and suppose both $V$ and $W$ are finite dimensional. Prove that, whatever the choice of bases for $V$ and $W$, the matrix of $T$ with respect to these bases must have at least $\dim \ran T$ entries that are not equal to $0$.\vspace{0.5cm}
\begin{itemize}
\item Let $A$ be the matrix of $T$ with respect to some pair of bases. \vspace{0.2cm}
\item If the $i$th column of $A$ is all zeroes, then this means $T(v_i)=0$, where $v_i$ is the $i$th basis vector for $V$. \vspace{0.2cm}
\item Since $(T(v_1),\ldots,T(v_n))$ spans $\ran T$, there must be at least $\dim\ran T$ columns of $A$ that are not all zeroes. \vspace{0.2cm}
\item This requires at least $\dim\ran T$ non-zero entries.
\end{itemize}
\end{frame}




\end{document}