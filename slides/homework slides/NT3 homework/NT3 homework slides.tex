\documentclass[handout]{beamer} 
\title{ITCS 531: Number Theory 3 solutions}
\date{}
\author{Rob Egrot}

\usepackage{amsmath, bbold, bussproofs,graphicx}
\usepackage{mathrsfs}
\usepackage{amsthm}
\usepackage{amssymb}
\usepackage[all]{xy}
\usepackage{multirow}
\usepackage{tikz-cd}


\newtheorem{Def}{Definition}
\newtheorem{Lem}{Lemma}
\newtheorem{Thm}{Theorem}
\newtheorem{Cor}{Corollary}
\newtheorem{Ex}{Example}
\newtheorem{Prop}{Proposition}
\newtheorem{Fact}{Fact}
\newtheorem{Que}{Question}

\newcommand{\bN}{\mathbb{N}}
\newcommand{\bZ}{\mathbb{Z}}
\newcommand{\bQ}{\mathbb{Q}}
\newcommand{\bR}{\mathbb{R}}
\newcommand{\bP}{\mathbb{P}}

\addtobeamertemplate{navigation symbols}{}{%
    \usebeamerfont{footline}%
    \usebeamercolor[fg]{footline}%
    \hspace{1em}%
    \insertframenumber/\inserttotalframenumber
}

\begin{document}

\begin{frame}
\titlepage
\end{frame}

\begin{frame}
\frametitle{Q1}
Prove that if $a,b,c\in\bZ$, with $a|bc$, and $a$ and $b$ be coprime, then $a|c$.
\begin{itemize}
\item Can't use lemma 1.11 directly as $a$ is not prime.
\item Since $a,b$ coprime take $x,y$ with $ax+by = 1$ (B\'ezout).
\item So $axc+byc = c$.
\item As $a\mid bc$ we have $bc= ak$ for some $k$.
\item So $axc +ayk = c$.
\item So $c = a(xc+yk)$. I.e. $a\mid c$.
\end{itemize}
\end{frame}

\begin{frame}
\frametitle{Q2}
Find all solutions to $x^2- 1 \equiv_{8} 0$. What does this tell us about Lagrange's theorem in the case where $p$ is not prime?
\vspace{1cm}
\begin{itemize}
\item The solutions are $1,3,5,7$. 
\vspace{0.5cm}
\item This tells us that Lagrange's theorem is false when $p$ is not prime.
\end{itemize}
\end{frame}

\begin{frame}
\frametitle{Q3}
Calculate $5^{30,000} - 6^{123,456}\mod 31$. 
\vspace{0.3cm}
\begin{itemize}
\item We have $5^{30000} = 5^{30(1000)}$, and $5^{30} \equiv_{31} 1$ by Fermat's little theorem. 
\item So $5^{30000}\equiv_{31} 1^{1000} \equiv_{31} 1$. 
\item Also, $123456= 30(4115)+6$, so 
\begin{align*}
6^{123456} &= 6^{30(4115)+6}\\
&= 6^{30(4115)}.6^6 \\
&\equiv_{31} 1^{4115}.6^6 \text{ using Fermat's little theorem}\\
&\equiv_{31} 6^2.6^2.6^2\\
&\equiv_{31} 5.5.5\\
&\equiv_{31} 125 \\
&\equiv_{31} 1.
\end{align*} 
So $5^{30,000} - 6^{123,456} = 0\mod 31$.
\end{itemize}
\end{frame}

\begin{frame}
\frametitle{Q4}
a) Prove that when $n=2$ we have $(n-1)! \equiv_n -1$.
\begin{itemize}
\item $(2-1)! = 1 = -1 \mod 2$.
\end{itemize}
\vspace{1cm}
b) Let $p$ be an odd prime. Define $g(x)= (x-1)(x-2)\ldots(x-(p-1))$.
\begin{itemize}
\item[i)] What are the roots of $g$ modulo $p$?
\item The roots are $1,2,\ldots,p-1$.
\item[ii)] What is the degree of $g$?
\item $p-1$.
\item[iii)] What is the leading term of $g$?
\item $x^{p-1}$.
\end{itemize}
\end{frame}

\begin{frame}
\frametitle{Q4}
c) Define $h(x)= x^{p-1} -1$. What are the roots of $h$ modulo $p$?
\vspace{1cm}
\begin{itemize}
\item $p$ is prime so little theorem says that $a^{p-1} \equiv_p 1$ whenever $a$ and $p$ are coprime.
\vspace{1cm}
\item In particular, $a^{p-1} \equiv_p 1$ for all $a\in\{1,\ldots,p-1\}$.
\vspace{1cm}
\item So $h(x)=x^{p-1} -1$ has roots $1,2,\ldots,p-1$ mod $p$ (these are all the roots as the degree of $h$ is $p-1$).
\end{itemize}
\end{frame}

\begin{frame}
\frametitle{Q4}
d) Define $f(x) = g(x)- h(x)$. Prove that $f_p$ must be the constant function $f(x)\equiv_ p 0$ for all $x$.
\begin{itemize}
\item Leading term of both $h$ and $g$ is $x^{p-1}$.
\item So degree of $g-h$ is at most $p-2$.
\item But every number that is a root of both $g$ and $h$ is also a root of $g-h$.
\item So $1,2,\ldots,p-1$ are all roots of $g-h$.
\item So $g-h$ has at least $p-1$ roots.
\item Since $p$ is prime, Lagrange's theorem says $g-h$ can have at most $p-2$ roots mod $p$, otherwise it is zero (mod $p$).
\item As $g-h$ has more than $p-2$ roots it must be zero (mod $p$).
\item I.e. $g(x)-h(x)\equiv_p 0$ for all $x$.
\end{itemize}
\end{frame}

\begin{frame}
\frametitle{Q4}
e) Prove that $n$ is prime if and only if $(n-1)! \equiv_n -1$.
\begin{itemize}
\item We have proved that if $n=2$ then $(n-1)! \equiv_n -1$ is true.
\item Let $n$ be an odd prime.
\item Then $g(x)\equiv_n h(x)$ for all $x$ - i.e. $(x-1)\ldots(x-(n-1)) \equiv_n x^{n-1}-1$.
\item With $x = n$ this gives $(n-1)! \equiv_n n^{n-1} -1 \equiv_n -1$.
\item Conversely, suppose $(n-1)! \equiv_n -1$ and choose $1\leq q < n$ with $q|n$.
\item Then as $kn = (n-1)! +1$ for some $k$, and as $q|(n-1)!$, we get $q|1$.
\item So $q=1$, and so $n$ must be prime.
\end{itemize}
\end{frame}

\end{document}