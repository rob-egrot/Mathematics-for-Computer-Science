\documentclass[handout]{beamer} 
\title{ITCS 531: L2 homework solutions}
\date{}
\author{Rob Egrot}

\usepackage{amsmath, bbold, bussproofs,graphicx}
\usepackage{mathrsfs}
\usepackage{amsthm}
\usepackage{amssymb}
\usepackage[all]{xy}
\usepackage{multirow}
\usepackage{tikz-cd}


\newtheorem{proposition}[theorem]{Proposition}
\newcommand{\bN}{\mathbb{N}}
\newcommand{\bZ}{\mathbb{Z}}
\newcommand{\bQ}{\mathbb{Q}}
\newcommand{\bR}{\mathbb{R}}
\newcommand{\bP}{\mathbb{P}}
\newcommand{\bC}{\mathbb{C}}
\newcommand{\bF}{\mathbb{F}}
\newcommand{\spa}{\mathrm{span}}

\addtobeamertemplate{navigation symbols}{}{%
    \usebeamerfont{footline}%
    \usebeamercolor[fg]{footline}%
    \hspace{1em}%
    \insertframenumber/\inserttotalframenumber
}
\setbeamertemplate{theorems}[numbered]
\begin{document}

\begin{frame}
\titlepage
\end{frame}

\begin{frame}
\frametitle{Deduction rules}
\scalebox{0.8}{\begin{minipage}{1.20\textwidth}
\begin{columns}
\begin{column}{0.5\textwidth}
\begin{prooftree}
\AxiomC{}
\LeftLabel{ $\top_I$:\quad}
\UnaryInfC{$\top$}
\end{prooftree}

\begin{prooftree}
\AxiomC{$\phi$}
\AxiomC{$\psi$}
\LeftLabel{$\wedge_I$:\quad}
\BinaryInfC{$\phi\wedge \psi$}
\end{prooftree} 

\begin{prooftree}
\AxiomC{$\phi$}
\LeftLabel{ $\vee_{I_l}$:\quad}
\UnaryInfC{$\phi\vee\psi$}
\end{prooftree}

\begin{prooftree}
\AxiomC{$\psi$}
\LeftLabel{ $\vee_{I_r}$:\quad}
\UnaryInfC{$\phi\vee\psi$}
\end{prooftree} 

\begin{prooftree}
\AxiomC{$[\phi]$}
\doubleLine
\UnaryInfC{$\bot$}
\LeftLabel{ $\neg_{I}$:\quad}
\UnaryInfC{$\neg\phi$}
\end{prooftree}

\begin{prooftree}
\AxiomC{$[\phi]$}
\doubleLine
\UnaryInfC{$\psi$}
\LeftLabel{ $\rightarrow_{I}$:\quad}
\UnaryInfC{$\phi\rightarrow\psi$}
\end{prooftree}
\end{column}

\begin{column}{0.5\textwidth}
\begin{prooftree}
\AxiomC{$\bot$}
\LeftLabel{ $\bot_E$:\quad}
\UnaryInfC{$\phi$}
\end{prooftree} 

\begin{prooftree}
\AxiomC{$\phi\wedge\psi$}
\LeftLabel{ $\wedge_{E_l}$:\quad}
\UnaryInfC{$\phi$}
\end{prooftree}

\begin{prooftree}
\AxiomC{$\phi\wedge\psi$}
\LeftLabel{ $\wedge_{E_r}$:\quad}
\UnaryInfC{$\psi$}
\end{prooftree}

\begin{prooftree}
\AxiomC{$\phi \vee \psi$}
\AxiomC{[$\phi$]}
\doubleLine
\UnaryInfC{$\theta$}
\AxiomC{[$\psi$]}
\doubleLine
\UnaryInfC{$\theta$}
\LeftLabel{$\vee_E$:\quad}
\TrinaryInfC{$\theta$}
\end{prooftree}

\begin{prooftree}
\AxiomC{$\phi$}
\AxiomC{$\neg\phi$}
\LeftLabel{$\neg_E$:\quad}
\BinaryInfC{$\bot$}
\end{prooftree} 


\begin{prooftree}
\AxiomC{$\phi\rightarrow\psi$}
\AxiomC{$\phi$}
\LeftLabel{$\rightarrow_E$:\quad}
\BinaryInfC{$\psi$}
\end{prooftree} 
\end{column}
\end{columns}
\begin{prooftree}
\AxiomC{$\neg\neg\phi$}
\LeftLabel{ $\neg\neg_{E}$:\quad}
\UnaryInfC{$\phi$}
\end{prooftree}
\end{minipage}}
\end{frame}

\begin{frame}
\frametitle{Q1}
The following deduction tree proves that $\phi\rightarrow\psi$ can be deduced from $\neg\phi\vee\psi$ in intuitionistic propositional logic. Add labels indicating the rules used at each stage.
\begin{prooftree}

\AxiomC{$\neg\phi\vee\psi$}

\AxiomC{$[\neg\phi]_1$}
\AxiomC{$[\phi]_2$}
\LeftLabel{ $\color{red}(\neg_E)$}
\BinaryInfC{$\bot$}
\LeftLabel{ $\color{red}(\bot_E)$}
\UnaryInfC{$\psi$}
\LeftLabel{ $\color{red}(\rightarrow_I)$}
\UnaryInfC{$\phi\rightarrow\psi$}

\AxiomC{$[\psi]_1$}
\UnaryInfC{$\psi$}
\AxiomC{$[\phi]_3$}
\RightLabel{ $\color{red}(\rightarrow_I)$}
\BinaryInfC{$\phi\rightarrow\psi$}
\LeftLabel{ $\color{red}(\vee_E)$}
\TrinaryInfC{$\phi\rightarrow\psi$}

\end{prooftree}
\end{frame}

\begin{frame}
\frametitle{Q2}
What is being proved in the following deduction tree? Add labels indicating the rules at each stage.

\begin{prooftree}
\AxiomC{$[\neg(\neg\phi\vee\psi)]_1$}
\AxiomC{$\phi\rightarrow\psi$}
\AxiomC{$[\phi]_2$}
\RightLabel{ $\color{red}(\rightarrow_E)$}
\BinaryInfC{$\psi$}
\RightLabel{ $\color{red}(\vee_{I_r})$}
\UnaryInfC{$\neg\phi\vee\psi$}
\LeftLabel{ $\color{red}(\neg_E)$}
\BinaryInfC{$\bot$}
\LeftLabel{ $\color{red}(\neg_I)$}
\UnaryInfC{$\neg\phi$}
\LeftLabel{ $\color{red}(\vee_{I_l})$}
\UnaryInfC{$\neg\phi\vee\psi$}
\AxiomC{$[\neg(\neg\phi\vee\psi)]_1$}
\UnaryInfC{$\neg(\neg\phi\vee\psi)$}
\RightLabel{ $\color{red}(\neg_E)$}
\BinaryInfC{$\bot$}
\RightLabel{ $\color{red}(\neg_I)$}
\UnaryInfC{$\neg\neg(\neg\phi\vee\psi)$}
\RightLabel{ $\color{red}(\neg\neg_E)$}
\UnaryInfC{$\neg\phi\vee\psi$}
\end{prooftree}
This tree proves that $\neg\phi\vee\psi$ can be deduced from $\phi\rightarrow\psi$ in \emph{classical} propositional logic. 
\end{frame}

\begin{frame}
\frametitle{Q3}
Show that $(\phi\wedge\psi)\rightarrow(\psi\wedge\phi)$ can be deduced from an empty set of axioms.
\vspace{0.5cm}
\begin{prooftree}
\AxiomC{$[\phi \wedge \psi]_1$}
\UnaryInfC{$\phi\wedge\psi$}
\LeftLabel{ $\color{red}(\wedge_{E_r})$}
\UnaryInfC{$\psi$}
\AxiomC{$[\phi \wedge \psi]_1$}
\UnaryInfC{$\phi\wedge\psi$}
\RightLabel{ $\color{red}(\wedge_{E_l})$}
\UnaryInfC{$\phi$}
\LeftLabel{ $\color{red}(\wedge_{I})$}
\BinaryInfC{$\psi\wedge\phi$}
\LeftLabel{ $\color{red}(\rightarrow_{I})$}
\UnaryInfC{$(\phi \wedge \psi)\rightarrow(\psi\wedge\phi)$}
\end{prooftree}
\end{frame}

\begin{frame}
\frametitle{Q4}
Show that we can deduce $\phi\wedge(\psi\vee\chi)$ if we start with $(\phi\wedge \psi)\vee (\phi\wedge \chi)$.

\vspace{1cm}
To save space let $\theta =(\phi\wedge \psi)\vee (\phi\wedge \chi)$.
\vspace{0.5cm}
\scalebox{0.8}{\begin{minipage}{1.20\textwidth}
\begin{prooftree}
\AxiomC{$\theta$}
\AxiomC{$[\phi\wedge\psi]$}
\LeftLabel{ $\color{red}(\wedge_{E_l})$}
\UnaryInfC{$\phi$}
\AxiomC{$[\phi\wedge\chi]$}
\RightLabel{ $\color{red}(\wedge_{E_l})$}
\UnaryInfC{$\phi$}
\LeftLabel{ $\color{red}(\vee_E)$}
\TrinaryInfC{$\phi$}

\AxiomC{$\theta$}
\AxiomC{$[\phi\wedge\psi]$}
\LeftLabel{ $\color{red}(\wedge_{E_r})$}
\UnaryInfC{$\psi$}
\LeftLabel{ $\color{red}(\vee_{I_l})$}
\UnaryInfC{$\psi\vee\chi$}
\AxiomC{$[\phi\wedge\chi]$}
\RightLabel{ $\color{red}(\wedge_{E_r})$}
\UnaryInfC{$\chi$}
\RightLabel{ $\color{red}(\vee_{I_r})$}
\UnaryInfC{$\psi\vee\chi$}
\RightLabel{ $\color{red}(\vee_E)$}
\TrinaryInfC{$\psi\vee\chi$}
\LeftLabel{ $\color{red}(\wedge_I)$}
\BinaryInfC{$\phi\wedge(\psi\vee\chi)$}
\end{prooftree}
\end{minipage}}
\end{frame}

\begin{frame}
\frametitle{Q5}
Show that we can deduce $(\phi\wedge \psi)\vee (\phi\wedge \chi)$ if we start with $\phi\wedge(\psi\vee\chi)$.\newline
\vspace{1cm}

\scalebox{0.8}{\begin{minipage}{1.20\textwidth}
\begin{prooftree}

\AxiomC{$\phi\wedge (\psi\vee \chi)$}
\LeftLabel{ $\color{red}(\wedge_{E_r})$}
\UnaryInfC{$\psi\vee \chi$}

\AxiomC{$\phi\wedge (\psi\vee \chi)$}
\LeftLabel{ $\color{red}(\wedge_{E_l})$}
\UnaryInfC{$\phi$}

\AxiomC{$[\psi]_1$}
\UnaryInfC{$\psi$}
\LeftLabel{ $\color{red}(\wedge_I)$}
\BinaryInfC{$\phi\wedge\psi$}
\LeftLabel{ $\color{red}(\vee_{I_l})$}
\UnaryInfC{$(\phi\wedge\psi)\vee(\phi\wedge\chi)$}

\AxiomC{$\phi\wedge (\psi\vee \chi)$}
\RightLabel{ $\color{red}(\wedge_{E_l})$}
\UnaryInfC{$\phi$}

\AxiomC{$[\chi]_1$}
\UnaryInfC{$\chi$}
\RightLabel{ $\color{red}(\wedge_I)$}
\BinaryInfC{$\phi\wedge\chi$}
\RightLabel{ $\color{red}(\vee_{I_r})$}
\UnaryInfC{$(\phi\wedge\psi)\vee(\phi\wedge\chi)$}


\LeftLabel{ $\color{red}(\vee_E)$}
\TrinaryInfC{$(\phi\wedge\psi)\vee(\phi\wedge\chi)$}

\end{prooftree}
\end{minipage}}
\end{frame}



\end{document}