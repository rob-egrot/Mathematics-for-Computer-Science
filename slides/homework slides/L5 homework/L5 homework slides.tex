\documentclass[handout]{beamer} 
\title{ITCS 531: L5 homework solutions}
\date{}
\author{Rob Egrot}

\usepackage{amsmath, bbold, bussproofs,graphicx}
\usepackage{mathrsfs}
\usepackage{amsthm}
\usepackage{amssymb}
\usepackage[all]{xy}
\usepackage{multirow}
\usepackage{tikz-cd}


\newtheorem{proposition}[theorem]{Proposition}
\newcommand{\bN}{\mathbb{N}}
\newcommand{\bZ}{\mathbb{Z}}
\newcommand{\bQ}{\mathbb{Q}}
\newcommand{\bR}{\mathbb{R}}
\newcommand{\bP}{\mathbb{P}}
\newcommand{\bC}{\mathbb{C}}
\newcommand{\bF}{\mathbb{F}}
\newcommand{\spa}{\mathrm{span}}
\newcommand{\sL}{\mathscr{L}}
\newcommand{\lequiv}{\models\text{\reflectbox{$\models$}}}
\newcommand{\cL}{\mathcal{L}}
\DeclareMathOperator{\ran}{\mathrm{ran}}

\addtobeamertemplate{navigation symbols}{}{%
    \usebeamerfont{footline}%
    \usebeamercolor[fg]{footline}%
    \hspace{1em}%
    \insertframenumber/\inserttotalframenumber
}
\setbeamertemplate{theorems}[numbered]
\begin{document}

\begin{frame}
\titlepage
\end{frame}


\begin{frame}
\frametitle{L1 Q1(a)}
Let $\phi$ be a formula where $x$ occurs free.
Write down a proof tree that shows $\forall x \phi \vdash \neg\exists x \neg\phi$. \vspace{0.5cm}
\begin{prooftree}
\AxiomC{$[\exists x\neg\phi]_1$}
\AxiomC{$[\neg\phi[x'/x]]_2$}
\AxiomC{$\forall x\phi$}
\RightLabel{ $\color{red}(\forall_E)$}
\UnaryInfC{$\phi[x'/x]$}
\RightLabel{ $\color{red}(\neg_E)$}
\BinaryInfC{$\bot$}
\RightLabel{ $\color{red}(\exists_E)$}
\BinaryInfC{$\bot$}
\RightLabel{ $\color{red}(\neg_I)$}
\UnaryInfC{$\neg\exists x \neg\phi$}
\end{prooftree}

\end{frame}

\begin{frame}
\frametitle{L1 Q1(b)}
Let $\phi$ be a formula where $x$ occurs free.
Write down a proof tree that shows $\exists x \phi \vdash \neg\forall x \neg \phi$.\vspace{0.5cm}
\begin{prooftree}
\AxiomC{$\exists x \phi$}
\AxiomC{$[\forall x\neg\phi]_1$}
\LeftLabel{ $\color{red}(\forall_E)$}
\UnaryInfC{$\neg\phi[x'/x]$}
\AxiomC{$[\phi[x'/x]]_2$}
\RightLabel{ $\color{red}(\neg_E)$}
\BinaryInfC{$\bot$}
\RightLabel{ $\color{red}(\exists_E)$}
\BinaryInfC{$\bot$}
\RightLabel{ $\color{red}(\neg_I)$}
\UnaryInfC{$\neg\forall x \neg\phi$}
\end{prooftree}
\end{frame}

\begin{frame}
\frametitle{L1 Q3}
Prove that if $\Gamma$ is an $\sL$-theory then there is an $\sL$-theory $\Gamma'$ with $\Gamma\subseteq \Gamma'$ such that $\Gamma'$ is \emph{complete} (i.e. if $\phi$ is an $\sL$-sentence, then either $\phi\in\Gamma'$ or $\neg\phi\in\Gamma'$).\vspace{0.5cm}
\begin{itemize}
\item An $\sL$-theory is a satisfiable set of $\sL$-sentences. \vspace{0.1cm}
\item Since $\Gamma$ is satisfiable, it must have a model. \vspace{0.1cm}
\item Let $A$ be a model for $\Gamma$, and let 
\[\Gamma'=\{\phi:\phi\text{ is an }\sL\text{-sentence and }A\models \phi\}.\] 
\item Then $\Gamma\subseteq \Gamma'$, because $A\models \Gamma$. 
\item $\Gamma'$ is complete because every $\sL$-sentence is either true or false in $A$.
\end{itemize}
\end{frame}

\begin{frame}
\frametitle{L1 Q4}
Let $\Gamma$ be an $\sL$-theory, and let $\phi$ be an $\sL$-sentence. Prove that if $\Gamma\models \phi$ then $\Delta\models \phi$ for some finite $\Delta\subseteq \Gamma$.\vspace{0.5cm}
\begin{itemize}
\item Suppose $\Gamma\models \phi$. \vspace{0.1cm}
\item By completeness we have $\Gamma\vdash \phi$. \vspace{0.1cm}
\item So there is a deduction tree using $\Gamma$ that proves $\phi$. \vspace{0.1cm}
\item As deduction trees are finite, this tree involves only a finite number of sentences from $\Gamma$. \vspace{0.1cm}
\item Define $\Delta$ to be the set of sentences from $\Gamma$ used in the proof of $\phi$. \vspace{0.1cm}
\item Then $\Delta\vdash \phi$, and so $\Delta\models \phi$ by soundness. 
\end{itemize}
\end{frame}

\begin{frame}
\frametitle{L1 Q5}
Let $\Gamma$ be a set of $\sL$-sentences. Prove that $\Gamma$ has a model if and only if every finite subset of $\Gamma$ has a model. \vspace{0.5cm}
\begin{itemize}
\item If $\Gamma$ has a model then every subset of $\Gamma$ has a model. \vspace{0.2cm}
\item Conversely, suppose $\Gamma$ does \emph{not} have a model. \vspace{0.2cm}
\item Then $\Gamma\models \bot$. \vspace{0.2cm}
\item So by exercise 4 there is finite $\Delta\subseteq \Gamma$ with $\Delta\models \bot$. \vspace{0.2cm}
\item I.e. $\Delta$ does not have a model. 
\end{itemize}
\end{frame}

\end{document}