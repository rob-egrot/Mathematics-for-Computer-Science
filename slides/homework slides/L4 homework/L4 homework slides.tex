\documentclass[handout]{beamer} 
\title{ITCS 531: L4 homework solutions}
\date{}
\author{Rob Egrot}

\usepackage{amsmath, bbold, bussproofs,graphicx}
\usepackage{mathrsfs}
\usepackage{amsthm}
\usepackage{amssymb}
\usepackage[all]{xy}
\usepackage{multirow}
\usepackage{tikz-cd}


\newtheorem{proposition}[theorem]{Proposition}
\newcommand{\bN}{\mathbb{N}}
\newcommand{\bZ}{\mathbb{Z}}
\newcommand{\bQ}{\mathbb{Q}}
\newcommand{\bR}{\mathbb{R}}
\newcommand{\bP}{\mathbb{P}}
\newcommand{\bC}{\mathbb{C}}
\newcommand{\bF}{\mathbb{F}}
\newcommand{\spa}{\mathrm{span}}
\newcommand{\sL}{\mathscr{L}}
\newcommand{\lequiv}{\models\text{\reflectbox{$\models$}}}

\addtobeamertemplate{navigation symbols}{}{%
    \usebeamerfont{footline}%
    \usebeamercolor[fg]{footline}%
    \hspace{1em}%
    \insertframenumber/\inserttotalframenumber
}
\setbeamertemplate{theorems}[numbered]
\begin{document}

\begin{frame}
\titlepage
\end{frame}

\begin{frame}
\frametitle{L4 Q1}
Which of the following are formulas? In the formulas identify the free and bound variables.  
\vspace{0.5cm}
\begin{enumerate}[a)]
\item $\forall x (f(c,f(x,d))$ -- \emph{not a formula}.
\vspace{0.3cm}
\item $R(x,y,z)\vee S(f(c,d))$ -- \emph{all free}.
\vspace{0.3cm}
\item $\exists y (R(x) \vee \forall z S(f(x,z),c))$ -- \emph{$x$ occurs free, $z$ occurs bound}.
\vspace{0.3cm}
\item $\exists y (R(x) \vee \forall y S(f(x,y),c))$ -- \emph{Here $y$ occurs bound, and $x$ occurs free. Note $\exists y$ is null}.
\vspace{0.3cm}
\item $R(x) \wedge \exists x S(x)$ -- \emph{$x$ occurs both free and bound.}.
\end{enumerate}
\end{frame}

\begin{frame}
\frametitle{L4 Q2 (1)}
Let $\sL=\{0,1,+,\times\}$ be the language of basic arithmetic. Let $\phi=\forall x(\neg(x\approx 0)\rightarrow \exists y(x\times y \approx 1))$. Let $\bN$ and $\bR$ have their usual meanings, and interpret $\sL$ into these languages by giving the non-logical symbols of $\sL$ their usual meanings. \vspace{0.4cm}
\begin{enumerate}[a)]
\item Does $\bN\models \phi$?
\item Does $\bR\models \phi$?
\end{enumerate} 
\vspace{0.5cm}
\begin{itemize}
\item $\phi$ is the statement that every non-zero element has an inverse.\vspace{0.2cm}
\item This is not true in $\bN$, but it is true in $\bR$. 
\end{itemize}
\end{frame}

\begin{frame}
\frametitle{L4 Q2 (2)}
\begin{enumerate}[]
\item[c)] Let $n\in\bN$ with $n\geq 2$, and let $\bZ_n$ be the integers mod $n$. For what values of $n$ does $\bZ_n\models \phi$? 
\item[d)] Let $A=(\{a,b\},I)$, where $I$ interprets $0$ and $1$ as $a$ and $b$ respectively, $b\times b = b$, and $a\times b = a\times a = a$. Does $A\models \phi$?
\end{enumerate} 
\vspace{0.5cm}
\begin{itemize}
\item Remember in $\bZ_n$, an element has an inverse iff it is coprime with $n$. 
\item So $\bZ\models \phi$ iff every non-zero element is coprime with $n$. I.e. iff $n$ is prime.\vspace{0.3cm}
\item In the structure $A$, the only non-zero element is $b$.
\item Does $b$ have an inverse? Yes, because $b\times b = b = 1$. 
\item So $A\models \phi$.
\end{itemize}
\end{frame}

\begin{frame}
\frametitle{L4 Q2 (3)}
\begin{enumerate}[]
\item[e)] Let $\psi= \exists x \forall y(\neg (y\approx 0)\rightarrow (x\times y \approx 1))$. Which of the structures in parts a)-d) is a model for $\psi$? 
\end{enumerate} 
\vspace{0.5cm}
\begin{itemize}
\item $\psi$ is the statement that there's an element which is an inverse to \emph{all} non-zero elements.\vspace{0.3cm}
\item This is clearly not true in $\bN$, $\bR$ or $\bZ_n$ for $n>2$.\vspace{0.3cm}
\item $\bZ_2\models \psi$ as 1 is the only non-zero element of $\bZ_2$.\vspace{0.3cm}
\item $\psi$ is true in $A$, as $b$ is the only non-zero element of $A$, and $b\times b = b = 1$.
\end{itemize}
\end{frame}

\begin{frame}
\frametitle{L4 Q3 (1)}
Let $R$ and $S$ be unary predicates. Let $\phi=\forall x R(x) \vee \forall x S(x)$, and let $\psi=\forall x \forall y (R(x)\vee S(y))$. Prove that $\phi\lequiv \psi$.
\vspace{0.5cm}
\begin{itemize}
\item Let $A$ be a structure of the appropriate kind. \vspace{0.2cm}
\item Suppose first that $A\models \phi$. \vspace{0.2cm}
\item Then either every element of $A$ satisfies $R$, or every element of $A$ satisfies $S$. \vspace{0.2cm}
\item In the former case, for every pair of elements $a,b\in A$ we must have $R(a)\vee S(b)$, because $R(a)$ must be true. \vspace{0.2cm}
\item So $A\models \forall x\forall y(R(x)\vee S(y))$. This shows $\phi\models \psi$.
\end{itemize}
\end{frame}

\begin{frame}
\frametitle{L4 Q3 (2)}
Let $R$ and $S$ be unary predicates. Let $\phi=\forall x R(x) \vee \forall x S(x)$, and let $\psi=\forall x \forall y (R(x)\vee S(y))$. Prove that $\phi\lequiv \psi$.
\vspace{0.5cm}
\begin{itemize}
\item Conversely, suppose $A\models \psi$. \vspace{0.2cm}
\item Suppose wlog that $A$ does not satisfy $\forall x R(x)$. \vspace{0.2cm}
\item Then there is $a\in A$ with $A\not\models R(a)$. I.e. $R(a)$ is not true. \vspace{0.2cm}
\item As $A\models \forall x\forall y(R(x)\vee S(y))$, \/$A\models \forall y(R(a)\vee S(y))$. \vspace{0.2cm}
\item As $R(a)$ is not true in $A$, it follows that $A\models \forall y S(y)$. \vspace{0.2cm}
\item By changing the variable name we have $A\models \forall x S(x)$, and so $A\models \forall x R(x) \vee \forall x S(x)$. \vspace{0.2cm}
\item This shows $\psi\models \phi$, and we are done.
\end{itemize}
\end{frame}


\begin{frame}
\frametitle{L4 Q4 (1)}
Let $\phi$ be an $\sL$-formula, let $A$ be an $\sL$-structure, and let $v$ be an assignment for $\sL$ to $A$ with $A,v \models \phi$. Prove that $A,u\models \phi$ for all assignments $u$ such that $u(x)= v(x)$ for all variables $x$ occurring free in $\phi$.\vspace{0.5cm}
\vspace{0.5cm}
\begin{itemize}
\item We induct on formula construction. \vspace{0.5cm}
\item It's obviously true for atomic formulas, because these have no bound variables. \vspace{0.5cm}
\item Suppose now that it's true for formulas $\phi$ and $\psi$.
\end{itemize}
\end{frame}

\begin{frame}
\frametitle{L4 Q4 (2)}
\begin{itemize}
\item[$\neg \phi$:]  \begin{itemize}
\item Suppose $A ,v\models \neg \phi$.\vspace{0.1cm} 
\item Then $A, v\not\models \phi$. \vspace{0.1cm} 
\item Suppose that $u$ agrees with $v$ about the free variables of $\phi$. \vspace{0.1cm} 
\item If $A,u\models \phi$, by the inductive hypothesis we would have $A, v\models \phi$, which would be a contradiction. \vspace{0.1cm} 
\item So we must have $A, u\models \neg\phi$ as required.
\end{itemize}\vspace{0.6cm}
\item[$\phi\vee\psi$] \begin{itemize}
\item Suppose $A,v\models \phi\vee\psi$. \vspace{0.1cm} 
\item Then, wlog we can assume that $A,v\models \phi$. \vspace{0.1cm} 
\item Let $u$ be an assignment agreeing with $v$ about the free variables of $\phi\vee\psi$. \vspace{0.1cm} 
\item Then $u$ agrees with $v$ about the free variables of $\phi$. \vspace{0.1cm} 
\item So $A, u\models \phi$, and thus $A,u\models \phi\vee\psi$ as required.
\end{itemize}
\end{itemize}
\end{frame}

\begin{frame}
\frametitle{L4 Q4 (3)}
\begin{itemize}
\item[$\forall x \phi$:]\begin{itemize} 
\item Suppose $A ,v\models \forall x \phi$, and let $u$ be an assignment agreeing with $v$ about the free variables of $\forall x\phi$. 
\item We must show that $A, u\models \forall x \phi$. 
\begin{itemize}
\item I.e. $A, u'\models \phi$ for all $u'$ agreeing with $u$ except possibly at $x$. 
\end{itemize}
\item Let $u'$ be such an assignment, and let $v'$ be an assignment agreeing with $v$ except possibly at $x$, where we define $v'(x)=u'(x)$. 
\item Then, if $F$ is the set of variables occurring free in $\phi$, we have
\begin{itemize}
\item $v'$ agrees with $v$ on $F\setminus\{x\}$.
\item $u$ agrees with $v$ on $F\setminus\{x\}$.
\item $u'$ agrees with $u$ on $F\setminus\{x\}$.
\item $u'$ agrees with $v'$ on $F$.
\end{itemize}
\item So,
\begin{align*}
A,v\models \forall x \phi &\implies A, v'\models \phi \text{ (by definition of $\models$)}  \\
&\implies A,u'\models \phi \text{ (by the inductive hypothesis)},
\end{align*}
\item And so $A,u\models \forall x\phi$ as required.
\end{itemize}
\end{itemize}
\end{frame}








\end{document}