\documentclass[handout]{beamer} 
\title{ITCS 531: Number Theory 1 solutions}
\date{}
\author{Rob Egrot}

\usepackage{amsmath, bbold, bussproofs,graphicx}
\usepackage{mathrsfs}
\usepackage{amsthm}
\usepackage{amssymb}
\usepackage[all]{xy}
\usepackage{multirow}
\usepackage{tikz-cd}


\newtheorem{Def}{Definition}
\newtheorem{Lem}{Lemma}
\newtheorem{Thm}{Theorem}
\newtheorem{Cor}{Corollary}
\newtheorem{Ex}{Example}
\newtheorem{Prop}{Proposition}
\newtheorem{Fact}{Fact}
\newtheorem{Que}{Question}

\newcommand{\bN}{\mathbb{N}}
\newcommand{\bZ}{\mathbb{Z}}
\newcommand{\bQ}{\mathbb{Q}}
\newcommand{\bR}{\mathbb{R}}
\newcommand{\bP}{\mathbb{P}}

\addtobeamertemplate{navigation symbols}{}{%
    \usebeamerfont{footline}%
    \usebeamercolor[fg]{footline}%
    \hspace{1em}%
    \insertframenumber/\inserttotalframenumber
}
\newtheorem*{theorem*}{Theorem}

\begin{document}

\begin{frame}
\titlepage
\end{frame}

\begin{frame}
\frametitle{Q1}
Consider the following (false) theorem: 
\begin{theorem*}If $a,b\in \bN$ and $a=b$ then $a=0$.
\end{theorem*}
\begin{proof}
\begin{align*}
a&=b\\
a^2&= ab\\
a^2-b^2&=ab-b^2\\
(a-b)(a+b)&=(a-b)b\\
a+b&= b\\
a&=0
\end{align*}
\end{proof}
What is wrong with this proof?
\begin{itemize}
\item Division by zero!
\end{itemize}
\end{frame}

\begin{frame}
\frametitle{Q2}
\scalebox{0.8}{\begin{minipage}{1.20\textwidth}
Use the well-ordering principle to show that \[2+4+6+\ldots + 2n = n(n+1).\]
\begin{itemize}
\item Suppose this is not true for all natural numbers.
\item By well-ordering, let $k$ be smallest number for which the identity is not true. 
\item $k$ cannot be $1$ as identity is true for $n=1$. 
\item Identity must be true for $k-1$. So we have 
\[2+\ldots + 2(k-1) = (k-1)(k),\] 
and so, adding $2k$ to both sides we have
\[2+\ldots + 2(k-1) +2k = (k-1)(k)+2k.\]
But 
\[(k-1)(k)+2k = (k)(k+1),\]
and so
\[2+\ldots +2k = (k)(k+1).\]
\item This is a contradiction because $k$ is supposed to invalidate the identity. 
\end{itemize}
\end{minipage}}
\end{frame}

\begin{frame}
\frametitle{Q3}
Let $n\in\bN$. If $n^2$ is even must $n$ also be even? Give a proof or a counterexample.
\vspace{0.5cm}
\begin{itemize}
\item By the fundamental theorem of arithmetic, $n^2$ has a unique prime factorization.
 \item The same is true for $n$. 
\item Suppose $n = p_1\ldots p_k$. 
\item Then $n^2=p_1\ldots p_kp_1\ldots p_k$. 
\item If $n^2$ is even then $2| n^2$. 
\item Since $n^2 = p_1\ldots p_kp_1\ldots p_k$, we must have $2| p_i$ for some $i\in \{1,\ldots,k\}$.
\item But this means $2|n$. 
\item So $n$ is even too.
\end{itemize}
\end{frame}

\begin{frame}
\frametitle{Q4}
Let $n\in\bN\setminus\{0\}$. Then using theorem 1.2 prove that $\log_5(n)$ is either a natural number or irrational.
\begin{itemize}
\item If $5^{\frac{a}{b}} = n$ then $5^a = n^b$. 
\item $n^b$ can be uniquely factorized into primes. 
\item Since $n^b = 5^a$ we know this factorization must just be $55\ldots 5$ (a list of $a$ fives). 
\item $n$ must also have a unique factorization into primes. 
\item As $n^b = 55\ldots 5$, the factorization of $n$ must be a list of fives.
\item I.e. $n = 5^k$ for some $k$.
\item  But if we take a product of $b$ copies of this list of fives we get $n^b$, which is $5^a$.
\item I.e. $5^a = (5^k)^b$, so $a = kb$. 
\item This means that $b$ must divide $a$. 
\item In other words, $\frac{a}{b}$ must be a natural number.  
\end{itemize}
\end{frame}

\begin{frame}
\frametitle{Q5}
Is the result from exercise 4 still true if we replace $5$ with $4$? Provide a proof or a counterexample.
\vspace{1cm}
\begin{itemize}
\item It's not true. 
\vspace{0.5cm}
\item For example, $\log_4{2} = \frac{1}{2}$.
\end{itemize}
\end{frame}

\end{document}