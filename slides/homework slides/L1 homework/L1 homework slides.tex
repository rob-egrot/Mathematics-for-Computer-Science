\documentclass[handout]{beamer} 
\title{ITCS 531: L1 homework solutions}
\date{}
\author{Rob Egrot}

\usepackage{amsmath, bbold, bussproofs,graphicx}
\usepackage{mathrsfs}
\usepackage{amsthm}
\usepackage{amssymb}
\usepackage[all]{xy}
\usepackage{multirow}
\usepackage{tikz-cd}


\newtheorem{proposition}[theorem]{Proposition}
\newcommand{\bN}{\mathbb{N}}
\newcommand{\bZ}{\mathbb{Z}}
\newcommand{\bQ}{\mathbb{Q}}
\newcommand{\bR}{\mathbb{R}}
\newcommand{\bP}{\mathbb{P}}
\newcommand{\bC}{\mathbb{C}}

\newcommand{\lequiv}{\models\text{\reflectbox{$\models$}}}
\newcommand{\lra}{\leftrightarrow}
\newcommand{\ra}{\rightarrow}
\newcommand{\la}{\leftarrow}

\addtobeamertemplate{navigation symbols}{}{%
    \usebeamerfont{footline}%
    \usebeamercolor[fg]{footline}%
    \hspace{1em}%
    \insertframenumber/\inserttotalframenumber
}
\setbeamertemplate{theorems}[numbered]
\begin{document}

\begin{frame}
\titlepage
\end{frame}

\begin{frame}
\frametitle{Q1}
Let $\phi$ and $\psi$ be sentences. Show that 
\[\phi\leftrightarrow \psi \lequiv (\phi \wedge \psi)\vee (\neg \phi \wedge \neg \psi)\]
\begin{itemize}
\item The truth tables are the same: 
\newline\newline
\begin{tabular*}{0.75\textwidth}{@{\extracolsep{\fill} }  c  c  c  c }
  $\phi$ & $\psi$ & $\phi\leftrightarrow \psi$ & $(\phi \wedge \psi)\vee (\neg \phi \wedge \neg \psi)$ \\
  \hline 
   T & T & T & T  \\
	 T & F & F & F  \\
	 F & T & F & F  \\
	 F & F & T & T  
\end{tabular*}
\end{itemize}
\end{frame}

\begin{frame}
\frametitle{Q2}
Prove that the set $\{\wedge,\neg\}$ is functionally complete.

\begin{itemize}
\item Must show that for every sentence $\phi$ there is a sentence $\phi'$ using only $\wedge$ and $\neg$ such that $\phi \lequiv\phi'$. 
\item Know that $\{\wedge,\vee,\neg,\lra\}$ is functionally complete.
\item Assume without loss of generality that $\phi$ only contains connectives from $\{\wedge,\vee,\neg,\lra\}$. 
\item Induction on formula construction. 
\begin{itemize}
\item Base case $\phi = p$. Just set $\phi' = p$.
\item Inductive step: if $\phi = \psi_1\vee \psi_2$ then define $\phi' = \neg(\neg \psi_1'\wedge \neg\psi_2')$. 
\item Use truth tables to show $\phi\lequiv \phi'$.
\item If $\phi = \psi_1\lra \psi_2$, notice that $(\psi'_1 \wedge \psi'_2)\vee (\neg \psi'_1 \wedge \neg \psi'_2)\lequiv \psi_1\lra\psi_2$. 
\item So, using the first case we can define $\phi' = \neg\big( \neg(\psi'_1 \wedge \psi'_2)\wedge \neg(\neg \psi'_1 \wedge \neg \psi'_2)\big)$.
\item Finally, if $\phi = \neg \psi$ or $\phi = \psi_1\wedge \psi_2$ just set $\phi' = \neg \psi'$ or $\phi' =  \psi_1'\wedge \psi_2'$.
\end{itemize}
\end{itemize}
\end{frame}

\begin{frame}
\frametitle{Q3}
\scalebox{0.8}{\begin{minipage}{1.20\textwidth}
Define a binary connective $|$ using the following truth table.

{\centering
\begin{tabular}{ c c c  }
 $\phi$ & $\psi$ & $\phi| \psi$ \\ \hline 
 T & T & F \\  
 T & F & T \\
 F & T & T \\
 F & F & T    
\end{tabular}\par}

Prove that $\{|\}$ is functionally complete.
\begin{itemize}
\item Show that if $\phi$ is a sentence involving only symbols from $\{\wedge,\neg\}$, there is a sentence $\phi'$ using only $|$ such that $\phi\lequiv \phi'$.
\item Suppose $\phi = \neg\psi$.  Observe that

{\centering
\begin{tabular}{ c c c c }
 $\psi$ & $\neg\psi$ & $\psi| \psi$ \\ \hline 
 T & F & F \\  
 F & T & T   
\end{tabular}\par} 
\item So we can define $\phi' = \psi'|\psi'$.

\item Suppose $\phi = \psi_1\wedge \psi_2$. Observe that

{\centering
\begin{tabular}{ c c c c c }
 $\psi_1$ & $\psi_2$ & $\psi_1\wedge \psi_2$ & $\psi_1|\psi_2$ & $(\psi_1|\psi_2)|(\psi_1|\psi_2)$ \\ \hline 
 T & T & T & F & T \\  
 T & F & F & T & F \\
 F & T & F & T & F \\
 F & F & F & T & F 
\end{tabular}\par}
\item So we can define $\phi' = (\psi'_1|\psi'_2)|(\psi'_1|\psi'_2)$.
\end{itemize}
\end{minipage}}
\end{frame}

\begin{frame}
\frametitle{Q4}
Let $p$ and $q$ be basic propositions. How many possible distinct truth tables are there for formulas involving only the propositions $p$ and $q$?
\vspace{0.5cm}
\begin{itemize}
\item There are 4 rows in each truth table for $p$ and $q$.

{\centering
\begin{tabular}{ c c c }
 $p$ & $q$ & $\phi$ \\ \hline 
 T & T & ?  \\  
 T & F & ?  \\
 F & T & ?  \\
 F & F & ?  
\end{tabular}\par}

\item Here each ? can be true or false. 
\item This gives $2^4 = 16$ distinct possibilities.
\end{itemize}
\end{frame}

\begin{frame}
\frametitle{Q5}
Show that every sentence is equivalent to a sentence in DNF.
\begin{itemize}
\item Consider this example. 
\item Suppose $\phi$ contains only the proposition symbols $p,q,r$, and that its truth table is as follows:

{\centering
\begin{tabular}{ c c c c }
 $p$ & $q$ & $r$ & $\phi$ \\ \hline 
 T & T & T & T  \\  
 T & T & F & F  \\
 T & F & T & F  \\
 T & F & F & F  \\
 F & T & T & T  \\  
 F & T & F & T  \\
 F & F & T & F  \\
 F & F & F & F  \\
\end{tabular}\par}
\item Then $\phi$ is obviously logically equivalent to $(p\wedge q \wedge r) \vee (\neg p\wedge q \wedge r) \vee (\neg p \wedge q \wedge \neg r)$. 
\item This is a DNF sentence. 
\item This method obviously generalizes. 
\item If $\phi$ is a contradiction then $\phi$ is equivalent to e.g. $p\wedge \neg p$.
\end{itemize}
\end{frame}


\end{document}