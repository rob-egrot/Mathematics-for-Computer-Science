\documentclass[handout]{beamer} 
\title{ITCS 531: Logic 2 - Deduction rules for propositional logic}
\date{}
\author{Rob Egrot}

\usepackage{amsmath, bbold, bussproofs,graphicx}
\usepackage{mathrsfs}
\usepackage{amsthm}
\usepackage{amssymb}
\usepackage[all]{xy}
\usepackage{multirow}
\usepackage{tikz-cd}



\addtobeamertemplate{navigation symbols}{}{%
    \usebeamerfont{footline}%
    \usebeamercolor[fg]{footline}%
    \hspace{1em}%
    \insertframenumber/\inserttotalframenumber
}

\setbeamertemplate{theorems}[numbered]

\newcommand{\bN}{\mathbb{N}}
\newcommand{\bZ}{\mathbb{Z}}
\newcommand{\bQ}{\mathbb{Q}}
\newcommand{\bR}{\mathbb{R}}
\newcommand{\bP}{\mathbb{P}}
\newcommand{\HCF}{\mathbf{HCF}}
\newcommand{\lequiv}{\models\text{\reflectbox{$\models$}}}

\begin{document}

\begin{frame}
\titlepage
\end{frame}

\begin{frame}
\frametitle{Semantic proof}
\begin{itemize}
\item Last week we saw how formulas and sets of formulas can imply other formulas according to truth tables. 
\vspace{0.4cm}
\item This allows us to make deductions about when a formula must be true assuming that certain other formulas are true. 
\vspace{0.4cm}
\item This method of deduction is \emph{semantic}. 
\vspace{0.4cm}
\item I.e. it is based on an idea of \emph{true} and \emph{false}. 
\vspace{0.4cm}
\item In other words, propositions have a meaning in a world where they are true or false.
\end{itemize}
\end{frame}

\begin{frame}
\frametitle{Syntactic proof}
\begin{itemize}
\item A different approach to logical deduction is to forget concepts like `true', `false' and `meaning'. 
\vspace{0.4cm}
\item I.e. just look at the structure of the formulas involved. 
\vspace{0.4cm}
\item This is known as \emph{syntax}.
\vspace{0.4cm}
\item We will develop a syntactical approach to deduction here. 
\end{itemize}
\end{frame}

\begin{frame}
\frametitle{Formal proofs in propositional logic}
\begin{itemize}
\item A formal proof begins with a (possibly empty) set of sentences, $\Gamma$, (considered to be axioms). 
\vspace{0.3cm}
\item In addition we have a collection of \textbf{deduction rules} (also called \emph{inference rules}). 
\vspace{0.3cm}
\item We use these to generate new sentences from combinations of ones previously generated. 
\vspace{0.3cm}
\item During this process the intended meaning of the sentences are irrelevant. 
\vspace{0.3cm}
\item The only important thing is their syntactic form. 
\vspace{0.3cm}
\item The set of sentences provable from $\Gamma$ is the set of sentences that can be obtained from $\Gamma$ using a finite number of applications of the inference rules.
\end{itemize}
\end{frame}

\begin{frame}
\frametitle{Natural deduction}
\begin{itemize}
\item There are many ways we can define deduction rules for propositional logic that are equivalent in a technical sense. 
\vspace{0.3cm}
\item We use a system called \emph{natural deduction}. 
\vspace{0.3cm}
\item The advantage is that it is relatively human readable. 
\vspace{0.3cm}
\item Natural deduction proofs resemble human argument. 
\vspace{0.3cm}
\item The disadvantage is that its proofs more difficult to formally reason about. 
\vspace{0.3cm}
\item This matters to proof theorists because they want to be able to prove theorems about the deductive power of formal systems. 
\vspace{0.3cm}
\item We're not worried about that though.
\end{itemize}
\end{frame}

\begin{frame}
\frametitle{Introduction rules}
\begin{prooftree}
\AxiomC{}
\LeftLabel{ $\top_I$:\quad}
\UnaryInfC{$\top$}
\end{prooftree}

\begin{prooftree}
\AxiomC{$\phi$}
\AxiomC{$\psi$}
\LeftLabel{$\wedge_I$:\quad}
\BinaryInfC{$\phi\wedge \psi$}
\end{prooftree} 

\begin{prooftree}
\AxiomC{$\phi$}
\LeftLabel{ $\vee_{I_l}$:\quad}
\UnaryInfC{$\phi\vee\psi$}
\end{prooftree}

\begin{prooftree}
\AxiomC{$\psi$}
\LeftLabel{ $\vee_{I_r}$:\quad}
\UnaryInfC{$\phi\vee\psi$}
\end{prooftree} 

\begin{prooftree}
\AxiomC{$[\phi]$}
\doubleLine
\UnaryInfC{$\bot$}
\LeftLabel{ $\neg_{I}$:\quad}
\UnaryInfC{$\neg\phi$}
\end{prooftree}

\begin{prooftree}
\AxiomC{$[\phi]$}
\doubleLine
\UnaryInfC{$\psi$}
\LeftLabel{ $\rightarrow_{I}$:\quad}
\UnaryInfC{$\phi\rightarrow\psi$}
\end{prooftree}
\end{frame}

\begin{frame}
\frametitle{Elimination rules}
\begin{prooftree}
\AxiomC{$\bot$}
\LeftLabel{ $\bot_E$:\quad}
\UnaryInfC{$\phi$}
\end{prooftree} 

\begin{prooftree}
\AxiomC{$\phi\wedge\psi$}
\LeftLabel{ $\wedge_{E_l}$:\quad}
\UnaryInfC{$\phi$}
\end{prooftree}

\begin{prooftree}
\AxiomC{$\phi\wedge\psi$}
\LeftLabel{ $\wedge_{E_r}$:\quad}
\UnaryInfC{$\psi$}
\end{prooftree}

\begin{prooftree}
\AxiomC{$\phi \vee \psi$}
\AxiomC{[$\phi$]}
\doubleLine
\UnaryInfC{$\theta$}
\AxiomC{[$\psi$]}
\doubleLine
\UnaryInfC{$\theta$}
\LeftLabel{$\vee_E$:\quad}
\TrinaryInfC{$\theta$}
\end{prooftree}

\begin{prooftree}
\AxiomC{$\phi$}
\AxiomC{$\neg\phi$}
\LeftLabel{$\neg_E$:\quad}
\BinaryInfC{$\bot$}
\end{prooftree} 


\begin{prooftree}
\AxiomC{$\phi\rightarrow\psi$}
\AxiomC{$\phi$}
\LeftLabel{$\rightarrow_E$:\quad}
\BinaryInfC{$\psi$}
\end{prooftree} 
\end{frame}

\begin{frame}
\frametitle{Intuitionistic propositional logic}
\begin{itemize}
\item These rules define \emph{intuitionistic propositional logic}. 
\vspace{0.5cm}
\item This is like classical propositional logic except that here $\neg\neg \phi$ does not imply $\phi$ 
\begin{itemize}
\vspace{0.2cm}
\item The converse is still true though! (see example \ref{E:neg} later). 
\end{itemize}
\vspace{0.5cm}
\item To get classical propositional logic we need one extra rule (double negation elimination).
\vspace{0.5cm}
\begin{prooftree}
\AxiomC{$\neg\neg\phi$}
\LeftLabel{ $\neg\neg_{E}$:\quad}
\UnaryInfC{$\phi$}
\end{prooftree}
\end{itemize}
\end{frame}

\begin{frame}
\frametitle{Using the deduction rules}
\begin{itemize}
\item Roughly speaking, introduction rules create new sentences by combining old ones with a logical connective. 
\vspace{0.3cm}
\item Elimination rules create new sentences by eliminating logical connectives from old ones. 
\begin{itemize}
\vspace{0.3cm}
\item There are some rules that don't fit this pattern in an obvious way. 
\end{itemize}
\vspace{0.3cm}
\item Derivations go from top to bottom.
\vspace{0.3cm} 
\item We can introduce sentences based on our axioms, then use the inference rules to derive new ones. 
\vspace{0.3cm}
\item Derived sentences go below the line. 
\end{itemize}
\end{frame}

\begin{frame}
\frametitle{Assumptions and subderivations}
\begin{itemize}
\item Sentences in square brackets, e.g. $[\phi]$, are \emph{assumptions}. 
\vspace{0.3cm}
\item When we make an assumption we have to discharge it later using one of the inferences rules $\neg_I$, $\rightarrow_I$, or $\vee_E$. 
\vspace{0.3cm}
\item We often use a subscript when making an assumption, e.g. $[\phi]_1$, so we can keep track of when we discharge it. 
\vspace{0.3cm}
\item We will discharge assumptions using `last in first out'. So, in a derivation, the last assumption made is the first to be discharged.  
\vspace{0.3cm}
\item Double lines (e.g. in $\vee_E$) represent a subderivation. 
\vspace{0.3cm}
\item That is, it stands for some arbitrary derivation beginning with the thing on the top and ending with the thing on the bottom. 
\end{itemize}
\end{frame}


\begin{frame}
\frametitle{Introduction rules again}
\begin{prooftree}
\AxiomC{}
\LeftLabel{ $\top_I$:\quad}
\UnaryInfC{$\top$}
\end{prooftree}

\begin{prooftree}
\AxiomC{$\phi$}
\AxiomC{$\psi$}
\LeftLabel{$\wedge_I$:\quad}
\BinaryInfC{$\phi\wedge \psi$}
\end{prooftree} 

\begin{prooftree}
\AxiomC{$\phi$}
\LeftLabel{ $\vee_{I_l}$:\quad}
\UnaryInfC{$\phi\vee\psi$}
\end{prooftree}

\begin{prooftree}
\AxiomC{$\psi$}
\LeftLabel{ $\vee_{I_r}$:\quad}
\UnaryInfC{$\phi\vee\psi$}
\end{prooftree} 

\begin{prooftree}
\AxiomC{$[\phi]$}
\doubleLine
\UnaryInfC{$\bot$}
\LeftLabel{ $\neg_{I}$:\quad}
\UnaryInfC{$\neg\phi$}
\end{prooftree}

\begin{prooftree}
\AxiomC{$[\phi]$}
\doubleLine
\UnaryInfC{$\psi$}
\LeftLabel{ $\rightarrow_{I}$:\quad}
\UnaryInfC{$\phi\rightarrow\psi$}
\end{prooftree}
\end{frame}

\begin{frame}
\frametitle{Elimination rules again}
\begin{prooftree}
\AxiomC{$\bot$}
\LeftLabel{ $\bot_E$:\quad}
\UnaryInfC{$\phi$}
\end{prooftree} 

\begin{prooftree}
\AxiomC{$\phi\wedge\psi$}
\LeftLabel{ $\wedge_{E_l}$:\quad}
\UnaryInfC{$\phi$}
\end{prooftree}

\begin{prooftree}
\AxiomC{$\phi\wedge\psi$}
\LeftLabel{ $\wedge_{E_r}$:\quad}
\UnaryInfC{$\psi$}
\end{prooftree}

\begin{prooftree}
\AxiomC{$\phi \vee \psi$}
\AxiomC{[$\phi$]}
\doubleLine
\UnaryInfC{$\theta$}
\AxiomC{[$\psi$]}
\doubleLine
\UnaryInfC{$\theta$}
\LeftLabel{$\vee_E$:\quad}
\TrinaryInfC{$\theta$}
\end{prooftree}

\begin{prooftree}
\AxiomC{$\phi$}
\AxiomC{$\neg\phi$}
\LeftLabel{$\neg_E$:\quad}
\BinaryInfC{$\bot$}
\end{prooftree} 


\begin{prooftree}
\AxiomC{$\phi\rightarrow\psi$}
\AxiomC{$\phi$}
\LeftLabel{$\rightarrow_E$:\quad}
\BinaryInfC{$\psi$}
\end{prooftree} 
\end{frame}

\begin{frame}
\frametitle{Some conventions}
\begin{itemize}
\item We implicitly assume we can deduce any formula from itself or an assumption of itself:\\
\vspace{0.5cm}
\begin{minipage}{0.5\textwidth}
\begin{prooftree} 
\AxiomC{$\phi$}
\UnaryInfC{$\phi$}
\end{prooftree}
\end{minipage}
\begin{minipage}{0.5\textwidth}
\begin{prooftree} 
\AxiomC{$[\phi]$}
\UnaryInfC{$\phi$}
\end{prooftree}
\end{minipage}
\vspace{0.5cm}
\item When we make deductions we can usually freely switch the order of sentences. 
\begin{itemize}
\vspace{0.2cm}
\item E.g. $\phi$ and $\neg \phi$ could be switched when applying rule $\neg_E$.
\end{itemize}
\end{itemize}
\end{frame}


\begin{frame}
\frametitle{Example: $\phi\rightarrow \phi$}
\begin{prooftree}
\AxiomC{$[\phi]$}
\doubleLine
\UnaryInfC{$\psi$}
\LeftLabel{ $\rightarrow_{I}$:\quad}
\UnaryInfC{$\phi\rightarrow\psi$}
\end{prooftree}
\begin{example}
We can deduce $\phi\rightarrow \phi$ from an empty set of axioms.
\begin{prooftree}
\AxiomC{$[\phi]_1$}
\UnaryInfC{$\phi$}
\RightLabel{\quad$(\rightarrow_I)_1$}
\UnaryInfC{$\phi\rightarrow\phi$}
\end{prooftree}
\end{example}
\end{frame}

\begin{frame}
\frametitle{Example: $\phi\vee \psi$ implies $\psi\vee \phi$}
\begin{columns}
\begin{column}{0.5\textwidth}
\begin{prooftree}
\AxiomC{$\phi$}
\LeftLabel{ $\vee_{I_l}$:\quad}
\UnaryInfC{$\phi\vee\psi$}
\end{prooftree}
\end{column}
\begin{column}{0.5\textwidth}
\begin{prooftree}
\AxiomC{$\phi \vee \psi$}
\AxiomC{[$\phi$]}
\doubleLine
\UnaryInfC{$\theta$}
\AxiomC{[$\psi$]}
\doubleLine
\UnaryInfC{$\theta$}
\LeftLabel{$\vee_E$:\quad}
\TrinaryInfC{$\theta$}
\end{prooftree}
\end{column}
\end{columns}
\begin{example}
If $\phi\vee \psi$ is an axiom then we can deduce $\psi\vee \phi$.
\begin{prooftree}
\AxiomC{$\phi\vee \psi$}
\AxiomC{$[\phi]_1$}
\UnaryInfC{$\phi$}
\RightLabel{\quad$(\vee_{I_r})$}
\UnaryInfC{$\psi\vee \phi$}
\AxiomC{$[\psi]_1$}
\UnaryInfC{$\psi$}
\RightLabel{\quad$(\vee_{I_l})$}
\UnaryInfC{$\psi\vee \phi$}
\RightLabel{\quad$(\vee_{E})_1$}
\TrinaryInfC{$\psi\vee\phi$}
\end{prooftree}
\end{example}
\end{frame}

\begin{frame}
\frametitle{Example: $\phi\rightarrow\neg\neg\phi$}
\begin{columns}
\begin{column}{0.3\textwidth}
\begin{prooftree}
\AxiomC{$\phi$}
\AxiomC{$\neg\phi$}
\LeftLabel{$\neg_E$:\quad}
\BinaryInfC{$\bot$}
\end{prooftree} 
\end{column}

\begin{column}{0.3\textwidth}
\begin{prooftree}
\AxiomC{$[\phi]$}
\doubleLine
\UnaryInfC{$\bot$}
\LeftLabel{ $\neg_{I}$:\quad}
\UnaryInfC{$\neg\phi$}
\end{prooftree}
\end{column}

\begin{column}{0.3\textwidth}
\begin{prooftree}
\AxiomC{$[\phi]$}
\doubleLine
\UnaryInfC{$\psi$}
\LeftLabel{ $\rightarrow_{I}$:\quad}
\UnaryInfC{$\phi\rightarrow\psi$}
\end{prooftree}

\end{column}
\end{columns}

\begin{example}\label{E:neg}
For all sentences $\phi$, we can derive $\phi\rightarrow\neg\neg\phi$ from an empty set of axioms, without using the rule $\neg\neg_E$.
\begin{prooftree}
\AxiomC{$[\phi]_1$}
\UnaryInfC{$\phi$}
\AxiomC{$[\neg\phi]_2$}
\UnaryInfC{$\neg\phi$}
\RightLabel{\quad$(\neg_E)$}
\BinaryInfC{$\bot$}
\RightLabel{\quad$(\neg_I)_2$}
\UnaryInfC{$\neg\neg\phi$}
\RightLabel{\quad$(\rightarrow_I)_1$}
\UnaryInfC{$\phi\rightarrow\neg\neg\phi$}
\end{prooftree} 
\end{example}
\end{frame}

\begin{frame}
\frametitle{Example: De Morgan's laws part 1}
\scalebox{0.85}{\begin{minipage}{1.20\textwidth}
\begin{columns}
\begin{column}{0.3\textwidth}
\begin{prooftree}
\AxiomC{$\phi\wedge\psi$}
\LeftLabel{ $\wedge_{E_l}$:\quad}
\UnaryInfC{$\phi$}
\end{prooftree}
\end{column}

\begin{column}{0.3\textwidth}
\begin{prooftree}
\AxiomC{$\phi$}
\AxiomC{$\neg\phi$}
\LeftLabel{$\neg_E$:\quad}
\BinaryInfC{$\bot$}
\end{prooftree}
\end{column}

\begin{column}{0.3\textwidth}
\begin{prooftree}
\AxiomC{$[\phi]$}
\doubleLine
\UnaryInfC{$\bot$}
\LeftLabel{ $\neg_{I}$:\quad}
\UnaryInfC{$\neg\phi$}
\end{prooftree}
\end{column}
\end{columns}
\begin{prooftree}
\AxiomC{$\phi \vee \psi$}
\AxiomC{[$\phi$]}
\doubleLine
\UnaryInfC{$\theta$}
\AxiomC{[$\psi$]}
\doubleLine
\UnaryInfC{$\theta$}
\LeftLabel{$\vee_E$:\quad}
\TrinaryInfC{$\theta$}
\end{prooftree}

\begin{example}
From $\phi\vee\psi$ we can deduce $\neg(\neg\phi\wedge \neg\psi)$.
\begin{prooftree}
\AxiomC{$(\phi\vee\psi)$}

\AxiomC{$[\phi]_1$}
\UnaryInfC{$\phi$}
\AxiomC{$[\neg\phi\wedge\neg \psi]_2$}
\RightLabel{\quad$(\wedge_{E_l})$}
\UnaryInfC{$\neg\phi$}
\RightLabel{\quad$(\neg_E)$}
\BinaryInfC{$\bot$}
\RightLabel{\quad$(\neg_I)_2$}
\UnaryInfC{$\neg(\neg\phi\wedge \neg\psi)$}

\AxiomC{$[\psi]_1$}
\UnaryInfC{$\psi$}
\AxiomC{$[\neg\phi\wedge\neg \psi]_3$}
\RightLabel{\quad$(\wedge_{E_r})$}
\UnaryInfC{$\neg\psi$}
\RightLabel{\quad$(\neg_E)$}
\BinaryInfC{$\bot$}
\RightLabel{\quad$(\neg_I)_3$}
\UnaryInfC{$\neg(\neg\phi\wedge \neg\psi)$}
\RightLabel{\quad $(\vee_E)_{1}$}
\TrinaryInfC{$\neg(\neg\phi\wedge \neg\psi)$}
\end{prooftree}
\end{example}\end{minipage}}
\end{frame}

\begin{frame}
\frametitle{Example: De Morgan's laws part 2}
\scalebox{0.85}{\begin{minipage}{1.20\textwidth}
\begin{columns}
\begin{column}{0.3\textwidth}
\begin{prooftree}
\AxiomC{$\phi$}
\LeftLabel{ $\vee_{I_l}$:\quad}
\UnaryInfC{$\phi\vee\psi$}
\end{prooftree}
\end{column}

\begin{column}{0.3\textwidth}
\begin{prooftree}
\AxiomC{$\phi$}
\AxiomC{$\neg\phi$}
\LeftLabel{$\neg_E$:\quad}
\BinaryInfC{$\bot$}
\end{prooftree}
\end{column}

\begin{column}{0.3\textwidth}
\begin{prooftree}
\AxiomC{$[\phi]$}
\doubleLine
\UnaryInfC{$\bot$}
\LeftLabel{ $\neg_{I}$:\quad}
\UnaryInfC{$\neg\phi$}
\end{prooftree}
\end{column}
\end{columns}

\begin{columns}

\begin{column}{0.3\textwidth}
\begin{prooftree}
\AxiomC{$\phi$}
\AxiomC{$\psi$}
\LeftLabel{$\wedge_I$:\quad}
\BinaryInfC{$\phi\wedge \psi$}
\end{prooftree}
\end{column}
\begin{column}{0.3\textwidth}
\begin{prooftree}
\AxiomC{$\neg\neg\phi$}
\LeftLabel{ $\neg\neg_{E}$:\quad}
\UnaryInfC{$\phi$}
\end{prooftree}
\end{column}
\end{columns}

\begin{example}
From $\neg(\neg\phi\wedge \neg\psi)$ we can deduce $\phi\vee\psi$.
\begin{prooftree}

\AxiomC{$\neg(\neg\phi\wedge \neg\psi)$}

\AxiomC{$[\phi]_2$}
\RightLabel{$(\vee_{I_l})$}
\UnaryInfC{$\phi\vee \psi$}
\AxiomC{$[\neg(\phi\vee\psi)]_1$}
\UnaryInfC{$\neg(\phi\vee\psi)$}
\RightLabel{$(\neg_E)$}
\BinaryInfC{$\bot$}
\RightLabel{$(\neg_I)_2$}
\UnaryInfC{$\neg\phi$}

\AxiomC{$[\psi]_3$}
\AxiomC{$[\neg(\phi\vee\psi)]_1$}
\doubleLine
\BinaryInfC{$\neg\psi$}

\RightLabel{$(\wedge_I)$}
\BinaryInfC{$\neg\phi\wedge\neg\psi$}
\RightLabel{$(\neg_E)$}
\BinaryInfC{$\bot$}
\RightLabel{$(\neg_I)_1$}
\UnaryInfC{$\neg\neg(\phi\vee\psi)$}
\RightLabel{$(\neg\neg_E)$}
\UnaryInfC{$\phi\vee\psi$}
\end{prooftree}


\end{example}\end{minipage}}
\end{frame}

\begin{frame}
\frametitle{Example: $\phi\vee\neg \phi$}
\scalebox{0.85}{\begin{minipage}{1.20\textwidth}
\begin{columns}
\begin{column}{0.3\textwidth}
\begin{prooftree}
\AxiomC{$\phi$}
\LeftLabel{ $\vee_{I_l}$:\quad}
\UnaryInfC{$\phi\vee\psi$}
\end{prooftree}
\end{column}

\begin{column}{0.3\textwidth}
\begin{prooftree}
\AxiomC{$\phi$}
\AxiomC{$\neg\phi$}
\LeftLabel{$\neg_E$:\quad}
\BinaryInfC{$\bot$}
\end{prooftree}
\end{column}

\begin{column}{0.3\textwidth}
\begin{prooftree}
\AxiomC{$[\phi]$}
\doubleLine
\UnaryInfC{$\bot$}
\LeftLabel{ $\neg_{I}$:\quad}
\UnaryInfC{$\neg\phi$}
\end{prooftree}
\end{column}
\end{columns}

\begin{columns}

\begin{column}{0.3\textwidth}
\begin{prooftree}
\AxiomC{$\neg\neg\phi$}
\LeftLabel{ $\neg\neg_{E}$:\quad}
\UnaryInfC{$\phi$}
\end{prooftree}
\end{column}
\end{columns}
\begin{example}
$\phi\vee\neg \phi$ is a theorem of classical propositional logic (i.e. it can be deduced from an empty set of axioms).
\begin{prooftree}
\AxiomC{$\neg(\neg\phi\vee\phi)$}
\AxiomC{$[\neg(\neg\phi\vee\phi)]_1$}
\UnaryInfC{$\neg(\neg\phi\vee\phi)$}
\AxiomC{$[\phi]_2$}
\UnaryInfC{$\phi$}
\RightLabel{\quad$(\vee_{I_r})$}
\UnaryInfC{$\neg\phi\vee\phi$}
\RightLabel{\quad$(\neg_E)$}
\BinaryInfC{$\bot$}
\RightLabel{\quad$(\neg_I)_2$}
\UnaryInfC{$\neg\phi$}
\RightLabel{\quad$(\vee_{I_l})$}
\UnaryInfC{$\neg\phi\vee\phi$}
\RightLabel{\quad$(\neg_E)$}
\BinaryInfC{$\bot$}
\RightLabel{\quad$(\neg_I)_1$}
\UnaryInfC{$\neg\neg(\neg\phi\vee\phi)$}
\RightLabel{\quad$(\neg\neg_E)$}
\UnaryInfC{$\neg\phi\vee\phi$}
\end{prooftree}
\end{example}\end{minipage}}
\end{frame}

\begin{frame}
\frametitle{Example: $\psi$ from $\phi\vee \psi$ and $\neg \phi$}
\begin{columns}
\begin{column}{0.3\textwidth}
\begin{prooftree}
\AxiomC{$\bot$}
\LeftLabel{ $\bot_E$:\quad}
\UnaryInfC{$\phi$}
\end{prooftree} 
\end{column}

\begin{column}{0.3\textwidth}
\begin{prooftree}
\AxiomC{$\phi$}
\AxiomC{$\neg\phi$}
\LeftLabel{$\neg_E$:\quad}
\BinaryInfC{$\bot$}
\end{prooftree}
\end{column}

\end{columns}
\begin{prooftree}
\AxiomC{$\phi \vee \psi$}
\AxiomC{[$\phi$]}
\doubleLine
\UnaryInfC{$\theta$}
\AxiomC{[$\psi$]}
\doubleLine
\UnaryInfC{$\theta$}
\LeftLabel{$\vee_E$:\quad}
\TrinaryInfC{$\theta$}
\end{prooftree}
\begin{example}\label{E:notor}
If $\phi\vee \psi$ and $\neg \phi$ are axioms then we can deduce $\psi$.
\begin{prooftree}
\AxiomC{$\phi\vee \psi$}
\AxiomC{$\neg\phi$}
\AxiomC{$[\phi]_1$}
\UnaryInfC{$\phi$}
\RightLabel{\quad$(\neg_{E})$}
\BinaryInfC{$\bot$}
\RightLabel{\quad$(\bot_{E})$}
\UnaryInfC{$\psi$}
\AxiomC{$[\psi]_1$}
\UnaryInfC{$\psi$}
\RightLabel{\quad$(\vee_{I_l})$}
\RightLabel{\quad$(\vee_{E})_1$}
\TrinaryInfC{$\psi$}
\end{prooftree}
\end{example}
\end{frame}

\end{document}