\documentclass[handout]{beamer} 
\title{ITCS 531: Linear Algebra - Vector spaces over fields}
\date{}
\author{Rob Egrot}

\usepackage{amsmath, bbold, bussproofs,graphicx}
\usepackage{mathrsfs}
\usepackage{amsthm}
\usepackage{amssymb}
\usepackage[all]{xy}
\usepackage{multirow}
\usepackage{tikz-cd}


\newtheorem{proposition}[theorem]{Proposition}
\newcommand{\bN}{\mathbb{N}}
\newcommand{\bZ}{\mathbb{Z}}
\newcommand{\bQ}{\mathbb{Q}}
\newcommand{\bR}{\mathbb{R}}
\newcommand{\bP}{\mathbb{P}}
\newcommand{\bC}{\mathbb{C}}
\newcommand{\bF}{\mathbb{F}}
\newcommand{\spa}{\mathrm{span}}

\addtobeamertemplate{navigation symbols}{}{%
    \usebeamerfont{footline}%
    \usebeamercolor[fg]{footline}%
    \hspace{1em}%
    \insertframenumber/\inserttotalframenumber
}
\setbeamertemplate{theorems}[numbered]
\begin{document}

\begin{frame}
\titlepage
\end{frame}

\begin{frame}
\frametitle{What is linear algebra?}
\begin{itemize}
\item Linear algebra is an abstract approach to thinking about Euclidean space.
\vspace{0.3cm}
\item What is a Euclidean space?
\vspace{0.3cm}
\item Examples include the plane in 2 dimensions, and the 3 dimensional grid.
\vspace{0.3cm}
\item These spaces have an \emph{origin} (the point at zero), and two, three, or some other number of dimensions.
\vspace{0.3cm}
\item For each dimension we have an axis, and we can define the position of points by how far along each axis they are.
\vspace{0.3cm}
\item Euclidean space is not curved. So, for example the Euclidean plane is a flat plane in space. It's not curved around the surface of a sphere, or in any other way.
\end{itemize}
\end{frame}

\begin{frame}
\frametitle{What is linear algebra for?}
\begin{itemize}
\item The axioms of linear algebra allow geometric facts to be proved with very clean arguments.
\vspace{0.6cm}
\item By abstracting away from intuitions about physical space we can see the underlying mathematics more clearly.
\vspace{0.6cm}
\item Conversely, by taking an abstract approach we can `see' systems that are not obviously geometric as `spaces in disguise'.
\vspace{0.6cm}
\item We can use geometric reasoning about these `secret spaces'. 
\end{itemize}
\end{frame}

\begin{frame}
\frametitle{Where is linear algebra used?}
\begin{itemize}
\item Linear algebra is used almost everywhere mathematics is used.
\vspace{0.3cm}
\item Physicists need it to understand e.g. quantum systems.
\vspace{0.3cm}
\item Statisticians use it, e.g. principal component analysis.
\vspace{0.3cm}
\item Pure mathematicians like to reformulate problems as linear algebra problems so they can solve them.
\vspace{0.3cm}
\item Computer scientists use linear algebra too, e.g:
\begin{itemize}
\vspace{0.3cm}
\item The Google page rank algorithm.
\vspace{0.3cm}
\item Machine learning, e.g. ANN, SVM.
\vspace{0.3cm}
\item 3D graphics.
\end{itemize}
\end{itemize}
\end{frame}

\begin{frame}
\frametitle{What will we cover on this course?}
\begin{itemize}
\item Since this is a short course we will only scratch the surface.
\vspace{0.3cm}
\item We will introduce the basic abstract definitions and try to understand how they relate to the idea of a space.
\vspace{0.3cm}
\item We will prove some fundamental results using abstract arguments.
\vspace{0.3cm}
\item At the end of the course we will use these abstract results to prove some geometric facts.
\vspace{0.3cm}
\item The idea is that the rigorous approach taken here will give you the background you need to go deeper. 
\end{itemize}
\end{frame}

\begin{frame}
\frametitle{Complex numbers}
\begin{itemize}
\item The complex numbers $\bC$ are obtained by adding a root for the equation $x^2+1 = 0$ to the real numbers $\bR$.
\item This root is a new number called $i$.
\item It turns out that if we add $i$, then we get roots for every other polynomial equation too.
\item So every polynomial over $\bC$ factorizes into linear factors (the \emph{Fundamental Theorem of Algebra}).
\item We can define $\bC$ as the set of all numbers $a+bi$ where $a,b\in \bR$.
\item We have 
\[(a+bi)\pm (c+di) = (a\pm c) +(b\pm d)i\]
and
\[(a+bi)\times(c+di) = ac - bd +(ad+bc)i.\]
\end{itemize}
\end{frame}

\begin{frame}
\frametitle{Complex arithmetic}
\begin{lemma}
Let $\alpha$, $\beta$ and $\gamma$ be complex numbers. Then:
\begin{enumerate}
\item $\alpha + \beta = \beta + \alpha$, and $\alpha\beta = \beta\alpha$ (commutativity).
\item $\alpha + (\beta + \gamma) = (\alpha + \beta) +\gamma$, and $\alpha(\beta\gamma) = (\alpha\beta)\gamma$ (associativity).
\item $0 + \alpha = \alpha$, and $1\alpha = \alpha$ (identities).
\item There is a unique $-\alpha\in\bC$ such that $\alpha + (-\alpha) = 0$ (inverse for addition).
\item If $\alpha\neq 0$ there is a unique $\alpha^{-1}$ such that $\alpha\alpha^{-1} = 1$ (inverse for multiplication).
\item $\alpha(\beta + \gamma) = \alpha\beta + \alpha\gamma$ (distributivity). 
\end{enumerate}
\end{lemma}
\end{frame}

\begin{frame}
\frametitle{Complex arithmetic - proof}
We'll prove  part 5. Part 4 is in the notes and the rest are exercises.
\begin{itemize}
\item Given $\alpha = a + bi$, suppose $(a+bi)(c+di) = 1$. 
\item Then $ac - bd + (ad + bc)i = 1$.
\item So 
\[\tag{$\dagger$}ac - bd = 1,\]
and 
\[\tag{$\ddagger$}ad = - bc.\] 
\item If $b = 0 $ then $\alpha^{-1}=\frac{1}{a}$, so we assume $b\neq 0$. 
\item So we can rewrite $(\ddagger)$ as $c = \frac{-ad}{b}$. 
\item Substituting into $(\dagger)$ gives $d = \frac{-b}{a^2+ b^2}$. 
\item Substituting this value for $d$ into $(\ddagger)$ produces $c = \frac{a}{a^2+b^2}$. 
\item So, we define $\alpha^{-1} = \frac{a-bi}{a^2+b^2}$. 
\end{itemize}
\end{frame}


\begin{frame}
\frametitle{Fields}
We can define division for complex numbers:
\begin{definition}
Let $\alpha,\beta\in\bC$, and suppose $\beta\neq 0$. Then $\frac{\alpha}{\beta} = \alpha\beta^{-1}$.
\begin{itemize}
\item A \textbf{field} is a mathematical structure generalizing the arithmetic of real numbers.
\item Fields have special elements zero and one, have addition and multiplication operations, and also inverses for non-zero elements.
\item E.g. $\bC$ is a field.
\item Fields behave like real numbers, but with important differences - e.g. they can be finite!
\item Abstract linear algebra can be done with arbitrary fields, but we will just use $\bR$ and $\bC$.
\end{itemize}
\end{definition}
\end{frame}

\begin{frame}
\frametitle{Vector spaces over fields}
\begin{definition}
For a field $\bF$, a \emph{vector space over $\bF$} is a set $V$ with operations $+:V\times V\to V$ and $\cdot:\bF\times V\to V$ satisfying:
\begin{enumerate}
\item $u + v = v + u$ for all $u,v\in V$.
\item $u + (v + w) = (u + v) + w$ for all $u,v,w\in V$.
\item $(ab)v = a(bv)$ for all $a,b\in\bF$ and for all $v\in V$.
\item There is a special element $0\in V$ such that $0 + v = v$ for all $v\in V$.
\item For all $v\in V$ there is $w\in V$ such that $v + w = 0$.
\item $1v = v$ for all $v\in V$ (i.e. scalar multiplication by $1$ does not change $v$).
\item $a(u+v) = au + av$ for all $a\in\bF$ and for all $u,v\in V$.
\item $(a+b)v = av + bv$ for all $a,b\in\bF$ and for all $v\in V$.
\end{enumerate}
\end{definition}
\end{frame}

\begin{frame}
\frametitle{Real and complex vector spaces}
\begin{itemize}
\item $+$ is known as \textbf{vector addition}.
\vspace{0.5cm}
\item $\cdot$ is known as \textbf{scalar multiplication}.
\vspace{0.5cm}
\item When $\bF=\bR$ we say $V$ is a \emph{real vector space}.
\vspace{0.5cm}
\item When $\bF=\bC$ we say $V$ is a \emph{complex vector space}.
\vspace{0.5cm}
\item We refer to elements of $V$ as \emph{vectors}, or \emph{points}.
\end{itemize}
\end{frame}

\begin{frame}
\frametitle{Examples of vector spaces}
\begin{itemize}
\item Any field as a vector space over itself. E.g. $\bR$ is a real vector space.
\vspace{0.5cm}
\item $\bR\times \bR$, i.e. the Euclidean plane, is a real vector space.
\vspace{0.5cm}
\item For any $n\in\bN\setminus\{0\}$,\/ $\bF^n$ is a vector space over $\bF$ 
\begin{itemize}
\item Define $(x_1,\ldots,x_n)+(y_1,\ldots,y_n) = (x_1+y_1,\ldots,x_n+ y_n)$, and $a(x_1,\ldots x_n)=(ax_1,\ldots,ax_n)$.
\end{itemize}
\vspace{0.5cm}
\item Let $\bR[x]$ be the set of all polynomials with the variable $x$. So 
\[\bR[x] = \{ a_0 + a_1x + \ldots +a_nx^n: n\in\bN\text{ and } a_i\in\bR\text{ for all }i\}.\]
Then $\bR[x]$ is a vector space over $\bR$.
\end{itemize}
\end{frame}

\begin{frame}
\frametitle{Properties of vector spaces}
\begin{proposition}
Let $V$ be a vector space over $\bF$. Then:
\begin{enumerate}
\item The additive identity $0$ is unique.
\item The additive inverse of $v$ is unique for all $v\in V$ (we call it $-v$).
\item $0v = 0$ for all $v\in V$.
\item $-1v = -v$ for all $v\in V$.
\end{enumerate}
\end{proposition}
\end{frame}

\begin{frame}
\frametitle{Properties of vector spaces - proof}
\begin{proof}\mbox{}
\begin{enumerate}
\vspace{0.5cm}
\item Suppose $0$ and $0'$ are both additive identities for $V$. Then $0 = 0 + 0' = 0'$.
\vspace{0.5cm}
\item Suppose $v+u = 0$ and $v + u' =0$. Then $(v + u) + u' = u'$, and so $(v + u') + u = u'$, which means $u = u'$.
\vspace{0.5cm}
\item $0v = (0+0)v = 0v + 0v$, so $0v +(-0v) = (-0v) + 0v + 0v$, and so $0 = 0v$.
\vspace{0.5cm}
\item Exercise 1.3.
\end{enumerate}
\end{proof}
\end{frame}

\begin{frame}
\frametitle{Subspaces}
\begin{definition}
Let $V$ be a vector space over $\bF$. Then a subset $U$ of $V$ is a \emph{subspace} of $V$ if it has the following properties:
\begin{enumerate}
\item $0\in U$.
\item $u + v \in U$ for all $u,v\in U$ (closure under vector addition). 
\item $au\in U$ for all $a\in \bF$ and for all $u\in U$ (closure under scalar multiplication).
\end{enumerate}
\end{definition}
\end{frame}

\begin{frame}
\frametitle{Another view of subspaces}
\begin{lemma}
If $V$ is a vector space over $\bF$ then $U\subseteq V$ is a subspace of $V$ if and only if it is also a vector space over $\bF$ with the addition and scalar multiplication inherited from $V$.
\end{lemma}
\begin{proof}
\begin{itemize}
\item If $U$ is a vector space with the inherited operations then it must be closed under the inherited operations and contain $0$. 
\item Conversely, if $U$ satisfies the conditions of definition 5 then it automatically satisfies all conditions of definition 3 except (5). 
\item To see that (5) also holds in $U$ note that, by proposition 4(4), given $u\in U$ we have $-u = -1u$, which is in $U$ by definition 5(3).
\end{itemize}
\end{proof}
\end{frame}

\begin{frame}
\frametitle{Sums of subspaces}
\begin{definition}
Given subspaces $U_1,\ldots,U_n$ of $V$, the \emph{sum} $U_1+\ldots +U_n$ is the smallest subspace of $V$ containing $\bigcup_{i=1}^n U_i$.
\end{definition}

\begin{lemma}
If $U_1,\ldots,U_n$ are subspaces of $V$, then 
\[U_1+\ldots +U_n = \{u_1+\ldots +u_n: u_i\in U_i\text{ for all }i\in\{1,\ldots,n\}\}.\]
\end{lemma}
\begin{proof}
\begin{itemize}
\item $\{u_1+\ldots +u_n: u_i\in U_i\text{ for all }i\in\{1,\ldots,n\}\}$ contains $\bigcup_{i=1}^n U_i$ because $u_i = 0+\ldots +0 + u_i + 0 +\ldots + 0$ for all $u_i\in U_i$. 
\item It is a subspace by the definition of a vector space. 
\item It must be the smallest subspace containing $\bigcup_{i=1}^n U_i$, because any such subspace must be closed under vector addition.
\end{itemize} 
\end{proof}
\end{frame}

\begin{frame}
\frametitle{Direct sums}
\begin{definition}
If $U_1,\ldots,U_n$ are subspaces of $V$, then the sum $U_1+\ldots +U_n$ is a \emph{direct sum} if, for all $u\in U_1+\ldots +U_n$, there is exactly one choice of $\{u_1,\ldots, u_n\}$ such that $u_i\in U_i$ for all $i$ and $u = u_1+\ldots +u_n$. In this case we write $U_1\oplus\ldots\oplus U_n$.
\end{definition}
\vspace{0.5cm}
\begin{itemize}
\item So direct sum is a sum where there is no redundancy. 
\vspace{0.5cm}
\item Every element in a direct sum is formed in exactly one way using the subspaces that make up the sum. 
\end{itemize}
\end{frame}

\begin{frame}
\frametitle{Direct sums - expressing zero}
\begin{lemma}
If $U_1,\ldots,U_n$ are subspaces of $V$, then $U_1+\ldots +U_n$ is a direct sum if and only if there is exactly one choice of $\{u_1,\ldots, u_n\}$ such that $u_i\in U_i$ for all $i$ and $0 = u_1+\ldots +u_n$. 
\end{lemma}
\begin{proof}
\begin{itemize}
\item If $U=U_1+\ldots +U_n$ is a direct sum, then by definition there is only one way to express $0$ (i.e. $0= 0+\ldots + 0$). 
\item Conversely, suppose there is only one way to express $0$.
\item Let $u\in U$, and suppose $u = u_1+\ldots + u_n = u'_1+\ldots + u'_n$. Then 
\[0 = u_1+\ldots + u_n - (u'_1+\ldots + u'_n) = (u_1-u_1') + \ldots + (u_n-u_n').\]
So $(u_i- u'_i) = 0$ for all $i$, as there is only one way to express $0$. 
\item Thus $u_i = u_i'$ for all $i$.  
\end{itemize}
\end{proof}
\end{frame}

\begin{frame}
\frametitle{Direct sums - two subspaces}
\begin{lemma}
Let $U$ and $W$ be subspaces of $V$. Then $U+W$ is a direct sum if and only if $U\cap W = \{0\}$.
\end{lemma}
\begin{proof}
\begin{itemize}
\item If there is $v\in U\cap W$ then $v = 0 +v$ and $v = v + 0$, so $U+W$ is not a direct sum. 
\item Conversely, suppose $U\cap W=\{0\}$ and that $v = u + w$ and $v = u' + w'$. 
\item Then $u-u' = w' -w$, and so $u - u'$ and $w'-w$ are both in $U\cap W$, and thus are both $0$. 
\item This implies $u= u'$ and $w = w'$, so $U+W$ is a direct sum.
\end{itemize}
\end{proof}
\end{frame}

\begin{frame}
\frametitle{Direct sums - Example}
\begin{itemize}
\item Let $V = \bR^3$, let $U_1 = \{(2x, 0, z): x,z\in \bR\}$, let $U_2 = \{(0,y,0): y\in \bR\}$, and let $U_3 = \{(0,z,z): z\in\bR\}$. 
\item Then $\bR^3 = U_1 + U_2 + U_3$, because given $(a,b,c)\in\bR^3$ we have 
\[(a,b,c) = (2(\frac{a}{2}), 0, 0) + (0, b-c, 0) + (0,c,c).\]

\item However, $U_1+U_2+U_3$ is not a direct sum as 
\[(0,0,0) = (0,0,1) + (0,1,0) + (0,-1,-1).\] 
\item I.e., $0$ is not uniquely expressible.

\item However, $U_i\cap U_j = \{0\}$ for all $i\neq j$, which indicates that lemma 11 only applies to binary sums.
\end{itemize}
\end{frame}

\begin{frame}
\frametitle{Span}
\begin{definition}
Given a vector space $V$ over $\bF$, and vectors $v_1,\ldots v_n\in V$, we say the \emph{span} of $(v_1,\ldots,v_n)$ is the smallest subspace of $V$ containing $\{v_1,\ldots,v_n\}$. 

By convention we define $\spa() = \{0\}$. If $\spa(v_1,\ldots,v_n) = V$ we say $(v_1,\ldots,v_n)$ \emph{spans} $V$.
\end{definition}

\end{frame}

\begin{frame}
\frametitle{Span}
\begin{lemma}
If $V$ is vector space over $\bF$, and $v_1,\ldots v_n\in V$, then 
\[\spa(v_1,\ldots v_n) = \{a_1v_1+\ldots + a_nv_n: a_i \in \bF\text{ for all }i\}.\]
\end{lemma}
\begin{proof}
\begin{itemize}
\item Let $U = \{a_1v_1+\ldots + a_nv_n: a_i \in \bF$ for all $i\}$. 
\item Then clearly $U\subseteq \spa(v_1,\ldots v_n)$, as $\spa(v_1,\ldots v_n)$ is closed under vector addition and scalar multiplication. 
\item Moreover, $U$ is closed under vector addition and scalar multiplication, so $U$ is a subspace of $V$. 
\item Since $\{v_1,\ldots,v_n\}\subseteq U$, it follows from the definition that $\spa(v_1,\ldots v_n)\subseteq U$. 
\item Thus $U = \spa(v_1,\ldots v_n)$ as required. 
\end{itemize} 
\end{proof}
\end{frame}

\begin{frame}
\frametitle{Linear independence}
\begin{definition}
Let $V$ be vector space over $\bF$, and let $v_1,\ldots v_n\in V$. Then $(v_1,\ldots,v_n)$ is \emph{linearly independent} if whenever $a_1v_1+\ldots + a_n v_n = 0$ we have $a_1=\ldots=a_n=0$. 

If $(v_1,\ldots,v_n)$ is not linearly independent then we say it is \emph{linearly dependent}.
\end{definition}
\end{frame}

\begin{frame}
\frametitle{Examples}
\begin{enumerate}
\item The vectors $(1,0,0)$, $(0,1,0)$ and $(0,0,1)$ are linearly independent and span $\bR^3$ and $\bC^3$.
\vspace{0.5cm}
\item The span of a single vector $v$ is $\{av:a\in \bF\}$. Single vectors are always linearly independent.
\vspace{0.5cm}
\item The vectors $(2,3,1)$, $(1,-1,2)$ and $(7,3,c)$ are linearly independent so long as $c\neq 8$.
\vspace{0.5cm}
\item Every list of vectors containing $0$ is linearly dependent.
\end{enumerate}
\end{frame}
\end{document}