\documentclass[handout]{beamer} 
\title{ITCS 531: Linear Algebra - Inner products on real vector spaces}
\date{}
\author{Rob Egrot}

\usepackage{amsmath, bbold, bussproofs,graphicx}
\usepackage{mathrsfs}
\usepackage{amsthm}
\usepackage{amssymb}
\usepackage[all]{xy}
\usepackage{multirow}
\usepackage{tikz-cd}


\newtheorem{proposition}[theorem]{Proposition}
\newcommand{\bN}{\mathbb{N}}
\newcommand{\bZ}{\mathbb{Z}}
\newcommand{\bQ}{\mathbb{Q}}
\newcommand{\bR}{\mathbb{R}}
\newcommand{\bP}{\mathbb{P}}
\newcommand{\bC}{\mathbb{C}}
\newcommand{\bF}{\mathbb{F}}
\newcommand{\spa}{\mathrm{span}}
\newcommand{\cL}{\mathcal{L}}
\DeclareMathOperator{\nul}{\mathrm{null}}
\DeclareMathOperator{\ran}{\mathrm{ran}}


\addtobeamertemplate{navigation symbols}{}{%
    \usebeamerfont{footline}%
    \usebeamercolor[fg]{footline}%
    \hspace{1em}%
    \insertframenumber/\inserttotalframenumber
}
\setbeamertemplate{theorems}[numbered]
\begin{document}

\begin{frame}
\titlepage
\end{frame}

\begin{frame}
\frametitle{What is an inner product?}
\begin{itemize}
\item  We will work with vector spaces over $\bR$.\vspace{0.5cm} 
\item Everything can be adapted for $\bC$, but the definitions are more complicated.\vspace{0.5cm} 
\item An inner product is a generalization of the dot product. \vspace{0.5cm} 
\item E.g. in $\bR^3$, we have $(a,b,c)\cdot(d,e,f)=ad+be+cf$.  \vspace{0.5cm} 
\item Many geometric ideas for Euclidean spaces can be described using dot products.\vspace{0.5cm} 
\item If a vector space has an inner product, then our geometric intuitions apply to it in some sense. 
\end{itemize}
\end{frame}

\begin{frame}
\frametitle{The definition of an inner product}
\begin{definition}
Let $V$ be a vector space over $\bR$. An \emph{inner product} for $V$ is a function that takes a pair $(u,v)\in V^2$ to a value $\langle u,v\rangle  \in\bR$, satisfying the following properties:
\begin{enumerate}
\item $\langle v,v\rangle  \geq 0$ for all $v\in V$ (positivity).
\item $\langle v,v\rangle  =0\iff v = 0$ (definiteness).
\item $\langle u+v,w\rangle  =\langle u,w\rangle  +\langle v,w\rangle  $ for all $u,v,w\in V$ (additivity in first slot).
\item $\langle \lambda u,v\rangle   = \lambda\langle u,v\rangle  $ for all $\lambda\in \bR$ and for all $u,v\in V$ (homogeneity in first slot).  
\item $\langle u,v\rangle   = \langle v,u\rangle  $ for all $u,v\in V$ (symmetry).
\end{enumerate}
\end{definition}\vspace{0.1cm}
\begin{definition}
A vector space with an inner product is an \emph{inner product space}.  
\end{definition}
\end{frame}

\begin{frame}
\frametitle{Examples of inner products}
\begin{example}\mbox{}
\begin{enumerate}
\item It's easy to check that the dot product as it is usually defined is indeed an inner product.\vspace{0.5cm}
\item It can be shown that the set of continuous real valued functions on the interval $[-1,1]$ is a vector space over $\bR$. We can define an inner product on this space using $\langle f,g\rangle   = \int_{-1}^1 f(x)g(x)dx$.
\end{enumerate}
\end{example}
\end{frame}

\begin{frame}
\frametitle{Basic properties of inner products}
\begin{proposition}\label{P:LA4inner}
The following properties hold in all real inner product spaces:\vspace{0.3cm}
\begin{enumerate}
\item Given $v\in V$, we can define a linear map $\langle -, v\rangle:V\to \bR$ by defining $\langle -, v\rangle(u) = \langle u, v\rangle$ for all $u\in V$.\vspace{0.3cm}
\item $\langle v, 0 \rangle = \langle 0 , v\rangle = 0$ for all $v\in V$.\vspace{0.3cm}
\item $\langle u, v+w\rangle = \langle u, v\rangle + \langle u, w\rangle$ for all $u,v,w\in V$. \vspace{0.3cm}
\item $\langle u,\lambda v\rangle   = \lambda\langle u,v\rangle  $ for all $\lambda\in \bR$ and for all $u,v\in V$
\end{enumerate}
\end{proposition}
\end{frame}

\begin{frame}
\frametitle{Basic properties of inner products - proof}
\begin{proof}
\begin{enumerate}
\item ``\emph{$\langle -, v\rangle(u) = \langle u, v\rangle$ is a linear map}". \begin{itemize}
\item Given $u_1,u_2\in V$ we have $\langle u_1+u_2, v\rangle = \langle u_1, v\rangle + \langle u_2, v\rangle$, by additivity in the first slot. 
\item We also have $\langle \lambda u,v\rangle = \lambda \langle u, v \rangle$ by homogeneity in the first slot.
\end{itemize}\vspace{0.1cm}
\item``\emph{$\langle v, 0 \rangle = \langle 0 , v\rangle = 0$}" \begin{itemize} \item That $\langle 0 , v\rangle = 0 $ follows from part (1) and the fact that $T(0)= 0$ for all linear maps. 
\item We then have $\langle v , 0\rangle = 0$ by symmetry.
\end{itemize}\vspace{0.1cm}
\item``\emph{$\langle u, v+w\rangle = \langle u, v\rangle + \langle u, w\rangle$}". \begin{itemize} \item $\langle u, v+w\rangle = \langle v+w, u\rangle$ by symmetry. 
\item The result follows from additivity and symmetry again.\end{itemize}\vspace{0.1cm}
\item ``\emph{$\langle u,\lambda v\rangle   = \lambda\langle u,v\rangle  $}".
\begin{itemize}
\item Symmetry and homogeneity in the first slot.
\end{itemize}

\end{enumerate}
\end{proof}
\end{frame}

\begin{frame}
\frametitle{Norms}
\begin{itemize}
\item In every real inner product space $V$ we can calculate the value of $\langle v, v\rangle$.
\item This  must be non-negative. 
\item This inspires the following definition:
\end{itemize} 

\begin{definition}
If $V$ is an inner product space, then given $v\in V$, the \emph{norm} of $v$, $||v||$, is defined by
\[\|v\| = \sqrt \langle v, v \rangle.\]
\end{definition} 

\begin{example}
In $\bR^2$ with the usual dot product, the norm of a vector $(a,b)$ is $\sqrt (a^2 + b^2)$. I.e., it is the Euclidean distance of the point $(a,b)$ from the origin.
\end{example}
\end{frame}

\begin{frame}
\frametitle{Basic properties of norms}
\begin{proposition}
The following hold for all real inner product spaces $V$, and for all $v\in V$:
\begin{enumerate}
\item $\|v\| = 0 \iff v = 0$.
\item $\|\lambda v\| = |\lambda|\|v\|$ for all $\lambda\in \bR$.
\end{enumerate}
\end{proposition}
\begin{proof}
\begin{itemize}
\item (1) follows immediately from definiteness of the inner product. 
\item (2) follows from homogeneity in the first slot and proposition \ref{P:LA4inner}(4).
\end{itemize}
\end{proof}
\end{frame}

\begin{frame}
\frametitle{Geometric interpretation of dot product}
\begin{proposition}\label{P:LA4cos}
Given $u,v\in \bR^2\setminus\{0\}$, we have
\[\langle u, v\rangle = \|u\|\|v\|\cos \theta,\]
where $\theta$ is the angle between $u$ and $v$ when these are thought of as arrows beginning at the origin.
\end{proposition}
\end{frame}

\begin{frame}
\frametitle{Geometric interpretation of dot product - proof}
\center \resizebox{1.6cm}{0.08cm}{$\xymatrix{ &\\
& & &\ar[ull]_{u-v} \\
\\
\ar[uuur]^u\ar[uurrr]_v
}$} 
\begin{itemize}
\item In $\bR^2$ the norm of a vector is its length.
\item Law of cosines: $\|u-v\|^2 = \|u\|^2+\|v\|^2 - 2\|u\|\|v\|\cos \theta.$ 

\item Now, $\|u-v\|^2 = \langle u-v, u-v\rangle$, by definition, and
\begin{align*}\langle u-v, u-v\rangle &= \langle u, u-v\rangle - \langle v, u - v \rangle\\
&= \langle u, u \rangle -\langle u , v \rangle - \langle v , u \rangle + \langle v, v \rangle\\
&= \langle u, u \rangle + \langle v, v \rangle - 2\langle u, v \rangle\\
&= \|u\|^2+\|v\|^2 - 2\langle u, v \rangle .\end{align*}

\item So $\langle u, v \rangle = \|u\|\|v\|\cos \theta$ as required. 
\end{itemize}

\end{frame}

\begin{frame}
\frametitle{Orthogonality}
\begin{definition}
If $u$ and $v$ are vectors in an inner product space, then we say $u$ and $v$ are \emph{orthogonal} if $\langle u ,v\rangle =0$.
\end{definition}
\vspace{0.5cm}
\begin{itemize}
\item By proposition \ref{P:LA4cos}, two non-zero vectors in $\bR^2$ are orthogonal if and only if the cosine of the angle between them is 0. \vspace{0.5cm}
\item I.e. if and only if they are perpendicular. \vspace{0.5cm}
\item You can think of `being orthogonal' as a generalization of the concept of `being perpendicular'.
\end{itemize}
\end{frame}

\begin{frame}
\frametitle{Orthogonality of zero}
\begin{lemma}\mbox{}
\begin{enumerate}
\item $0$ is orthogonal to everything.
\item $0$ is the only thing that is orthogonal to itself.
\end{enumerate}
\end{lemma}
\begin{proof}
These follows from proposition \ref{P:LA4inner}(2) and the definiteness of inner products, respectively.
\end{proof}
\end{frame}

\begin{frame}
\frametitle{Pythagoras for inner products}
\begin{proposition}\label{P:LA4pythag}
If $u$ and $v$ are vectors in a real inner product space, then
\[\|u\|^2+\|v\|^2 = \|u+v\|^2 \iff u\text{ and } v \text{ are orthogonal}.\]
\end{proposition}
\begin{proof}
\center \resizebox{2cm}{0.1cm}{$\xymatrix{ & & \\
 \ar[rr]_u\ar[rru]^{u+v}& &\ar[u]_v
}$}
\begin{align*}
\|u+v\|^2 &= \langle u+v, u+v\rangle \\
&= \langle u, u \rangle + \langle u, v \rangle + \langle v,u \rangle + \langle v,v \rangle \\
&= \|u\|^2+\|v\|^2 + 2\langle u, v\rangle. 
\end{align*}
So $\|u\|^2+\|v\|^2 = \|u+v\|^2$ if and only if $\langle u, v\rangle = 0$.
\end{proof}
\end{frame}

\begin{frame}
\frametitle{Some geometric intuition 1}
\begin{itemize}
\item We can think of vectors as arrows. E.g: 

\[\xymatrix{ & & & \\
\ar[rrru]^u\ar[rr]_v & & 
}\] 

\item Geometric intuition says we should be able to turn this into a right angled triangle by drawing some lines. I.e:
\[\xymatrix{ & & & \\
\ar[rrru]^u\ar[rr]_v & &\ar@{.>}[r] &\ar@{.>}[u]
}\]
\item In the picture above we have essentially extended $v$ as far as we need, then added a third line. 
\item `Extending' $v$ is multiplying by some scalar $c$ to get $cv$. 
\end{itemize}
\end{frame}

\begin{frame}
\frametitle{Some geometric intuition 2}
\begin{itemize}
\item The associated vector equation is $u = cv + (u -cv)$.
\[\xymatrix{ & & & \\
\ar[rrru]^u\ar@{.>}[rrr]_{cv} & & &\ar@{.>}[u]_{u-cv}
}\]

\item In an inner product space, the triangle being `right angled' corresponds to the vectors $v$ and $(u-cv)$ being orthogonal.
\item I.e. $\langle v, u-cv\rangle = 0$. 
\item We should always be able to find a scalar value $c$ such that this is true (so long as $u$ and $v$ are non-zero). 

\item From the properties of the inner product we have
\[\langle v, u-cv\rangle = 0 \iff \langle v, u\rangle - c\|v\|^2 = 0.\]
\item So we can take 
\[c = \frac{\langle v, u \rangle}{\|v\|^2}.\]
\end{itemize}
\end{frame}

\begin{frame}
\frametitle{Some geometric intuition 3}
\begin{lemma}\label{L:LA4orth}
Let $V$ be a real inner product space, let $u,v\in V$ and suppose $v\neq 0$. Then there is $w\in V$ such that $\langle v, w\rangle =0$, and $u = cv + w$ for some $c\in \bR$. 
\[\xymatrix{ & & & \\
\ar[rrru]^u\ar@{.>}[rrr]_{cv} & & &\ar@{.>}[u]_{w}
}\]
\end{lemma}
\begin{proof}
Set $c = \frac{\langle v, u \rangle}{\|v\|^2}$ and $w = u - cv$.
\end{proof}
\end{frame}

\begin{frame}
\frametitle{The Cauchy-Schwarz inequality}
\begin{theorem}[Cauchy-Schwarz]
Let $V$ be an inner product space, and let $u,v\in V$. Then
\[|\langle u, v\rangle| \leq\|u\|\|v\|.\]
Moreover, we have equality if and only if $u$ is a scalar multiple of $v$ or vice versa.
\end{theorem}
\end{frame}

\begin{frame}
\frametitle{The Cauchy-Schwarz inequality - proof}
\begin{itemize}
\item If $v$ is zero, then everything is zero, and there is nothing to do. 
\item Let $v\neq 0$ and write $u = cv + w$ where $\langle v, w \rangle = 0$. 
\item By Pythagoras we have $\|u\|^2 = c^2\|v\|^2 +\|w\|^2$.
\item We have $c = \frac{\langle v, u \rangle}{\|v\|^2}$, so 
\[\|u\|^2 =\frac{\langle v, u \rangle^2}{\|v\|^4}\|v\|^2 + \|w\|^2.\]
\item As $\|w\|^2\geq 0$ this implies 
\[\|u\|^2 \geq \frac{\langle v, u \rangle^2}{\|v\|^4}\|v\|^2.\]
\item So $\|u\|\|v\| \geq |\langle u, v\rangle|.$  

\item Note that $|\langle u, v\rangle| =\|u\|\|v\|$ if and only if $\|w\| = 0$, which happens if and only if $w = 0$. I.e. if $u=cv$.
\end{itemize}
\end{frame}

\begin{frame}
\frametitle{Applications of Cauchy-Schwarz}
The Cauchy-Schwarz inequality is extremely useful. Here's a simple application, and we will see more soon.\vspace{0.5cm}
\begin{example}
Let $x_1,\ldots,x_n,y_1,\ldots,y_n \in\bR$. Then, using Cauchy-Schwarz we have
\[|x_1y_1+\ldots +x_ny_n|^2\leq (x_1^2+\ldots + x_n^2)(y_1^2+\ldots +y_n^2).\]
\end{example}
\end{frame}

\begin{frame}
\frametitle{The triangle inequality}
\begin{itemize}
\item It is a basic fact of Euclidean geometry that the length of a side of a triangle is less than the sum of the lengths of the other two sides. 
\item This generalizes to inner product spaces.
\end{itemize}

\begin{proposition}\label{P:LA4tri}
Let $V$ be a real inner product space, and let $u,v\in V$. Then 
\[\|u + v \|\leq \|u\|+\|v\|.\]
Moreover, we have equality if and only if $u$ is a scalar multiple of $v$ or vice versa.
\end{proposition} 
\end{frame}

\begin{frame}
\frametitle{The triangle inequality - proof}
\begin{itemize}
\item Appealing to Cauchy-Schwarz for the inequality marked $*$ we have
\begin{align*}
\|u+v\|^2 &= \langle u+v, u+v \rangle \\
&= \langle u,u \rangle + \langle v,u \rangle + \langle u,v \rangle + \langle v,v \rangle\\
&= \|u\|^2 + \|v\|^2 + 2\langle u, v \rangle\\
*&\leq \|u\|^2 + \|v\|^2 + 2\|u\|\|v\| \\
&= (\|u\|+ \|v\|)^2.
\end{align*}
\item So $\|u+v\|\leq \|u\|+ \|v\|$ as claimed.
\item We have equality if and only if $\|u\|\|v\|= \langle u, v \rangle$. 
\item By Cauchy-Schwarz this happens if and only if one of $u$ or $v$ is a scalar multiple of the other.
\end{itemize}
\end{frame}

\begin{frame}
\frametitle{The parallelogram equality}
Now lets use what we have proved about inner product spaces to prove a less obvious fact about plain geometry.
\vspace{1cm}
\begin{proposition}
In a parallelogram, the sum of the squares of the lengths of the diagonals equals the sum of the squares of the sides.
\end{proposition}
\end{frame}

\begin{frame}
\frametitle{The parallelogram equality - proof}
\begin{itemize}
\item Expressed in terms of vectors, a parallelogram has form
\center \resizebox{3cm}{0.3cm}{$\xymatrix{ &\ar[rrr]^u & & &\\
\\
\ar[ruu]^v\ar[rrr]_u & & &\ar[ruu]_v
}$}
\item The diagonals are given by $u-v$ and $u+v$. Now
\begin{align*}
&\|u+v\|^2 + \|u-v\|^2\\ 
=& \langle u+v, u+v \rangle + \langle u-v, u-v \rangle\\
=& \|u\|^2 + \|v\|^2 + 2\langle u, v \rangle +\|u\|^2 + \|v\|^2 -  2\langle u,v \rangle\\
=& 2(\|u\|^2 + \|v\|^2).
\end{align*}
\end{itemize} 
The identity $\|u+v\|^2 + \|u-v\|^2 = 2(\|u\|^2 + \|v\|^2)$
is called the \emph{parallelogram equality}. 
\end{frame}

\end{document}