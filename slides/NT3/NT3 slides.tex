\documentclass[handout]{beamer} 
\title{ITCS 531: Number Theory 3 - Primality testing}
\date{}
\author{Rob Egrot}

\usepackage{amsmath, bbold, bussproofs,graphicx}
\usepackage{mathrsfs}
\usepackage{amsthm}
\usepackage{amssymb}
\usepackage[all]{xy}
\usepackage{multirow}
\usepackage{tikz-cd}


\newtheorem{Def}{Definition}
\newtheorem{Lem}{Lemma}
\newtheorem{Thm}{Theorem}
\newtheorem{Cor}{Corollary}
\newtheorem{Ex}{Example}
\newtheorem{Prop}{Proposition}
\newtheorem{Fact}{Fact}
\newtheorem{Que}{Question}
\newtheorem{proposition}[theorem]{Proposition}{\bfseries}{\itshape}

\newcommand{\HCF}{\mathbf{HCF}}
\newcommand{\bN}{\mathbb{N}}
\newcommand{\bZ}{\mathbb{Z}}
\newcommand{\bQ}{\mathbb{Q}}
\newcommand{\bR}{\mathbb{R}}
\newcommand{\bP}{\mathbb{P}}

\addtobeamertemplate{navigation symbols}{}{%
    \usebeamerfont{footline}%
    \usebeamercolor[fg]{footline}%
    \hspace{1em}%
    \insertframenumber/\inserttotalframenumber
}
\setbeamertemplate{theorems}[numbered]
\begin{document}

\begin{frame}
\titlepage
\end{frame}

\begin{frame}
\frametitle{$\bZ_n$}
\begin{itemize}
\item In the previous class we saw that arithmetic modulo $n$ `makes sense'.
\vspace{0.5cm} 
\item I.e. we can define operations of addition, subtraction and multiplication on equivalence classes modulo $n$ for all $n\in\bN \setminus\{0\}$. 
\vspace{0.5cm} 
\item The definition below is used to pick out the number system defined by looking at integers mod $n$. 
\vspace{0.5cm} 
\begin{definition}[$\bZ_n$]
If $n\in \bN\setminus\{0\}$ then $\bZ_n$ is the set of integers mod $n$.
\end{definition}
\end{itemize}
\end{frame}

\begin{frame}
\frametitle{Multiplicative inverses}
\begin{itemize}
\item In standard arithmetic over $\bR$, every number except $0$ has an inverse under multiplication. 
\vspace{0.5cm} 
\item That is, for all $x\in \bR\setminus\{0\}$ there is $y\in \bR\setminus\{0\}$ with $xy =1$. 
\vspace{0.5cm} 
\item We write $x^{-1}$ or $\frac{1}{x}$ for the multiplicative inverse of $x$. 
\vspace{0.5cm} 
\item In the integers $\bZ$, only the numbers $1$ and $-1$ have an inverse. 
\vspace{0.5cm} 
\item In $\bZ_n$ this is not usually true.
\end{itemize}
\end{frame}

\begin{frame}
\frametitle{Inverses in $\bZ_n$}
\begin{definition}[Modular multiplicative inverse] 
For $a\in \bZ$ we define $b\in\bZ$ to be the multiplicative inverse, or just the \emph{inverse}, of $a \mod n$ if  $ab \equiv_n 1$. 


\end{definition}
\vspace{0.5cm} 
\begin{itemize}
\item We write $a^{-1}$ for the multiplicative inverse of $a$ (when it exists!).
\vspace{0.5cm} 
\item Soon we will prove a result that tells us exactly when integers have an inverse mod $n$.
\end{itemize}

\end{frame}

\begin{frame}
\frametitle{Coprimality}
\begin{definition}[Coprime]
Integers $a$ and $b$ are \emph{coprime} if their highest common factor ($\HCF$)  is $1$
\end{definition}

\begin{lemma}\label{L:div3}
Let $a,b,c\in\bZ$, let $a|bc$, and let $a$ and $b$ be coprime. Then $a|c$.
\end{lemma}
\begin{proof}
This is exercise 3.1.
\end{proof}
\vspace{0.5cm} 
\begin{itemize}
\item Now we have all we need to prove the first important result of the class.
\end{itemize}

\end{frame}

\begin{frame}
\frametitle{When do inverses exist in $\bZ_n$?}
\begin{proposition}\label{P:inv}
\begin{itemize}
\item Let $a\in \bZ$ and let $n\in\bN\setminus\{0\}$. 
\vspace{0.2cm} 
\item Then $a$ has multiplicative inverse $\mod n$ if and only if $a$ and $n$ are coprime. 
\vspace{0.2cm} 
\item Moreover, the multiplicative inverse of $a\mod n$ is unique in $\bZ_n$, whenever it exists. 
\end{itemize}
\end{proposition}

\end{frame}

\begin{frame}
\frametitle{Proof of proposition \ref{P:inv} part 1}
\begin{itemize}
\item This proof has three parts. 
\vspace{0.2cm} 
\begin{itemize}
\item We must show that \emph{if} $a$ and $n$ are coprime, \emph{then} $a$ has an inverse in $\bZ_n$. 
\vspace{0.2cm} 
\item Also \emph{if} $a$ has an inverse in $\bZ_n$, \emph{then} $a$ and $n$ are coprime. 
\vspace{0.2cm} 
\item Finally, if $b$ and $c$ are both inverses to $a$ mod $n$, then $b\equiv_n c$.
\vspace{0.2cm} 
\end{itemize}
\item Suppose $a$ and $n$ are coprime. 
\vspace{0.2cm} 
\item Since $a$ and $n$ are coprime, it follows from B\'ezout's identity that there are $x,y\in\bZ$ with $xa + yn = 1$. 
\vspace{0.2cm} 
\item So $xa-1 = -yn$, but this means that $xa \equiv_n 1$ by definition. 
\vspace{0.2cm} 
\item So $a$ has an inverse in $\bZ_n$ as required.
\end{itemize}
\end{frame}

\begin{frame}
\frametitle{Proof of proposition \ref{P:inv} part 2}
\begin{itemize}
\item Now suppose that $a$ has an inverse, and call it $x$. 
\vspace{0.2cm} 
\item Then we have $xa \equiv_n 1$.
\vspace{0.2cm} 
\item I.e. there is $y$ with $xa - 1 = yn$. 
\vspace{0.2cm} 
\item We can rewrite this as $xa - yn = 1$. 
\vspace{0.2cm} 
\item Suppose $d|a$ and $d|n$. 
\vspace{0.2cm} 
\item Then $d|(ax-yn)$, and so $d|1$. 
\vspace{0.2cm} 
\item The only way this can be true is if $d=\pm 1$. 
\vspace{0.2cm} 
\item This means $HCF(a,n)=1$, and so $a$ and $n$ are coprime.
\end{itemize}
\end{frame}

\begin{frame}
\frametitle{Proof of proposition \ref{P:inv} part 3}
\begin{itemize}
\item Finally, if an inverse to $a$ exists then we have just shown that $(a,n)$ must be coprime. 
\vspace{0.2cm} 
\item Let $ab\equiv_n 1$ and $ac\equiv_n 1$. 
\vspace{0.2cm} 
\item Then there are $k,l\in\bZ$ with $ab - 1 = k n$ and $ac - 1 = ln$. 
\vspace{0.2cm} 
\item So $a(b-c) = (k-l)n$. 
\vspace{0.2cm} 
\item We obviously have $a|a(b-c)$.
\vspace{0.2cm} 
\item So by lemma \ref{L:div3} we must have $a|(k-l)$. 
\vspace{0.2cm} 
\item So $b-c = \frac{k-l}{a}n$, and $\frac{k-l}{a}\in\bZ$.
\vspace{0.2cm} 
\item So $b\equiv_n c$. 
\end{itemize}
\end{frame}

\begin{frame}
\frametitle{Investigating prime factorization}
\begin{itemize}
\item Now we know the basics of modular arithmetic, we can start to seriously study prime numbers and prime factorizations. 
\vspace{0.2cm} 
\item The difficulty of finding the prime factors of large numbers is the basis for much of modern cryptography (i.e. RSA). 
\vspace{0.2cm} 
\item An old result about prime numbers known as \emph{Fermat's little theorem} will be important. 
\vspace{0.2cm} 
\item This neat theorem gives us a kind of detector for numbers which are not prime (i.e. composite numbers).
\vspace{0.2cm} 
\item With some ingenuity this can be turned into a powerful probabilistic method for testing whether a number is prime. 
\vspace{0.2cm} 
\item First we will need another small technical lemma. 
\end{itemize}
\end{frame}

\begin{frame}
\frametitle{Injective multiplication}

\begin{lemma}\label{L:distinct}
Let $a\in\bZ\setminus\{0\}$ and $n\in\bN\setminus\{0\}$ be coprime. Then, for all $b,c\in \bZ$, if $ab \equiv_n ac$, we have $b\equiv_n c$.
\end{lemma}
\begin{proof}
\begin{itemize}
\item Since $a$ and $n$ are coprime, by proposition \ref{P:inv} we know $a$ has a multiplicative inverse $a^{-1}$ (mod $n$). 
\vspace{0.5cm} 
\item So $a^{-1}ab \equiv_n a^{-1}ac$. 
\vspace{0.5cm}
\item And so $b\equiv_n c$ by definition of the inverse. 
\end{itemize}
\end{proof}
\end{frame}

\begin{frame}
\frametitle{Fermat's little theorem}
\begin{theorem}[Fermat's little theorem]\label{T:fermat}
If $p$ is prime then $a^{p-1}\equiv_p 1$ whenever $a$ and $p$ are coprime.
\end{theorem}
\begin{proof}
\begin{itemize}
\item By lemma \ref{L:distinct} we have 
\scriptsize
\[\{1,2,3,\ldots,p-1\}=\{a \mod p,2a \mod p,3a \mod p,\ldots, (p-1)a \mod p\}.\] 
\normalsize
\item So, by multiplying 
\[\tag{$\dagger$}(p-1)! \equiv_p a^{p-1}(p-1)!.\] 
\item Now, since $p$ is prime, it follows that $p$ cannot divide $(p-1)!$. 
\item So $p$ and $(p-1)!$ are coprime. 
\item By proposition \ref{P:inv}  it follows that $(p-1)!$ has an inverse modulo $p$. 
\item Multiplying $(\dagger)$ by this inverse gives $a^{p-1}\equiv_p 1$ as required. 
\end{itemize}
\end{proof}
\end{frame}

\begin{frame}
\frametitle{Primality testing with Fermat's little theorem}
\begin{itemize}
\item Fermat's little theorem gives us an efficient way we can test whether a number is prime. 
\vspace{0.2cm} 
\item Given $n\in \bN$ we pick $a$ with $1<a<n$, then calculate $a^{n-1} \mod n$. 
\vspace{0.2cm} 
\item If this is not 1 then $n$ is not prime, by Fermat's little theorem.
\begin{itemize}
\item As if $n$ is prime then $a$ would automatically be coprime with $n$.
\end{itemize} 
\vspace{0.2cm} 
\item However, $a^{n-1} \equiv_n 1$ does not imply that $n$ is prime. 
\vspace{0.2cm} 
\item This is because Fermat's little theorem only tells us that \emph{if} $p$ is prime \emph{then} $a^{p-1}\equiv_p 1$. 
\vspace{0.2cm} 
\item It doesn't say that if $p$ is \emph{not} prime then $a^{p-1}\not\equiv_p 1$. 
\vspace{0.2cm} 
\item For example, $341 = 11\times31$, but $2^{340} \equiv_{341} 1$. 
\end{itemize}
\end{frame}

\begin{frame}
\frametitle{Evidence for primality}
\begin{itemize}
\item Passing Fermat's test does give us evidence that a number is prime, due to the following result.
\end{itemize}
\begin{lemma}\label{L:half}
Let $n\in \bN$ and suppose there is $1\leq a<n$ such that $a$ is coprime with $n$ and $a^{n-1}\not\equiv_n 1$. Then the modular inequality $b^{n-1}\not\equiv_n 1$ must hold for at least half the natural numbers $b$ less than $n$. 
\end{lemma} 
\begin{proof}
\begin{itemize}
\item Suppose $b<n$ and $b$ passes Fermat's test (i.e. $b^{n-1}\equiv_n 1$). 
\item Then $ab$ fails Fermat's test, because $(ab)^{n-1} = a^{n-1}b^{n-1}\equiv_n a^{n-1}\not\equiv 1$. 
\item Moreover, if $ab\equiv_n ac$ for $1<b,c< n$ then $b=c$ (lemma \ref{L:distinct}). 
\item So every element that passes Fermat's test has a partner that doesn't, and these partners are all distinct.
\item So there are at least as many elements that fail as that pass. 
\end{itemize}
\end{proof}
\end{frame}

\begin{frame}
\frametitle{Probabilistic primality testing}
\begin{itemize} 
\item If there is at least one value $a$ that is coprime with $n$ with $a^{n-1}\not\equiv_n 1$, this gives us a good test for determining whether a number $n$ is prime: \vspace{0.2cm} 
\begin{itemize}
\item We repeat Fermat's test $k$ times with different random numbers $a$ with $1<a<n$. 
\vspace{0.2cm} 
\item If the test fails for any $a$ we conclude with certainty that $n$ is not prime (by the little theorem). 
\vspace{0.2cm} 
\item If every test is passed we conclude that the probability that $n$ is not prime must be at most $\frac{1}{2^k}$.
\vspace{0.2cm}  
\item Because, if $n$ is not prime, every $a$ provides at least a 50\% chance of making $n$ fail the test. 
\vspace{0.2cm} 
\item So, for high confidence just choose large $k$.
\vspace{0.2cm} 
\item This test is always correct when it says a number is composite, but it occasionally says a number is prime when it is not.
\end{itemize}
\end{itemize}

\end{frame}

\begin{frame}
\frametitle{Problems with Carmichael numbers}
\begin{itemize}
\item There is a small problem with this. 
\item Lemma \ref{L:half} relies on the existence of at least one $a$ that is coprime with $n$ and fails Fermat's test (i.e. $a^{n-1}\not\equiv_n 1$). 
\item Unfortunately, there are composite numbers where every coprime $a$ passes Fermat's test. 
\item These numbers are called \emph{Carmichael numbers}. 
\item The smallest Carmichael number is 561. 
\item This is not prime as $561 = 3\times 11 \times 17$, but for every $1<a<561$ that is coprime to $561$ we have $a^{560}\equiv_{561} 1$. 
\item So our probability calculation from before is not correct. 
\item There are an infinite number of Carmichael numbers, but they are rare, so Fermat's test works most of the time. 
\item There are also more advanced methods, like the Rabin-Miller test.  
\end{itemize}
\end{frame}

\begin{frame}
\frametitle{Roots of polynomials in $\bZ_n$}
\begin{itemize}
\item  Remember that a polynomial with variable $x$ and degree $n$ is a function 
\[a_0 +a_1x +a_2x^2+\ldots +a_nx^n,\] 
where $a_0,\ldots,a_n$ are fixed parameters. 
\item A polynomial over $\bR$ can have, at most, the same number of real roots as its degree 
\begin{itemize} 
\item A \emph{root} of a single variable function $f$ is a value $x$ such that $f(x)=0$.
\end{itemize} 
\item The Fundamental Theorem of Algebra says that polynomial over $\bR$ has exactly the number of complex roots as its degree, but this is not in the scope of this course. 
\item We will show soon that the limit on the number of roots of a polynomial we have just described also applies to polynomials over $\bZ_p$, when $p$ is prime.
\end{itemize}
\end{frame}

\begin{frame}
\frametitle{A special case of polynomial division}
\begin{itemize}
\item We will use this following lemma.
\end{itemize}
\begin{lemma}\label{L:Lagdiv}
If $x,y\in\bR$ then 
\[\frac{x^n-y^n}{x-y} = x^{n-1} + x^{n-2}y + x^{n-3}y^2 + \ldots + x y^{n-2} + y^{n-1}.\]
\end{lemma}
\begin{proof}
Direct calculation of \[(x-y)(x^{n-1} + x^{n-2}y + x^{n-3}y^2 + \ldots + x y^{n-2} + y^{n-1})\] shows it is equal to $x^n-y^n$.
\end{proof}
\begin{itemize}
\item I.e. the polynomial $x-y$ divides the polynomial $x^n-y^n$.
\item Note that this works even if $x=y$.
\item For polynomials, potential division by zero makes sense.
\end{itemize}
\end{frame}

\begin{frame}
\frametitle{Polynomials over $\bZ_p$}
\begin{itemize}
\item Let $f$ be a polynomial over $\bZ$ of degree $n$. I.e. 
\[f(x)= a_0 + a_1 x + a_2 x^2 +\ldots + a_n x^n,\]
where $a_i\in\bZ$ for all $i$ and $a_n\neq 0$.
\vspace{0.5cm} 
\item Let $p$ be a prime number. 
\vspace{0.5cm} 
\item We define $f_p(x) = a'_0 + a'_1 x + a'_2 x^2 +\ldots + a'_n x^n$ where each $a'_i = a_i\mod p$ for all $i$. 
\vspace{0.5cm} 
\item So $f_p$ is $f$ converted to being a polynomial over $\bZ_p$. 
\vspace{0.5cm} 
\item E.g. if $f(x) = 8 + 14x +3x^2$, then $f_5(x) = 3 + 4x + 3x^2$.
\end{itemize}
\end{frame}

\begin{frame}
\frametitle{Lagrange's theorem on polynomial roots}
\begin{theorem}[Lagrange]
Let $p$ be prime, let $f(x) = a_0 + a_1x +\ldots + a_m x^m$ be a polynomial over $\bZ$, and let $f_p$ be as above. Suppose the degree of $f_p$ is $n$. Then, unless every coefficient of $f_p$ is zero, $f_p$ has at most $n$ distinct roots modulo $p$. 
\end{theorem}

\end{frame}

\begin{frame}
\frametitle{Lagrange's theorem on polynomial roots - proof 1}

\begin{itemize}
\item You don't need to remember this proof. 
\vspace{0.5cm} 
\item First note that the degree of $f_p$ must be less than or equal to the degree of $f$, i.e. we must have $n\leq m$. 
\vspace{0.5cm} 
\item We induct on $n$, the degree of $f_p$. 
\vspace{0.5cm} 
\item Remember we're trying to show $f_p$ is either zero or has at most $n$ roots (mod $p$).
\vspace{0.5cm} 
\item The result is clearly true when $n=1$, because here we have $f_p = a'_0 + a'_1x$, and the root occurs when $x \equiv_p -a'_0a'^{-1}_1$. 
\end{itemize}
\end{frame}

\begin{frame}
\frametitle{Lagrange's theorem on polynomial roots - proof 2}
\begin{itemize}
\item Suppose now that the result is true for all $n\leq k$. 
\vspace{0.2cm} 
\item Let the degree of $f_p$ be $k+1$. 
\vspace{0.2cm} 
\item Suppose that $f_p$ has a root $b$ modulo $p$. I.e. $f_n(b)\equiv_p 0$. 
\vspace{0.2cm} 
\item If such a root does not exist then we are done, as $0\leq n$. 
\vspace{0.2cm} 
\item Consider the polynomial 
\[f_p(x)- f_p(b) = a'_1(x-b) + a'_2(x^2-b^2) +\ldots + a'_{k+1}(x^{k+1}-b^{k+1}).\]
\item By lemma \ref{L:Lagdiv}, $(x-b)$ divides $(x^l-b^l)$ for all $1\leq l\leq k+1$, so we can define a polynomial $g(x)=\frac{f_p(x) - f_p(b)}{x-b}$ over $\bZ_p$. 
\end{itemize}
\end{frame}

\begin{frame}
\frametitle{Lagrange's theorem on polynomial roots - proof 3}

\begin{itemize}
\item By definition of $g$ we have $f_p(x)-f_p(b) = (x-b)g(x)$. 
\vspace{0.2cm} 
\item Moreover, $g$ has degree at most $k$. 
\vspace{0.2cm} 
\item Let $c$ be a root of $f_p(x)$ modulo $p$. 
\vspace{0.2cm} 
\item Then, setting $x=c$ we get $0 \equiv_p (c-b)g(c)$, as $b$ is also a root of $f_p$ mod $p$. 
\vspace{0.2cm} 
\item I.e. $p|(c-b)g(c)$. 
\vspace{0.2cm} 
\item Since $p$ is prime this means either 
\begin{enumerate}[a)]
\item $p|(b-c)$, which happens if and only $c\equiv_p b$, or
\item $p|g(c)$, in which case $c$ is a root of $g(x)$ modulo $p$.
\end{enumerate} 
\vspace{0.2cm} 
\item But, by the inductive hypothesis, there are at most $k$ roots of $g$ modulo $p$. 
\vspace{0.2cm} 
\item So there at most $k+1$ roots of $f_p$ modulo $p$. 
\end{itemize}
\end{frame}

\end{document}