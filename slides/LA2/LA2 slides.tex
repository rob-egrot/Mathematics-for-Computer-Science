\documentclass[handout]{beamer} 
\title{ITCS 531: Linear Algebra - Dimension}
\date{}
\author{Rob Egrot}

\usepackage{amsmath, bbold, bussproofs,graphicx}
\usepackage{mathrsfs}
\usepackage{amsthm}
\usepackage{amssymb}
\usepackage[all]{xy}
\usepackage{multirow}
\usepackage{tikz-cd}


\newtheorem{proposition}[theorem]{Proposition}
\newcommand{\bN}{\mathbb{N}}
\newcommand{\bZ}{\mathbb{Z}}
\newcommand{\bQ}{\mathbb{Q}}
\newcommand{\bR}{\mathbb{R}}
\newcommand{\bP}{\mathbb{P}}
\newcommand{\bC}{\mathbb{C}}
\newcommand{\bF}{\mathbb{F}}
\newcommand{\spa}{\mathrm{span}}

\addtobeamertemplate{navigation symbols}{}{%
    \usebeamerfont{footline}%
    \usebeamercolor[fg]{footline}%
    \hspace{1em}%
    \insertframenumber/\inserttotalframenumber
}
\setbeamertemplate{theorems}[numbered]
\begin{document}

\begin{frame}
\titlepage
\end{frame}

\begin{frame}
\frametitle{Bases}
\begin{itemize}
\item Think about the Euclidean plane $\bR^2$. 
\item In $\bR^2$, every vector is defined by coordinates $(x,y)$.
\item I.e. every vector in $\bR^2$ can be written as a sum $x(1,0)+y(0,1)$ of the vectors $(1,0)$ and $(0,1)$.
\item These vectors $(1,0)$ and $(0,1)$ are special, as they form a minimal set that generates the whole space.
\item We can generalize this idea.
\end{itemize}
\begin{definition}
If $V$ is a vector space, then a \emph{basis} for $V$ is a linearly independent set that spans $V$.
\end{definition}
\end{frame}

\begin{frame}
\frametitle{Expressing vectors with a basis}
\begin{lemma}
Let $V$ be a vector space over $\bF$. Let $v_1,\ldots,v_n\in V$. Then $(v_1,\ldots,v_n)$ is a basis for $V$ if and only if every element $u$ can be expressed as $a_1v_1+\ldots + a_nv_n$, for some unique $\{a_1,\ldots,a_n\}\subseteq \bF$. 
\end{lemma}
\begin{proof}
\begin{itemize}
\item Suppose $(v_1,\ldots,v_n)$ is a basis for $V$. 
\item Given $u\in V$, we have $u= a_1v_1+\ldots + a_nv_n$ for some $\{a_1,\ldots,a_n\}\subseteq \bF$ as$(v_1,\ldots,v_n)$ spans $V$. 
\item If $u = a_1v_1+\ldots + a_nv_n = b_1v_1+\ldots + b_nv_n$. Then $0 = (a_1-b_1)v_1+\ldots +(a_n-b_n)v_n$, so $a_i=b_i$ for all $i\in \{1,\ldots,n\}$, as $(v_1,\ldots,v_n)$ is linearly independent.
\item Conversely, if $(v_1,\ldots,v_n)$ satisfies the two stated properties then it is certainly a linearly independent spanning set.
\end{itemize}
\end{proof}
\end{frame}

\begin{frame}
\frametitle{The importance of bases}
\begin{itemize}
\item Bases are extremely important in the study of vector spaces. 
\vspace{0.3cm}
\item Like the prime numbers generate the integers, a vector space is generated by a basis. 
\vspace{0.3cm}
\item In other words, if you have a basis, then you know the space. 
\vspace{0.3cm}
\item There are natural questions we can ask about bases. 
\vspace{0.3cm}
\item Does every vector space have one? Can a space have more than one? 
\vspace{0.3cm}
\item If a space has two (or more) possible bases, does it matter what basis we choose? 
\vspace{0.3cm}
\item We will see answers to these questions soon.
\end{itemize}
\end{frame}

\begin{frame}
\frametitle{Removing redundant vectors from a spanning list}
\begin{lemma}\label{L:LA2tech}
Let $V$ be a vector space over $\bF$, and let $v_1,\ldots v_n\in V$. Suppose that $(v_1,\ldots,v_n)$ is linearly dependent. Then there is $j\in\{1,\ldots,n\}$ such that:
\begin{enumerate}
\item $v_j \in \spa(v_1,\ldots,v_{j-1})$.
\item $\spa(v_1,\ldots , v_{j-1}, v_{j+1},\ldots, v_n) = \spa(v_1,\ldots,v_n)$.
\end{enumerate} 
\end{lemma}
\begin{proof}
\begin{itemize}
\item Since $(v_1,\ldots,v_n)$ is linearly dependent there are $a_1,\ldots,a_n\in \bF$ with $a_1v_1+\ldots + a_nv_n = 0$ and at least one $a_i\neq 0$. 
\item Let $j$ be the largest value such that $a_j\neq 0$. 
\item Then $a_1v_1 + \ldots +a_j v_j = 0$, and, since $a_j\neq 0$ we can rewrite this as $v_j = -\frac{a_1}{a_j}v_1-\ldots -\frac{a_{j-1}}{a_j}v_{j-1}$. 
\item This proves (1), and (2) follows easily from (1). 
\end{itemize}
\end{proof}
\end{frame}

\begin{frame}
\frametitle{Linearly independent lists cannot be bigger than spanning lists}
\begin{proposition}\label{P:LA2length}
Let $V$ be a vector space over $\bF$, let $(u_1,\ldots,u_k)$ be linearly independent, and let $(v_1,\ldots,v_n)$ span $V$. Then $k\leq n$. 
\end{proposition}
\vspace{0.5cm}
Proof.
\begin{itemize}
\item We will use lemma \ref{L:LA2tech} multiple times.
\vspace{0.5cm}
\item The idea is to replace elements of $(v_1,\ldots,v_n)$ with different elements of $(u_1,\ldots,u_k)$, till we have used all the elements of $(u_1,\ldots,u_k)$.
\vspace{0.5cm}
\item Being able to do this implies that $k\leq n$.
\end{itemize}
\end{frame}

\begin{frame}
\frametitle{Proof continued}
\begin{itemize}
\item Consider the list $(u_1,v_1,\ldots,v_n)$.
\item By lemma \ref{L:LA2tech} there is an element $w_1$ of $(u_1,v_1,\ldots,v_n)$ such that $w_1$ is in the span of the part of the list $(u_1,v_1,\ldots,v_n)$ that precedes it.
\item Obviously we can't have $w_1 = u_1$, so $w_1$ is in $(v_1,\ldots,v_n)$.
\item Let e.g. $(v_1,\ldots,v_n)\setminus \{w_1\}$ be $(v_1,\ldots,v_n)$ with $w_1$ removed.
\item Applying lemma \ref{L:LA2tech} to $(u_2,u_1,v_1,\ldots,v_n)\setminus\{w_1\}$ we get $w_2$ in the span of the part of the list $(u_1,v_1,\ldots,v_n)$ that precedes it.
\item $w_2$ can't be in $(u_2,u_1)$ as this is linearly independent.
\item Apply lemma \ref{L:LA2tech} to $(u_3,u_2,u_1,v_1,\ldots,v_n)\setminus\{w_1,w_2\}$ to get $w_3$ and so on. These lists all span $V$.
\item In the end we get $(u_k,\ldots,u_1,v_1,\ldots,v_n)\setminus\{w_1,\ldots,w_k\}$, and each $w_i\in (v_1,\ldots,v_n)$.
\item Thus $k\leq n$ as claimed.
\end{itemize}
\end{frame}

\begin{frame}
\frametitle{Finite dimensional spaces}
\begin{definition}
A vector space $V$ is \emph{finite dimensional} if it contains a finite spanning list $(v_1,\ldots,v_n)$. If $V$ is not finite dimensional then it is \emph{infinite dimensional}. 
\end{definition}
\end{frame}

\begin{frame}
\frametitle{Obtaining bases from spanning/linearly independent lists}
\begin{theorem}\label{T:LA2basis}
Let $V$ be a vector space over $\bR$. Then:
\begin{enumerate}
\item If $s = (v_1,\ldots,v_n)$ spans $V$, then $s$ can be reduced to a basis for $V$.
\item If $V$ is finite dimensional, and if $t=(u_1,\ldots,u_k)$ is linearly independent in $V$, then $t$ can be extended to a basis for $V$.
\end{enumerate}
\end{theorem}
\begin{proof}
\begin{itemize} 
\item For (1), we apply lemma \ref{L:LA2tech} as many times as we can. The resulting list has the same span as the original, but is linearly independent as we can't apply the lemma again. 

\item For (2), since $V$ is finite dimensional it has a spanning list $(w_1,\ldots,w_m)$. Now, the list $(u_1,\ldots,u_k,w_1,\ldots,w_m)$ also spans $V$, and so, by (1), reduces to a basis for $V$. \item This reduction does not remove any elements of $t$, as $t$ is linearly independent.
\end{itemize}
\end{proof}
\end{frame}

\begin{frame}
\frametitle{The existence of bases}
\begin{corollary}
Every finite dimensional vector space has a basis.
\end{corollary}
\begin{proof}
Just reduce the finite spanning list to a basis. 
\end{proof}
\vspace{1cm}
\begin{itemize}
\item Every \emph{infinite} dimensional vector space also has a basis, but this proof is more difficult.
\vspace{0.5cm}
\item We need an infinite choice principle. 
\end{itemize}
\end{frame}

\begin{frame}
\frametitle{The size of bases}
\begin{proposition}\label{P:LA2basis}
If $V$ is a finite vector space then every basis for $V$ has the same length.
\end{proposition}
\begin{proof}
\begin{itemize}
\item Let $s$ and $t$ be bases for $V$. 
\vspace{0.5cm}
\item Then, as $s$ is linearly independent and $t$ spans $V$, by proposition \ref{P:LA2length}, we must have $|s|\leq |t|$. 
\vspace{0.5cm}
\item But $t$ is also linearly independent, and $s$ also spans $V$, so by the same proposition we also have $|t|\leq |s|$. 
\vspace{0.5cm}
\item So $|s|=|t|$ as claimed.
\end{itemize}
\end{proof}
\end{frame}

\begin{frame}
\frametitle{Defining dimension}
\begin{definition}
If $V$ is a finite dimensional vector space, then we define the \emph{dimension} of $V$ to be the size of its bases. We use $\dim(V)$ to denote the dimension of $V$.
\end{definition}

\begin{example}
\begin{enumerate}
\item The vectors $(1,0,0)$, $(0,1,0)$ and $(0,0,1)$ provide a basis for $\bR^3$. So $\dim(\bR^3) = 3$.
\item The vectors $(2,0,1)$, $(2,3,0)$ and $(0,6,-1)$ also provide a basis for $\bR^3$.
\item The vectors $(1,2,3)$, $(-1,-1,0)$, $(1,1,1)$ and $(3,-2,0)$ must be linearly dependent in $\bR^3$.
\item The vectors, $1,x,x^2,x^4,\ldots$ provide a basis for $\bR[x]$, which is infinite dimensional.  
\end{enumerate}
\end{example}
\end{frame}

\begin{frame}
\frametitle{Spanning/linearly independent lists of the right size are bases}
\begin{theorem}
Let $V$ be a finite dimensional vector space. Then:
\begin{enumerate}
\item If $s$ is a spanning list for $V$ and $|s|=\dim(V)$ then $s$ is a basis for $V$.
\item If $t$ is a linearly independent list in $V$ and $|t|=\dim(V)$ then $t$ is a basis for $V$. 
\end{enumerate}
\end{theorem}
\begin{proof}
\begin{enumerate}
\item If $s$ spans $V$ then $s$ can be reduced to a basis, $s'$, for $V$. By proposition \ref{P:LA2basis} we must have $|s'| = \dim(V) = |s|$, so $s'$ must be equal to $s$.
\vspace{0.3cm}
\item If $t$ is linearly independent then $t$ can be extended to a basis, $t'$. for $V$. We have $|t'|=\dim(V)=|t|$, so $t$ is a basis for $V$. 
\end{enumerate}  
\end{proof}
\end{frame}

\begin{frame}
\frametitle{Subspaces of finite dimensional spaces}
\begin{proposition}\label{P:LA2subbase}
Every subspace of a finite dimensional vector space is finite dimensional.
\end{proposition}
\begin{proof}
\begin{itemize}
\item Let $V$ be a finite dimensional vector space and let $U$ be a subspace of $V$. 
\item If $U=\{0\}$ then the empty list spans $U$. 
\item If $U\neq \{0\}$ then we construct a basis for $U$ by recursion:
\begin{itemize}
\item Since $U\neq\{0\}$ we can choose $v_1 \in U\setminus\{0\}$. Define $s_1 = (v_1).$
\item Given linearly independent $s_i = (v_1,\ldots, v_i)$ in $U$, if $s_i$ does not span $U$ then there is $v_{i+1}\in U\setminus \spa(s_i)$. 
\item In this case define $s_{i+1} = (v_1,\ldots, v_i, v_{i+1})$.  
\end{itemize} 
\item $s_i$ is linearly independent for all $i$, and $|s_i|\leq \dim(U)$.
\item There is $k$ with $|s_k|= \dim(U)$. This $s_k$ is a basis for $U$.
\end{itemize}
\end{proof}
\end{frame}

\begin{frame}
\frametitle{The dimension of subspaces}
\begin{corollary}
If $V$ is a finite dimensional vector space and $U$ is a subspace of $V$, then $\dim(U)\leq \dim(V)$.
\end{corollary}
\begin{proof}
\begin{itemize}
\item Let $t= (v_1,\ldots,v_n)$ be a basis for $V$.
\vspace{0.5cm}
\item Let $s= (u_1,\ldots,u_k)$ be a basis for $U$. 
\vspace{0.5cm}
\item Then $s$ is linearly independent in $V$, and $t$ spans $V$, so $|s|\leq |t|$. 
\vspace{0.5cm}
\item Thus $\dim(U)\leq \dim(V)$ as claimed.
\end{itemize}
\end{proof}
\end{frame}

\begin{frame}
\frametitle{Subspaces and direct sums}
\begin{proposition}
Let $V$ be a finite dimensional vector space, and let $U$ be a subspace of $V$. Then there is a subspace $W$ of $V$ such that $V=U\oplus W$.
\end{proposition}
\begin{proof}
\begin{itemize}
\item Let $s=(u_1,\ldots,u_k)$ be a basis for $U$. 
\item Then $s$ is linearly independent in $V$, so $s$ can be extended to a basis $(u_1,\ldots,u_k,w_1,\ldots,w_m)$ for $V$. 
\item Define $W$ to be $\spa(w_1,\ldots,w_m)$.
\item To show $V = U\oplus W$ we check $V = U + W$, and $U\cap W =\{0\}$. 
\item $V = U + W$ as $(u_1,\ldots,u_k,w_1,\ldots,w_m)$ spans $V$. 
\item $U\cap W =\{0\}$ because $(u_1,\ldots,u_k)$ is basis for $U$, $(w_1,\ldots,w_m)$ is a basis for $W$, and $(u_1,\ldots,u_k,w_1,\ldots,w_m)$ is linearly independent. 
\end{itemize}  
\end{proof}
\end{frame}

\begin{frame}
\frametitle{The dimension of a sum}
\begin{proposition}
Let $V$ be a finite dimensional vector space, and let $U$ and $W$ be subspaces of $V$. Then $\dim(U+W)= \dim(U) +\dim(W) - \dim(U\cap W)$.
\end{proposition}
Proof.
\begin{itemize}
\item Let $(v_1,\ldots,v_n)$ be a basis for $U\cap W$.
\item We can extend $(v_1,\ldots,v_n)$ to a basis $(u_1,\ldots,u_k,v_1,\ldots,v_n)$ for $U$, and a basis $(v_1,\ldots,v_n, w_1,\ldots, w_m)$ for $W$. 
\item We claim that 
\[s=(u_1,\ldots,u_k,v_1,\ldots,v_n, w_1,\ldots, w_m)\] 
is a basis for $U+W$. 
\item $s$ clearly spans $U+W$, so we must check linear independence.
\end{itemize}    
\end{frame}

\begin{frame}
\frametitle{The dimension of a sum - proof continued}
\begin{itemize}
\item Suppose that 
\[a_1u_1+\ldots + a_k u_k + b_1v_1+\ldots+ b_nv_n+ c_1w_1+\ldots+ c_mw_m = 0.\]
\item Then 
\[c_1w_1+\ldots+ c_mw_m = -a_1u_1-\ldots - a_k u_k - b_1v_1-\ldots- b_nv_n,\]
\item So $c_1w_1+\ldots+ c_mw_m\in U\cap W$, and there are $b'_1,\ldots b'_n\in\bF$ with $c_1w_1+\ldots+ c_mw_m = b'_1v_1+\ldots+ b'_nv_n$. I.e.
\[c_1w_1+\ldots+ c_mw_m - b'_1v_1-\ldots- b'_nv_n = 0.\]
\item But $(v_1,\ldots,v_n, w_1,\ldots, w_m)$ is linearly independent, so $c_i = 0$ for all $i$. 
\item So $a_1u_1+\ldots + a_k u_k + b_1v_1+\ldots+ b_nv_n = 0$.
\item As $(u_1,\ldots,u_k,v_1,\ldots,v_n)$ is linearly independent $a_i=b_j=0$ for all $i,j$. 
\item So $s$ is linearly independent as required. 
\end{itemize}     
\end{frame}

\end{document}