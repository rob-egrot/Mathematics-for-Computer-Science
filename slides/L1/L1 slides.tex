\documentclass[handout]{beamer} 
\title{ITCS 531: Logic 1 - Semantics for propositional formulas}
\date{}
\author{Rob Egrot}

\usepackage{amsmath, bbold, bussproofs,graphicx}
\usepackage{mathrsfs}
\usepackage{amsthm}
\usepackage{amssymb}
\usepackage[all]{xy}
\usepackage{multirow}
\usepackage{tikz-cd}



\addtobeamertemplate{navigation symbols}{}{%
    \usebeamerfont{footline}%
    \usebeamercolor[fg]{footline}%
    \hspace{1em}%
    \insertframenumber/\inserttotalframenumber
}

\setbeamertemplate{theorems}[numbered]

\newcommand{\bN}{\mathbb{N}}
\newcommand{\bZ}{\mathbb{Z}}
\newcommand{\bQ}{\mathbb{Q}}
\newcommand{\bR}{\mathbb{R}}
\newcommand{\bP}{\mathbb{P}}
\newcommand{\HCF}{\mathbf{HCF}}
\newcommand{\lequiv}{\models\text{\reflectbox{$\models$}}}
\newcommand{\ra}{\rightarrow}



\newtheorem{proposition}[theorem]{Proposition}{\bfseries}{\itshape}


\begin{document}

\begin{frame}
\titlepage
\end{frame}

\begin{frame}
\frametitle{The logic of mathematical proofs}
\begin{itemize}
\item Formal mathematical arguments: 
\begin{itemize}
\item Start with assumptions (axioms). 
\item Sequence of logical deductions.
\item Desired conclusion. 
\end{itemize}
\vspace{0.2cm}
\item This axiom-theorem-proof style dates back to the Ancient Greeks, e.g. Euclid (around 300 BCE). 
\vspace{0.2cm}
\item This neat picture of mathematics does not really correspond to how mathematicians actually work as.
\vspace{0.2cm}
\item Mathematicians use a lot of informal intuition. 
\vspace{0.2cm}
\item The modern style of being very explicit about assumptions and definitions started in the late 19th century.
\end{itemize}
\end{frame}

\begin{frame}
\frametitle{The evolution of rigour}
\begin{itemize}
\item As mathematics became more advanced, mathematicians started proving contradictory things (e.g. in calculus). 
\vspace{0.2cm}
\item To resolve this, mathematicians became very precise. 
\vspace{0.2cm}
\item By doing this they were able to see that the contradictions often came from people starting from different assumptions.
\vspace{0.2cm}
\item Imre Lakatos' book \emph{proofs and refutations} explores this.
\vspace{0.2cm}
\item The formal style is not how mathematicians \emph{think}, but it is important to communication. 
\vspace{0.2cm}
\item It also helps prevent logic errors in mathematical reasoning. 
\vspace{0.2cm}
\item In practice: Think informally, write formally.
\begin{itemize}
\item Often this is taken too far!
\end{itemize}
\end{itemize}
\end{frame}

\begin{frame}
\frametitle{The role of formal logic}
\begin{itemize}
\item It is debatable whether the formal style captures the true essence of mathematics.
\vspace{0.2cm}
\item But, mathematics should be, in principle, capable of being expressed as a formal procession of axioms and deductions. 
\vspace{0.2cm}
\item I.e., mathematics can be treated as a formal system.
\vspace{0.2cm}
\item So we can use mathematical reasoning on mathematics! 
\vspace{0.2cm}
\item Mathematics about mathematics (metamathematics). 
\begin{itemize}
\item Spoiler: This leads to the development of computers, which we will study next semester. 
\end{itemize}
\vspace{0.2cm}
\item Computers also behave very much like formal systems...
\end{itemize}
\end{frame}

\begin{frame}
\frametitle{Applying logic}
\begin{itemize}
\item But, before we can understand the role of formal logic in:
\vspace{0.2cm}
\begin{itemize}
\item The theory of computation.
\vspace{0.2cm}
\item Analysis of the difficulty of computation problems.
\vspace{0.2cm}
\item Understanding the behaviour of software.
\vspace{0.2cm}
\item Etc.  
\end{itemize}
\vspace{0.2cm}
\item We need to understand the basics. 
\vspace{0.2cm}
\item That is what this course is about. 
\end{itemize}
\end{frame}

\begin{frame}
\frametitle{What is logic anyway?}
\begin{itemize}
\item To properly describe mathematical and computational ideas symbolically we will need a complex language.
\vspace{0.2cm}
\item We will get to this later in the course. 
\vspace{0.2cm}
\item First, we can think about the abstract structure of logical arguments with a relatively simple formal system.
\vspace{0.2cm}
\item The Ancient Greeks thought a lot about this. 
\vspace{0.2cm}
\item For example, Aristotle gave the following example of a logical deduction: 
\begin{enumerate}
\item \emph{All humans are mortal.} 
\item\emph{All Greeks are human.}
\item\emph{Therefore, all Greeks are mortal.}
\end{enumerate}
\end{itemize}
\end{frame}

\begin{frame}
\frametitle{The syllogism}
\begin{itemize}
\item Aristotle's argument again:
\begin{enumerate}
\item \emph{All humans are mortal.} 
\item\emph{All Greeks are human.}
\item\emph{Therefore, all Greeks are mortal.}
\end{enumerate}
\vspace{0.2cm}
\item This is an example of something call a \emph{syllogism}. 
\vspace{0.2cm}
\item The conclusion here is true in reality, but also, it \emph{has} to be true if the assumptions are true:
\begin{enumerate}
\item \emph{All $X$ have property $Y$.} 
\item\emph{$Z$ is $X$.}
\item\emph{Therefore, $Z$ has property $Y$.}
\end{enumerate}
\vspace{0.2cm}
\item Medieval Christian scholars studied syllogisms in great detail.
\vspace{0.2cm}
\item But for us syllogisms are not enough.
\end{itemize}
\end{frame}

\begin{frame}
\frametitle{Propositional logic}
\begin{itemize}
\item For our formal system of propositional logic we need three things:
\vspace{0.2cm}
\begin{itemize}
\item Basic propositions (AKA \emph{propositional variables}),  $\{p_0,p_1,p_2,\ldots\}$. Abstract true/false statements.
\vspace{0.2cm}
\item Logical connectives $\{\wedge,\vee,\neg,\rightarrow,\leftrightarrow\}$.
\vspace{0.2cm}
\begin{itemize}
\item[$\wedge$:] $p\wedge q$ is supposed to mean ``$p$ and $q$". 
\vspace{0.2cm}
\item[$\vee$:] $p\vee q$ is supposed to mean ``$p$ or $q$".
\vspace{0.2cm}
\item[$\neg$:] $\neg p$ is supposed to mean ``not $p$".
\vspace{0.2cm}
\item[$\rightarrow$:] $p\rightarrow q$ is supposed to mean ``$p$ implies $q$" specifically, \emph{material implication}.
\vspace{0.2cm}
\item[$\leftrightarrow$:] $p\leftrightarrow q$ for mutual implication.
\end{itemize}
\vspace{0.2cm}
\item Brackets $($ and $)$. We use these to delimit formulas.
\end{itemize}
\end{itemize}
\end{frame}

\begin{frame}
\frametitle{Giving meaning to propositional statements}
\begin{itemize}
\item If we assign meaning to some of the basic propositions we can combine them into new statements using the logical connectives and brackets.
\vspace{0.2cm}
\begin{example}
\begin{itemize}
\item Let $a,b,c\in \bN$, and suppose  $p$ means ``$a|b$", $q$ means ``$a|(b+c)$", and $r$ means ``$a|c$". 
\vspace{0.2cm}
\item Then $(p\wedge q)\rightarrow r$ means ``If $a$ divides $b$, and $a$ divides $(b+c)$, then $a$ divides $c$". 
\vspace{0.2cm}
\item This statement is true, which we proved in the number theory class.
\end{itemize}
\end{example}
\end{itemize}
\end{frame}

\begin{frame}
\frametitle{Another example}
\begin{example}
\begin{itemize}
\item Again let $a,b,c\in \bN$.
\vspace{0.2cm} 
\item Suppose $p$ means ``$a|b$", $q$ means ``$a|c$", and $r$ means ``$a|bc$". 
\vspace{0.2cm}
\item Then $(p\wedge q)\leftrightarrow r$ means ``$a$ divides $b$, and $a$ divides $b$, if and only if $a$ divides $bc$". 
\vspace{0.2cm}
\item This is not true (why?). 
\vspace{0.2cm}
\item The `only if' part is true, but the `if' part is not (though it looks similar to a true statement).
\end{itemize} 
\end{example}
\end{frame}

\begin{frame}
\frametitle{What is a `formula'?}
\begin{itemize}
\item Not every string we can make using basic propositions and logical connectives makes sense.
\begin{example}
$(p\rightarrow \wedge q)\vee \neg r$ doesn't make sense, whatever meaning we give to $p$, $q$ and $r$. It's not true or false, it just doesn't mean anything.
\end{example}
\vspace{0.2cm}
\item A \textbf{well-formed formula} (WFF)  is a string in propositional logic that is capable of making sense.
\vspace{0.2cm}
\item A recursive definition:
\begin{itemize}
\item Individual basic proposition symbols are well-formed formulas.
\item If $\phi$ is well-formed then $\neg \phi$ is well-formed.
\item If $\phi$ and $\psi$ are well-formed then $(\phi * \psi)$ is well-formed for all $*\in\{\wedge, \vee, \rightarrow,\leftrightarrow\}$.
\item Everything else is not well-formed.
\item (We sometimes cheat with the brackets).
\end{itemize}
\end{itemize}
\end{frame}

\begin{frame}
\frametitle{Subformulas}
\begin{itemize}
\item In propositional logic we often refer to well-formed formulas as just \emph{formulas}, and sometimes as \emph{sentences}.
\vspace{0.3cm}
\item If $\phi$ is a formula, then a \textbf{subformula} of $\phi$ is a substring of $\phi$ that is also a sentence (i.e. can be obtained by our recursive construction).
\vspace{0.3cm}
\item E.g.  $(p\wedge q)$ is a subformula of $(p\wedge q)\rightarrow r$, and so is e.g. $r$. 
\vspace{0.3cm}
\item We consider $\phi$ to be a subformula of itself. 
\vspace{0.3cm}
\item We define the \textbf{length} of a sentence $\phi$ to be the number of logical connectives that occur in $\phi$. 
\vspace{0.3cm}
\item E.g. if $\phi = \neg((p\vee q)\wedge q)$ then the length of $\phi$ is 3.
\end{itemize}
\end{frame}

\begin{frame}
\frametitle{When is a sentence true?}
\begin{itemize}
\item Every basic proposition must be either true or false, and cannot be both.
\vspace{0.3cm}
\item The same applies to sentences.
\vspace{0.3cm}
\item Whether a sentence is true or false is depends only on the true/false values of the basic propositions it is built from.
\vspace{0.3cm}
\item This truth value can be calculated recursively from the truth values of the basic propositions. 
\vspace{0.3cm}
\item We use \textbf{truth tables} to represent the recursion rules.
\end{itemize}
\end{frame}

\begin{frame}
\frametitle{Truth tables}
\begin{itemize}
\item[$\neg$:]
\begin{tabular}{ c c  }
 $\phi$ & $\neg \phi$  \\ \hline 
 T & F  \\  
 F & T     
\end{tabular}
      $\phantom{\iff}\wedge$:
\begin{tabular}{ c c c  }
 $\phi$ & $\psi$ & $\phi\wedge \psi$ \\ \hline 
 T & T & T \\  
 T & F & F \\
 F & T & F \\
 F & F & F    
\end{tabular} 
      $\phantom{iff}\vee$: 
\begin{tabular}{ c c c  }
 $\phi$ & $\psi$ & $\phi\vee \psi$ \\ \hline 
 T & T & T \\  
 T & F & T \\
 F & T & T \\
 F & F & F    
\end{tabular} 
\vspace{0.5cm}
\item[$\rightarrow$:] 
\begin{tabular}{ c c c  }
 $\phi$ & $\psi$ & $\phi\rightarrow \psi$ \\ \hline 
 T & T & T \\  
 T & F & F \\
 F & T & T \\
 F & F & T    
\end{tabular} 
$\phantom{iff}\leftrightarrow$:
\begin{tabular}{ c c c  }
 $\phi$ & $\psi$ & $\phi\leftrightarrow \psi$ \\ \hline 
 T & T & T \\  
 T & F & F \\
 F & T & F \\
 F & F & T    
\end{tabular} 
\end{itemize}
\end{frame}

\begin{frame}
\frametitle{Using truth tables}
\begin{example}    
\[\begin{tabular}{ c c c c c }
 $p$ & $q$ & $r$ & $p\wedge q$ & $(p\wedge q)\rightarrow r$ \\ \hline 
 T & T & T & T & T \\  
 T & T & F & T & F \\
 T & F & T & F & T \\
 T & F & F & F & T \\
 F & T & T & F & T \\
 F & T & F & F & T \\
 F & F & T & F & T \\
 F & F & F & F & T 
\end{tabular} \]
\end{example}
\end{frame}

\begin{frame}
\frametitle{Implication}
\begin{itemize}
\item Truth table for $\ra$ gives the \emph{material conditional}.
\begin{itemize}
\item $\phi\ra\psi$ is true whenever $\phi$ is false or $\psi$ is true. 
\end{itemize}
\item  Contrast with \emph{subjunctive implication}: 
\begin{itemize}
\item E.g: "if I dropped it, then it would break". 
\item Should be true for a chicken egg, false for a tennis ball.
\item According to material implication, this is true for tennis balls and eggs if I don't drop them! 
\end{itemize}
\item Consider also \emph{indicative implication}: 
\begin{itemize}
\item E.g. suppose a student likes yoghurt, studies hard and does well in her exams. 
\item ``studied hard so did well in her exams" is probably true.
\item  ``likes yoghurt so did well in her exams" is probably false. 
\item I.e. the first part should be relevant to the second part.
\item Material conditional says both are true.
\end{itemize}
\item So, material implication is not appropriate for everything. 
\item But it is good for formal systems. 
\begin{itemize}
\item I.e. if $\phi$ is false then $\phi\ra \psi$ makes no claim, so is true.
\end{itemize}
\end{itemize}
\end{frame}

\begin{frame}
\frametitle{Satisfaction}
\begin{itemize}
\item Setting every propositional variable to be either true or false is making a \textbf{truth assignment} (or just \textbf{assignment}). 
\vspace{0.3cm}
\item If a sentence is true under some assignment we say it is \textbf{satisfied} by that assignment. 
\vspace{0.3cm}
\item A sentence is \textbf{satisfiable} if there is some assignment that satisfies it. 
\begin{itemize}
\item I.e. if there is a way we can interpret each basic proposition as true or false so that the whole thing becomes true. 
\end{itemize}
\vspace{0.3cm}
\item If $\Gamma$ is a set of sentences, then we say $\Gamma$ is satisfiable if there is an assignment that satisfies every sentence in $\Gamma$. 
\end{itemize}
\end{frame}

\begin{frame}
\frametitle{Tautologies and contradictions}
\begin{itemize}
\item A sentence that is satisfied by every assignment is called a \textbf{tautology}. 
\begin{itemize}
\item I.e. a tautology is something that is always true.
\item E.g. $p\vee \neg p$ (a proposition must be either true or false). 
\item Warning: in some logic systems this is not something we can assume! (see the next class).  
\item We sometimes use the symbol $\top$ to denote a tautology. 
\end{itemize}
\vspace{0.3cm}
\item A sentence that is not satisfiable is called a \textbf{contradiction}. 
\begin{itemize}
\item I.e. a contradiction can never be true.
\item E.g. $p\wedge \neg p$ (a proposition cannot be both true and false at the same time). 
\item We sometimes use the symbol $\bot$ to denote a contradiction.
\end{itemize}
\vspace{0.3cm}
\item  If $\phi$ is a tautology then $\neg \phi$ is a contradiction, and vice versa.
\end{itemize}
\end{frame}

\begin{frame}
\frametitle{Logical implication}
\begin{itemize}
\item Given $\phi$ and $\psi$, we say that $\phi$ \textbf{logically implies} $\psi$ if whenever an assignment satisfies $\phi$, it also satisfies $\psi$. 
\vspace{0.3cm}
\item We also say $\psi$ is a \textbf{logical consequence} of $\phi$.
\vspace{0.3cm}
\item We write $\phi\models \psi$.
\vspace{0.3cm}
\item This is another way of saying that $\phi\ra\psi$ is true. 
\vspace{0.3cm}
\item We say that $\phi$ and $\psi$ are \textbf{logically equivalent} if each is a logical consequence of the other. 
\vspace{0.3cm}
\item In this case we write $\phi \lequiv \psi$. 
\end{itemize}
\end{frame}

\begin{frame}
\frametitle{Theories}
\begin{itemize}
\item We can also do this with sets of sentences. 
\vspace{0.3cm}
\item If $\Gamma$ is a set of sentences and $\psi$ is a sentence, then $\psi$ is a logical consequence of $\Gamma$ if, whenever an assignment satisfies $\phi$ for all $\phi\in\Gamma$, it also satisfies $\psi$. 
\vspace{0.3cm}
\item We write $\Gamma\models \psi$. 
\vspace{0.3cm}
\item We sometimes call a set of sentences a \textbf{theory}, and then we might say that $\psi$ is a consequence of the theory $\Gamma$. 
\vspace{0.3cm}
\item We want to be able to say things like ``the fact that the set of primes is infinite is a consequence of the theory of numbers". 
\vspace{0.3cm}
\item A theory can be empty (i.e. have no members). 
\vspace{0.3cm}
\item We write $\models \phi$ if $\phi$ follows from the empty theory, i.e. if $\phi$ is a tautology. 
\end{itemize}
\end{frame}

\begin{frame}
\frametitle{A simple theory example}
\begin{itemize}
\item Let $\Gamma$ be the theory $\{p\wedge \neg q, q\vee r\}$. 
\item Then $r\wedge p$ is a logical consequence of $\Gamma$ (i.e. $\Gamma\models p\wedge r$).
\[\begin{tabular}{ c c c c c c c}
 $p$ & $q$ & $r$ & $p\wedge \neg q$ & $q\vee r$ & $r\wedge p$ \\ \hline 
 T & T & T & F & T & T \\  
 T & T & F & F & T & F \\
 T & F & T & T & T & T \\
 T & F & F & T & F & F \\
 F & T & T & F & T & F \\
 F & T & F & F & T & F \\
 F & F & T & F & T & F \\
 F & F & F & F & F & F
\end{tabular}\]
\item There is only one assignment that makes both $p\wedge \neg q$ and $q\vee r$ true, and that is $p=r=T$, $q=F$. 
\item $r\wedge p$ is also true with this assignment. 
\item So, every assignment that makes everything in $\Gamma$ true must also make $r\wedge p$ true, which means $\Gamma\models r\wedge p$.
\end{itemize}
\end{frame}

\begin{frame}
\frametitle{Avoiding truth tables}
\begin{itemize}
\item We could also work this out without writing out the whole truth table. 
\begin{itemize}
\item E.g. we might notice just by looking at the formulas that $p\wedge \neg q$ being true means $p$ is true and $q$ is false.
\item Then $q\vee r$ can only be true if $r$ is true. 
\item Which means $r\wedge p$ must be true too.
\end{itemize}
\vspace{0.2cm}
\item For complicated formulas, writing out a truth table is quite a lot of effort.
\vspace{0.2cm}
\item So it's usually a good idea to look at the formulas first and see if you can find a quick argument for why one thing logically implies another. 
\vspace{0.2cm}
\item But, if you get stuck, the option of working out the truth table is always there.
\end{itemize}
\end{frame}

\begin{frame}
\frametitle{Eliminating $\ra$}
\begin{itemize}
\item The set $\{\wedge,\vee,\neg,\rightarrow,\leftrightarrow\}$ is bigger than we need. 
\item We can use truth tables to check that some of the connectives can be reproduced using combinations of different ones.
\end{itemize}
\begin{lemma}\label{L:imp}
If $\phi$ and $\psi$ are sentences, then $\neg\phi \vee \psi \lequiv \phi\rightarrow \psi$.
\end{lemma}
\begin{proof}\mbox{}
\[\begin{tabular}{ c c c c }
 $\phi$ & $\psi$ & $\phi\rightarrow \psi$ & $\neg\phi\vee\psi$ \\ \hline 
 T & T & T & T \\  
 T & F & F & F \\
 F & T & T & T \\
 F & F & T & T   
\end{tabular}\]
\newline
We can see that the last two columns are the same. 
\end{proof}
\end{frame}

\begin{frame}
\frametitle{Why is this important?}
\begin{itemize}
\item What this means is that whenever $\phi\ra \psi$ appears in a formula, we could replace it with $\neg\phi\vee\psi$ without changing the truth value of the formula. 
\vspace{0.5cm}
\item In other words, we don't really need the connective $\ra$, because for every formula involving $\ra$ there's an equivalent one where it does not appear. 
\vspace{0.5cm}
\item This might be intuitively obvious, but we will provide proof now, as the proof method will be very important.
\end{itemize}
\end{frame}

\begin{frame}
\frametitle{The formal version}
\begin{corollary}\label{C:imp}
If $\phi$ is a sentence, then there is a sentence $\phi'$ where the symbol $\rightarrow$ does not occur, and with $\phi\lequiv \phi'$.
\end{corollary}
\begin{proof}
\begin{itemize}
\item We induct on the length of $\phi$. 
\item Base case: $n=0$, so $\phi$ is a basic proposition. Define $\phi = \phi'$.
\item Inductive step: suppose true for length $n$, and let $\phi$ have length $n+1$. There are three cases:
\begin{enumerate}
\item $\phi = \neg \psi$ for some $\psi$.
\begin{itemize}
\item  $\psi$ has length $n$, so there is $\psi'$ that does not contain `$\rightarrow$' and with $\psi\lequiv \psi'$. 
\item Define $\phi'=\neg\psi'$ to complete the proof, as, since $\psi\lequiv \psi'$ we must have $\neg\psi\lequiv \neg\psi'$, and $\phi=\neg\psi$.
\end{itemize}
\item $\phi = \psi_1*\psi_2$ for some $*\in\{\wedge,\vee,\leftrightarrow\}$. 
\begin{itemize}
\item Define $\phi'=\psi_1'*\psi_2'$.
\end{itemize}
\item $\phi = \psi_1\rightarrow \psi_2$. 
\begin{itemize}
\item By Lemma \ref{L:imp} we can define $\phi'=\neg\psi_1'\vee \psi_2'$.
\end{itemize}
\end{enumerate} 
\end{itemize}
\end{proof}
\end{frame}

\begin{frame}
\frametitle{Functional completeness}
\begin{itemize}
\item A set of connectives defined by truth tables, $S$, is \textbf{functionally complete} if every formula that can be constructed from $\{\wedge,\vee,\neg,\rightarrow,\leftrightarrow\}$ is logically equivalent to one constructed from $S$. 
\end{itemize}

\begin{proposition}
$\{\wedge,\vee,\neg,\leftrightarrow\}$ is functionally complete.
\end{proposition}
\begin{proof}
This is what we showed in corollary \ref{C:imp}.
\end{proof}
\begin{itemize}
\item The proof of corollary \ref{C:imp} generalizes to other connectives. 
\item If any sentence containing a particular connective is equivalent to another not containing it, then we still have a functionally complete set of connectives if we eliminate it. 
\item We will see in the exercises that $\{\wedge, \neg\}$ is functionally complete, and the same is true for $\{\vee,\neg\}$. 
\end{itemize}
\end{frame}


\end{document}