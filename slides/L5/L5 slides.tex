\documentclass[handout]{beamer} 
\title{ITCS 531: Logic 5 - Basic model theory}
\date{}
\author{Rob Egrot}

\usepackage{amsmath, bbold, bussproofs,graphicx}
\usepackage{mathrsfs}
\usepackage{amsthm}
\usepackage{amssymb}
\usepackage[all]{xy}
\usepackage{multirow}
\usepackage{tikz-cd}



\addtobeamertemplate{navigation symbols}{}{%
    \usebeamerfont{footline}%
    \usebeamercolor[fg]{footline}%
    \hspace{1em}%
    \insertframenumber/\inserttotalframenumber
}

\setbeamertemplate{theorems}[numbered]

\newcommand{\trm}{\mathbf{term}}
\newcommand{\sL}{\mathscr{L}}
\newcommand{\cR}{\mathcal{R}}
\newcommand{\cF}{\mathcal{F}}
\newcommand{\cC}{\mathcal{C}}
\newcommand{\bN}{\mathbb{N}}
\newcommand{\bZ}{\mathbb{Z}}
\newcommand{\bQ}{\mathbb{Q}}
\newcommand{\bR}{\mathbb{R}}
\newcommand{\bP}{\mathbb{P}}
\newcommand{\HCF}{\mathbf{HCF}}
\newcommand{\lequiv}{\models\text{\reflectbox{$\models$}}}

\begin{document}

\begin{frame}
\titlepage
\end{frame}

\begin{frame}
\frametitle{Structures and languages}
\begin{itemize}
\item In the last class we introduced the idea of a first-order language $\sL$.
\vspace{0.2cm}
\item A mathematical structure is an $\sL$-structure if it has relations, functions and constants corresponding to the non-logical symbols of $\sL$.
\vspace{0.2cm}
\item If $\phi$ is an $\sL$-sentence and $A$ is an $\sL$-structure, if $\phi$ is true in $A$ we say $A$ is a \emph{model} for $\phi$ (we write $A\models \phi$).
\vspace{0.2cm}
\item Given a set of $\sL$-sentences, we can try to understand them by studying its models.
\vspace{0.2cm}
\item Conversely, given a structure $A$ we can try to think of it as an $\sL$-structure for some $\sL$. 
\vspace{0.2cm}
\item We can study $A$ by looking at the $\sL$-sentences that it models.
\end{itemize}
\end{frame}

\begin{frame}
\frametitle{Model theory}
\begin{itemize}
\item This two-way study of logic through structures, and structures through logic, is the starting point of \emph{model theory}.
\vspace{0.4cm}
\item In this class we will look more closely at the concept of models and semantic implication.
\vspace{0.4cm}
\item We will also define a concept of deduction based on that for propositional logic.
\vspace{0.4cm}
\item We will state soundness and completeness theorems for first-order logic.
\vspace{0.4cm}
\item We will see some limitations of first-order logic for studying infinite structures. 
\end{itemize}
\end{frame}

\begin{frame}
\frametitle{Logical implication}
\begin{itemize}
\item If $\Gamma$ is a set of $\sL$-formulas, and if $\phi$ is an $\sL$-formula, we write $\Gamma\models \phi$ if, whenever $v$ is an assignment of the variables of $\sL$ into an $\sL$-structure $A$, we have \[A, v \models \Gamma\implies A,v \models \phi.\]
\item We say that $\phi$ is a \emph{logical consequence} of $\Gamma$.
\vspace{0.2cm}
\item Remember \emph{sentences} are formulas that have no free variables.
\vspace{0.2cm}
\item For sentences we can just write, e.g. $A\models \phi.$
\vspace{0.2cm}
\item When $A\models \phi$, we say $A$ is a \emph{model} for $\phi$. 
\vspace{0.2cm}
\item If $\Delta$ is a set of $\sL$-sentences we can write e.g. $A\models \Delta$ when $A\models \phi$ for all $\phi\in\Delta$, and say $A$ is a model for $\Delta$.
\end{itemize}
\end{frame}

\begin{frame}
\frametitle{Validity and satisfiability}
\begin{definition}
If $\phi$ is an $\sL$-formula then we say $\phi$ is:
\begin{itemize}
\item \emph{Valid} if $A,v\models \phi$ whenever $A$ is an $\sL$-structure and $v$ is an assignment.
\item \emph{Satisfiable} if there is an $\sL$-structure $A$ and an assignment $v$ with $A,v\models \phi$.
\item A \emph{contradiction} if it is not satisfiable, i.e. if there is no $A,v$ with $A,v\models\phi$.
\end{itemize}

Similarly, if $\Gamma$ is a set of $\sL$-formulas then $\Gamma$ is:
\begin{itemize}
\item \emph{Valid} if $A,v\models \Gamma$ whenever $A$ is an $\sL$-structure and $v$ is an assignment.
\item \emph{Satisfiable} if there is an $\sL$-structure $A$ and an assignment $v$ with $A,v\models \Gamma$.
\item \emph{Contradictory} if it is not satisfiable, i.e. if there is no $A,v$ with $A,v\models\Gamma$. If $\Gamma$ is not satisfiable we write $\Gamma\models \bot$.
\end{itemize}
\end{definition}
\end{frame}

\begin{frame}
\frametitle{Example - arithmetic}
\begin{example}
Let $\sL=\{0,1,\times,+\}$ be the language of arithmetic.
\begin{enumerate}
\item Let $\phi = \forall x\big((x\approx 0)\vee \neg(x\approx 0)\big)$. Then $\phi$ is valid. More generally, if $\sL$ is a language, and if $\phi_1,\ldots,\phi_n$ are $\sL$-sentences, then any propositional tautology constructed by treating the $\phi_i$ as basic propositions will be valid.
\vspace{0.2cm}
\item Let $\psi= \forall x(\neg(x\approx 0)\rightarrow \exists y(x\times y\approx 1))$. This is true if we take $\bR$ as our structure, but not if we take $\bZ$. So $\psi$ is satisfiable but not valid.
\vspace{0.2cm}
\item If $\phi_1,\ldots,\phi_n$ are $\sL$-sentences, then any propositional contradiction using the $\phi_i$ as basic propositions will be a contradiction. 
\end{enumerate}
\end{example}
\end{frame}

\begin{frame}
\frametitle{Checking satisfiability}
\begin{definition}[Theory]\label{D:theory}
If $\sL$ is a language, then an $\sL$-\emph{theory} is a satisfiable set of $\sL$-sentences. 
\end{definition}
\vspace{0.5cm}
\begin{itemize}
\item Checking logical consequence, validity etc. is much more complicated for first-order logic than for propositional logic.
\vspace{0.2cm}
\item In propositional logic, you just construct a truth table.
\vspace{0.2cm}
\item In first-order logic, you might have to check every possible $\sL$-structure.
\vspace{0.2cm}
\item Generally this is not feasible.
\vspace{0.2cm}
\item There is no algorithm for saying whether an arbitrary $\sL$-sentence has a model (proved next semester).
\end{itemize}
\end{frame}

\begin{frame}
\frametitle{Intended models}
\begin{itemize}
\item When we write down axioms in first-order logic, there is often some particular system whose behaviour we are trying to formalize.
\vspace{0.3cm}
\item E.g. we might write down axioms for defining real numbers.
\vspace{0.3cm}
\item The intended model here is $\bR$, and we can choose axioms so that $\bR$ is indeed a model.
\vspace{0.3cm}
\item But can we choose first-order axioms so that $\bR$ is the only model? 
\vspace{0.3cm}
\item No. In fact, it is impossible to use first-order logic to define a specific infinite structure.
\vspace{0.3cm}
\item This is due to the theorem on the next slide.
\end{itemize}
\end{frame}

\begin{frame}
\frametitle{The L\"owenheim-Skolem theorem}
\begin{theorem}\label{T:LS}
Let $\Gamma$ be a countable $\sL$-theory. Then, if $\Gamma$ has an infinite model, it has models of every infinite cardinality.
\end{theorem}
\vspace{1cm}
\begin{itemize}
\item Theorem \ref{T:LS} gives us an infinite supply of extra models for any theory that has at least one infinite model. \vspace{0.5cm}
\item It also tells us that first-order logic can't `tell the difference' between different infinite cardinalities. \vspace{0.5cm}
\item Unintended models need not have different cardinalities though, as the example on the next slide illustrates.
\end{itemize}
\end{frame}

\begin{frame}
\frametitle{Unnatural natural numbers}
\begin{example}\label{E:nums}
Let $\sL=\{0,s\}$, where $s$ is a unary function. Let $\Gamma$ consist of the following sentences.
\begin{itemize}
\item[$\phi_1$:] $\forall x (\neg (x\approx 0)\rightarrow \exists y (x= s(y))$.
\item[$\phi_2$:] $\forall x (\neg (x\approx s(x)))$. 
\item[$\phi_3$:] $\forall x\forall y ((s(x)\approx s(y))\rightarrow (x\approx y))$. 
\end{itemize}
\vspace{0.3cm}
\begin{itemize}
\item One model of $\Gamma$ is $\bN$, where $s$ is interpreted as the `successor' function. 
\item Is $\bN$ the only model? 
\item No, for example, the disjoint union of $\bN$ and $\bZ$ is also a model if we interpret $0$ as the zero of $\bN$, and $s$ as the successor function in both $\bN$ and $\bZ$. 
\end{itemize}
\end{example}
\end{frame}

\begin{frame}
\frametitle{Technical aside - substitution}
\begin{itemize}
\item Let $\phi$ be an $\sL$-formula with free variables $x_1,\ldots x_n$.
\vspace{0.2cm}
\item We can express this fact by writing $\phi[x_1,\ldots,x_n]$.
\vspace{0.2cm}
\item Let $t$ be an $\sL$-term, and let $i\in \{1,\ldots,n\}$.
\vspace{0.2cm}
\item We can create a new formula from $\phi$ by replacing every occurrence of the variable $x_i$ with the term $t$.
\vspace{0.2cm}
\item We use the notation $\phi[x_1,\ldots,x_{i-1},t/x_i,x_{i+1},\ldots,x_n]$ to denote this new formula.
\vspace{0.2cm}
\item E.g. let $\phi[x,y] = s(x)\approx y$, let $t = s(s(z))$.
\vspace{0.2cm}
\item Substituting $t$ for $x$ gives in $\phi$ gives $\phi[t/x,y]=s(s(s(z)))\approx y$.
\vspace{0.2cm}
\item Note that $\phi[x,y]$ and $\phi[t/x,y]$ have different free variables.
\end{itemize}
\end{frame}

\begin{frame}
\frametitle{Deduction in first-order logic}
\begin{itemize}
\item We can extend the natural deduction system for propositional logic to first-order logic. 
\vspace{1cm}
\item We have all the same deduction rules as before (but with first-order formulas in place of propositional sentences). 
\vspace{1cm}
\item We also have extra ones for $\approx$ and the quantifiers $\forall$ and $\exists$.
\end{itemize}
\end{frame}

\begin{frame}
\frametitle{Inroduction rules}


\begin{prooftree}
\AxiomC{}
\LeftLabel{ $\approx_I$:\quad}
\UnaryInfC{$t\approx t$}
\end{prooftree}
\vspace{1cm}
\begin{prooftree}
\AxiomC{$\phi[x'/x]$}
\LeftLabel{ $\forall_I$:\quad}
\UnaryInfC{$\forall x \phi$}
\end{prooftree}
\vspace{1cm}
\begin{prooftree}
\AxiomC{$\phi[t/x]$}
\LeftLabel{ $\exists_I$:\quad}
\UnaryInfC{$\exists x \phi$}
\end{prooftree}
\end{frame}

\begin{frame}
\frametitle{Elimination rules}
\begin{prooftree}
\AxiomC{$t_1\approx t_2$}
\AxiomC{$\phi[t_1/z]$}
\LeftLabel{ $\approx_E$:\quad}
\BinaryInfC{$\phi[t_2/z]$}
\end{prooftree}
\vspace{1cm}
\begin{prooftree}
\AxiomC{$\forall x \phi$ }
\LeftLabel{ $\forall_E$:\quad}
\UnaryInfC{$\phi[t/x]$}
\end{prooftree}
\vspace{1cm}
\begin{prooftree}
\AxiomC{$\exists x \phi$}
\AxiomC{$[\phi[x'/x]]$}
\doubleLine
\UnaryInfC{$\psi$}
\LeftLabel{ $\exists_E$:\quad}
\BinaryInfC{$\psi$}
\end{prooftree}

\end{frame}

\begin{frame}
\frametitle{First-order deduction example}
\begin{example} Let $\phi$ and $\psi$ be formulas where $x$ occurs free. Then we can deduce $\forall x \psi$ from $\forall x \neg \phi$ and $\forall x (\phi \vee \psi)$.
\end{example}
\begin{prooftree}
\AxiomC{$\forall x \neg\phi$}
\LeftLabel{ $(\forall_E)$}
\UnaryInfC{$\neg\phi[x'/x]$}
\AxiomC{$\forall x (\phi\vee \psi)$}
\LeftLabel{ $(\forall_E)$}
\UnaryInfC{$\phi[x'/x]\vee \psi[x'/x]$}
\doubleLine
\LeftLabel{(p.d.)}
\BinaryInfC{$\psi[x'/x]$}
\LeftLabel{ $(\forall_I)$}
\UnaryInfC{$\forall x\psi$}
\end{prooftree}
\vspace{1cm}
p.d. stands for \emph{propositional deduction}.
\end{frame}

\begin{frame}
\frametitle{Another first-order deduction example}
\begin{example}
Let $\phi$ and $\psi$ be formulas where $x$ occurs free. Then we can deduce $\exists x \psi$ from $\exists x \neg \phi$ and $\forall x (\phi \vee \psi)$.
\end{example}
\begin{prooftree}
\AxiomC{$\exists x \neg\phi$}
\AxiomC{$[\neg\phi[x'/x]]$}
\UnaryInfC{$\neg\phi[x'/x]$}
\LeftLabel{$(\exists_E)$}
\BinaryInfC{$\neg\phi[x'/x]$}
\AxiomC{$\forall x (\phi\vee \psi)$}
\RightLabel{$(\forall_E)$}
\UnaryInfC{$\phi[x'/x]\vee\psi[x'/x]$}
\doubleLine
\LeftLabel{(p.d.)}
\BinaryInfC{$\psi[x'/x]$}
\LeftLabel{$(\exists_I)$}
\UnaryInfC{$\exists x\psi$}
\end{prooftree}
\end{frame}

\begin{frame}
\frametitle{Consistency}
\begin{itemize}
\item As with propositional logic we write $\Gamma\vdash \phi$ if $\phi$ can be deduced from a set of formulas $\Gamma$. \vspace{0.5cm}
\item We say a set of $\sL$-sentences, $\Gamma$, is \textbf
{consistent} if we do not have $\Gamma\vdash \bot$. \vspace{0.5cm}
\item We sometimes describe a consistent set of $\sL$-sentences as an $\sL$-\emph{theory}. \vspace{0.5cm}
\item This is consistent with definition \ref{D:theory} because, as in propositional logic, there is a strong link between $\vdash$ and $\models$ (see next slide).
\end{itemize}
\end{frame}

\begin{frame}
\frametitle{Soundness and completeness}
\begin{theorem}[G\"odel]\label{T:G1}
Let $\Gamma$ be a set of $\sL$-formulas. Then $\Gamma$ is consistent if and only if it is satisfiable.
\end{theorem}\vspace{0.5cm}
\begin{theorem}[Extended soundness and completeness]\label{T:G2}
Let $\Gamma$ be a set of $\sL$-formulas and let $\phi$ be an $\sL$-formula. Then
\[\Gamma\vdash \phi\iff \Gamma\models \phi.\]
\end{theorem}\vspace{0.5cm}
These two results are equivalent (this is proved in the exercises this week).
\end{frame}

\begin{frame}
\frametitle{Proving theorem \ref{T:G2} - Soundness}
\begin{itemize}
\item Like the soundness theorem for propositional logic.
\item Must show each deduction rule is sound.
\item E.g. 
\begin{itemize}
\item[$\wedge_I$:] 
\begin{itemize} 
\item Have deduced $\phi$ and $\psi$ from $\Gamma$, and from these have deduced $\phi\wedge \psi$. 
\item Assuming deductions of $\phi$ and $\psi$ are  sound, any pair $(A,v)$ satisfying $\Gamma$ must satisfy $\phi$ and $\psi$. 
\item So also satisfies $\phi\wedge \psi$.
\end{itemize}
\end{itemize}
\begin{itemize}
\item The new rules are trickier, but as a sketch:
\item[$\forall_I$:] 
\begin{itemize}
\item Have deduced $\phi[x'/x]$ for arbitrary choice of $x'$. 
\item Assuming this deduction is sound, any pair $(A,v)$ satisfying $\Gamma$ will also satisfy $\phi[x'/x]$ 
\item Must show that $A,v\models \forall x \phi[x]$ too.
\item I.e. if $v'$ agrees with $v$ except possibly about $x$ then $A,v'\models \phi[x]$.
\item With a little fiddling we can do this (see notes).
\end{itemize}
\end{itemize}
\end{itemize}
\end{frame}

\begin{frame}
\frametitle{Proving theorem \ref{T:G2} - Completeness}
\begin{itemize}
\item Completeness is harder, but conceptually similar to the propositional version.
\vspace{0.5cm}
\item Again, proving completeness is equivalent to proving that every consistent set of formulas is satisfiable.
\vspace{0.5cm}
\item Rather than just building a true/false assignment that satisfies a consistent set of propositional sentences, we must find a pair $(A,v)$ satisfying a set of first-order formulas.
\vspace{0.5cm}
\item It is possible to do this, using the formulas themselves as the base for a model. 
\end{itemize}
\end{frame}


\end{document}