\documentclass[handout]{beamer} 
\title{ITCS 531: Counting - Cardinal numbers}
\date{}
\author{Rob Egrot}

\usepackage{amsmath, bbold, bussproofs,graphicx}
\usepackage{mathrsfs}
\usepackage{amsthm}
\usepackage{amssymb}
\usepackage[all]{xy}
\usepackage{multirow}
\usepackage{tikz-cd}


\newtheorem{proposition}[theorem]{Proposition}
\newcommand{\bN}{\mathbb{N}}
\newcommand{\bZ}{\mathbb{Z}}
\newcommand{\bQ}{\mathbb{Q}}
\newcommand{\bR}{\mathbb{R}}
\newcommand{\bP}{\mathbb{P}}

\addtobeamertemplate{navigation symbols}{}{%
    \usebeamerfont{footline}%
    \usebeamercolor[fg]{footline}%
    \hspace{1em}%
    \insertframenumber/\inserttotalframenumber
}
\setbeamertemplate{theorems}[numbered]
\begin{document}

\begin{frame}
\titlepage
\end{frame}

\begin{frame}
\frametitle{What is a set?}
\begin{itemize}
\item A \textbf{set} is a collection of objects.
\vspace{0.2cm}
\item E.g. the natural numbers, the real numbers, the integers etc. are all sets.
\vspace{0.2cm}
\item So is something like $\{1,2,5,11,6\}$.
\vspace{0.2cm}
\item Sets don't just contain numbers.
\vspace{0.2cm}
\item We can have e.g. the set of all students in this class, the set of all toasters made in Germany etc.
\vspace{0.2cm}
\item We can also define sets arbitrarily, e.g. the set that contains my left shoe, the set of all prime numbers, and the current president of France.
\vspace{0.2cm}
\item Notice that in the previous example a set contains another set as a member.
\end{itemize}
\end{frame}

\begin{frame}
\frametitle{When are two sets the same?}
\begin{itemize}
\item Sets are unordered collections that contain no duplicates.
\vspace{0.4cm}
\item Two sets are the same (equal) if they have exactly the same members.
\vspace{0.4cm}
\item So there may be multiple ways to define the same set.
\vspace{0.4cm}
\item E.g. The set of all natural numbers greater than 1 is the same as the set of all possible products of prime numbers.
\vspace{0.4cm}
\item There's a special set that contains nothing, the \textbf{empty set}, $\emptyset$.
\end{itemize}
\end{frame}

\begin{frame}
\frametitle{Defining sets from other sets}
\begin{itemize}
\item Given a set $X$ we can define the powerset $\wp(X) = \{S:S\subseteq X\}$, the set of all subsets of $X$. 
\vspace{0.5cm}
\item Given sets $X$ and $Y$ we can define things like the union $X\cup Y$ and intersection $X\cap Y$. 
\vspace{0.5cm}
\item If $I$ is an \emph{indexing set}, that is, a set we use just to label things, and for each $i\in I$ there is a set $X_i$, then we can take the infinite union $\bigcup_I X_i$ and the infinite intersection $\bigcap_I X_i$.
\vspace{0.5cm}
\item There are also other ways we can build sets from other sets. E.g. if $X$ is a set, and each element $x\in X$ is associated with some other object, $y_x$ say, then $\{y_x:x\in X\}$ is a set too.
\end{itemize}
\end{frame}

\begin{frame}
\frametitle{Set theory as a foundation for mathematics}
\begin{itemize}
\item Gottlob Frege and Bertrand Russell wanted to use the idea of a set to formalize mathematics. 
\vspace{0.2cm}
\item So the results of mathematics could be derived just by thinking hard enough about the logic of sets.
\vspace{0.2cm}
\item They wanted to do this for philosophical reasons related to the work of the German idealist philosopher Immanuel Kant.
\vspace{0.2cm}
\item They believed it is a fact of pure logic that every concept defines a set. I.e., if $P$ is a property of objects, then I can define the set of all the things that $P$ applies to (in symbols $\{x:P(x)\}$).  
\end{itemize}
\end{frame}

\begin{frame}
\frametitle{Russell's paradox}
\begin{itemize}
\item The problem with this assumption is that it leads to a contradiction.
\item Some sets are members of themselves, e.g. the set of all abstract ideas. Other sets are not, e.g. the set of all chocolate biscuits. 
\item So `not being a member of itself' is a property of sets.
\item Let $X$ be the set of all sets that are not members of themselves.
\item Suppose $X$ is a member of itself. Then, by the definition of $X$, it must also not be a member of itself.
\item On the other hand, if $X$ is not a member of itself, then, again by definition of $X$, it must be a member of itself after all.
\item This contradiction reveals that not all properties can define sets.
\end{itemize}
\end{frame}

\begin{frame}
\frametitle{The consequences of Russell's paradox}
\begin{itemize}
\item Russell's paradox tells us we can't naively assume that every property defines a set.
\vspace{0.2cm}
\item So to be safe mathematicians must restrict the notion of `set'. This gives us what we know as $ZFC$ set theory, which we can define as a theory in first-order logic.
\vspace{0.2cm}
\item This is a hack, but it produces enough sets for the (most of) the needs of mathematicians, and seems to block the paradoxes of naive set theory (Russell's paradox is just one of these).
\vspace{0.2cm}
\item To avoid his paradox, Russell and others developed the foundations of mathematical logic in the early 20th century.  
\end{itemize}
\end{frame}

\begin{frame}
\frametitle{Comparing the sizes of sets}
\begin{itemize}
\item If $X,Y$ are sets, a function $f:X\to Y$ is:
\begin{itemize}
\item \emph{1-1 (injective)} if for all $y\in Y$ there is \emph{at most} one $x\in X$ with $f(x) = y$.
\item \emph{onto (surjective)} if for all $y\in Y$ there is \emph{at least} one $x\in X$ with $f(x) = y$.
\item \emph{bijective} if it is both 1-1 and onto.
\end{itemize}
\item We say $X$ is \emph{at most as big} as $Y$ if there is a 1-1 function $f:X\to Y$.
\item We write $|X|\leq |Y|$.
\item If $X$ and $Y$ are finite then $|X|\leq|Y|$ according to this definition if and only if $X$ is actually at most as big as $Y$ as we usually understand it.
\begin{itemize}
\item Because if $X=\{x_1,\ldots,x_k\}$ and $Y=\{y_1,\ldots,y_n\}$ with $k\leq n$ we can define $f:X\to Y$ by $f(x_i) = y_i$ for $i\in \{1,\ldots,k\}$. 
\end{itemize}
\end{itemize}
\end{frame}

\begin{frame}
\frametitle{Defining cardinality}
\begin{fact}\mbox{}
\begin{enumerate}
\item $|X|\leq|Y|\iff$ there is an onto (surjective) function from $Y$ to $X$.
\item (Cantor-Bernstein theorem). $|X|=|Y|\iff$ there is a bijection between $X$ and $Y$.
\item Given two sets $X$ and $Y$, either $|X|\leq |Y|$ or $|Y|\leq|X|$, or both.
\end{enumerate}
\end{fact}

\begin{definition}[cardinality]
We define the \emph{cardinality} of $X$ to be the equivalence class defined by $|X|$.
\end{definition}
\end{frame}

\begin{frame}
\frametitle{Why is this useful?}
\begin{itemize}
\item This definition of cardinality agrees with the usual one for finite sets.
\vspace{0.3cm}
\item I.e., if $X$ and $Y$ are finite then $|X|=|Y|$ if and only if $X$ and $Y$ have the same number of elements.
\vspace{0.3cm}
\item So, for example, if $X$ has 3 elements, then $|X|$ contains every set that has 3 elements. 
\vspace{0.3cm}
\item We can use this as a definition for the number 3.
\vspace{0.3cm}
\item But this definition also applies to infinite sets, for example $\bN$.
\vspace{0.3cm}
\item $\bN$ is obviously bigger than every finite set, but what about other infinite sets?
\end{itemize}
\end{frame}

\begin{frame}
\frametitle{$\bN$ and $\bZ$}
\begin{theorem}
$|\bN|=|\bZ|$.
\end{theorem}
\begin{proof}
We define a bijection $f:\bZ\to\bN$ as follows.
\[f(z)=\begin{cases}2z \text{ when $z\geq 0$} \\
2|z| - 1 \text{ when $z<0$} \end{cases}\]
\end{proof}
\end{frame}

\begin{frame}
\frametitle{$\bN$ and $\bN\times \bN$}
\begin{theorem}
$|\bN|=|\bN\times\bN|$.
\end{theorem}
\begin{proof}
\begin{itemize}
\item $f:\bN\to\bN\times\bN$ defined by $f(n)=(n,n)$ is clearly 1-1. 
\item If we define a 1-1 function $g:\bN\times \bN\to \bN$ then fact 1(2) says $|\bN|=|\bN\times\bN|$. We get $g$ by listing the elements of $\bN\times \bN$: 
\end{itemize}
\xymatrixrowsep{.3cm}
\xymatrixcolsep{.3cm}
\[\xymatrix{ 
\bullet_{(0,4)} & \bullet_{(1,4)}\ar@{.>}[dr] & \bullet_{(2,4)} & \bullet_{(3,4)}\ar@{.>}[dr] & \bullet_{(4,4)}\\
\bullet_{(0,3)}\ar@{.>}[dr]  & \bullet_{(1,3)}\ar@{.>}[ul] & \bullet_{(2,3)}\ar@{.>}[dr]  & \bullet_{(3,3)}\ar@{.>}[ul] & \bullet_{(4,3)} \\
\bullet_{(0,2)}\ar@{.>}[u]  & \bullet_{(1,2)}\ar@{.>}[dr]  & \bullet_{(2,2)}\ar@{.>}[ul] & \bullet_{(3,2)}\ar@{.>}[dr]  & \bullet_{(4,2)}\ar@{.>}[ul]  \\
\bullet_{(0,1)}\ar@{.>}[dr]  & \bullet_{(1,1)}\ar@{.>}[ul]  & \bullet_{(2,1)}\ar@{.>}[dr]  & \bullet_{(2,1)}\ar@{.>}[ul] & \bullet_{(4,1)} \\
\bullet_{(0,0)}\ar@{.>}[u] & \bullet_{(1,0)}\ar@{.>}[r]  & \bullet_{(2,0)}\ar@{.>}[ul] & \bullet_{(3,0)}\ar@{.>}[r]  & \bullet_{(4,0)}\ar@{.>}[ul]  }\]
\end{proof}
\end{frame}

\begin{frame}
\frametitle{$\bN$ and $\bQ$}
\begin{corollary}
$|\bN|=|\bQ|$.
\end{corollary}
\begin{proof}
\begin{itemize}
\item Since $\bN\subset\bQ$ the inclusion function is an injection from $\bN$ to $\bQ$, so we just need to find a 1-1 function $\bQ\to\bN$. 
\item Let $h:\bQ\to\bZ\times\bZ$ be defined by 
\[h(q) = \begin{cases}
(0,0) \text{ when $q = 0$}\\
(a,b) \text{ when $\frac{a}{b}$ is the most reduced form of $q$ }
\end{cases}\]
\item Let $g:\bN\times \bN\to \bN$ be the function from theorem 4.
\item Let $f_1,f_2:\bZ\to \bN$ be copies of the function $f$ from theorem 3.  
\item Then $g\circ(f_1,f_2)\circ h:\bQ\to \bN$ is 1-1 as $g$, $f$ and $h$ are.
\end{itemize}
\[\xymatrix{\bQ\ar[r]^h & \bZ\times \bZ\ar[r]^{(f_1,f_2)} & \bN\times \bN\ar[r]^g & \bN}\]
\end{proof}
\end{frame}

\begin{frame}
\frametitle{$\bN$ and $\bR$}
\begin{theorem}
$|\bN|<|\bR|$.
\end{theorem}
\emph{Proof:}
\begin{itemize}
\item Since $\bN\subset \bR$ we know that $|\bN|\leq |\bR|$ as the inclusion function is 1-1. 
\item We will show that $|\bN|\neq |\bR|$ by proving that there is no onto function from $\bN$ to $\bR$. 
\item Let $f$ be a function from $\bN$ to the interval $(0,1)\subset \bR$.
\item We will show that there is an $x\in(0,1)$ such that $f(n)\neq x$ for all $n\in\bN$. 
\item This proof technique is known as \emph{Cantor's diagonal argument}, or just \emph{the diagonal argument}. 
\end{itemize}
\end{frame}

\begin{frame}
\frametitle{Theorem 6 proof continued}
\begin{itemize}
\item Every number in $(0,1)$ can be expressed as an infinite decimal expansion, e.g. $0.x_1x_2x_3\ldots$, where $x_n$ is the $n$th digit. 
\item Define $y=0.y_1y_2y_3\ldots$ by defining the digits as follows: 
\[y_n = \begin{cases} 7 \text{ if the $n$th digit of $f(n)$ is not $7$}\\
3  \text{ if the $n$th digit of $f(n)$ is $7$}\end{cases}\]  
\item Then, by definition, the $n$th digit of $y$ is different from the $n$th digit of $f(n)$ for all $n$, and so $y\neq f(n)$ for all $n\in\bN$.
\item So $f$ cannot be onto.
\item Since there's no onto function $\bN\to (0,1)$, there's no onto function $\bN\to \bR$ either.
\item So $|\bN|<|\bR|$.
\end{itemize}
\end{frame}

\begin{frame}
\frametitle{Countable and uncountable sets}
\begin{itemize} 
\item We have seen there's at least one set bigger than $\bN$.
\vspace{.3cm}
\begin{definition}[countable]
A set $X$ is \emph{countable} if $|X|\leq|\bN|$. Otherwise it is \emph{uncountable}.
\end{definition}
\vspace{.3cm}
\item It turns out that there's a never ending increasing hierarchy of uncountable cardinals. 
\vspace{.3cm}
\item You'll see a justification for this by thinking about powersets in the exercises.
\vspace{.3cm}
\item This is just the tip of the iceberg.
\vspace{.3cm}
\item Understanding this hierarchy is part of the work of modern set theorists.
\end{itemize}
\end{frame}

\begin{frame}
\frametitle{Cardinal arithmetic}
\begin{itemize} 
\item Given disjoint sets $X$ and $Y$, we extend the familiar arithmetic operations as follows:
\vspace{.3cm}
\begin{itemize}
\item $|X|+|Y|=|X\cup Y|$.
\vspace{.3cm}
\item $|X|\times |Y| = |X\times Y|$.
\vspace{.3cm}
\item $|X|^{|Y|} = |X^Y|$ (here $X^Y$ stands for the set of functions from $Y$ to $X$).
\end{itemize}
\vspace{.3cm}
\item You'll see in the exercises that these operations agree with the usual ones for finite sets.
\end{itemize}
\end{frame}

\begin{frame}
\frametitle{Powersets and exponentials}

\begin{proposition}
If $X$ is a set, then $|\wp(X)|=|2^X|$, where $2$ is the two element set $\{0,1\}$.
\end{proposition}
\begin{proof}
\begin{itemize}
\item We define a bijection $g$ from $\wp(X)$ to $2^X$ by $g(S) = f_S$, where $f_S:X\to\{0,1\}$ is defined by setting 
\[f_S(x)=\begin{cases} 1 \text{ when $x\in S$} \\
0 \text{ otherwise.}\end{cases}\]
\item $f_S$ is known as the \emph{characteristic function} of $S$.
\item $g$ is well defined because every set $S\subseteq X$ defines a unique $f_S$. 
\item It is clearly 1-1, and it is onto because given $f:X\to 2$ we can define $S_f=\{x\in X: f(x)=1\}$, and then $g(S_f)=f$.
\end{itemize}
\end{proof}
\end{frame}

\begin{frame}
\frametitle{The continuum hypothesis}

\begin{fact}\label{Fa:R}
$|\bR|=|2^\bN|$.
\end{fact}

\begin{itemize} 
\item We know that $|\bN|<|\bR| = |2^\bN|$. 
\item Is there a set $Y$ such that $|\bN|<|Y|<|\bR|$?. 
\item Cantor, the founder of set theory, believed the answer is no. 
\item This idea that there is no such $Y$ is the \emph{continuum hypothesis}.
\item It turns out that the continuum hypothesis ($CH$) can neither be proved nor disproved using the $ZFC$ axioms. 
\item G\"odel showed that it can not be disproved in 1940, and, in 1963, Cohen showed that it can not be proved either. 
\item The basic idea is that there are models of $ZFC$ where $CH$ is true, and others where it is false. 
\end{itemize}
\end{frame}
\end{document}