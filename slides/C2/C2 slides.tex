\documentclass[handout]{beamer} 
\title{ITCS 531: Counting - Introduction to enumerative combinatorics}
\date{}
\author{Rob Egrot}

\usepackage{amsmath, bbold, bussproofs,graphicx}
\usepackage{mathrsfs}
\usepackage{amsthm}
\usepackage{amssymb}
\usepackage[all]{xy}
\usepackage{multirow}
\usepackage{tikz-cd}


\newtheorem{proposition}[theorem]{Proposition}
\newcommand{\bN}{\mathbb{N}}
\newcommand{\bZ}{\mathbb{Z}}
\newcommand{\bQ}{\mathbb{Q}}
\newcommand{\bR}{\mathbb{R}}
\newcommand{\bP}{\mathbb{P}}

\addtobeamertemplate{navigation symbols}{}{%
    \usebeamerfont{footline}%
    \usebeamercolor[fg]{footline}%
    \hspace{1em}%
    \insertframenumber/\inserttotalframenumber
}
\setbeamertemplate{theorems}[numbered]
\begin{document}

\begin{frame}
\titlepage
\end{frame}

\begin{frame}
\frametitle{What is enumerative combinatorics?}
\begin{itemize}
\item Enumerative combinatorics is the art of counting in finite sets.\vspace{0.5cm}
\item E.g. how many ways can we choose 3 balls from a bag of 20 balls?\vspace{0.2cm}
\begin{itemize} 
\item This is an easy question.
\end{itemize}\vspace{0.5cm}
\item How many $n\times n$ matrices there are whose entries are $0$ or $1$ and such that every row and every column contains exactly 3 ones.
\begin{itemize} \vspace{0.2cm}
\item This is a very hard question for most values of $n$.
\end{itemize}
\end{itemize}
\end{frame}

\begin{frame}
\frametitle{What will we cover?}
\begin{itemize}
\item We first cover some important basic results producing formulas we can use to easily count things like combinations and permutations (e.g. to answer the `balls from a bag' question). \vspace{0.5cm}
\item Then we'll introduce the \emph{pigeon hole principle} with several examples of applications. \vspace{0.5cm}
\item We will briefly introduce the subject of `Ramsey numbers'. \vspace{0.5cm}
\item Finally we will look at a more difficult version of the `balls from a bag' question.  
\end{itemize}
\end{frame}

\begin{frame}
\frametitle{Inclusion exclusion}
\begin{proposition}
If $A$ and $B$ are finite sets, then $|A\cup B|=|A|+|B|-|A\cap B|$.
More generally, if $A_1,\ldots,A_n$ are finite sets, then
\begin{align*}|A_1\cup\ldots\cup A_n| & =\sum_{i=1}^n |A_i| \\
&- \sum _{i_1\neq i_2} |A_{i_1}\cap A_{i_2}| \\
&+ \sum_{i_1\neq i_2\neq i_3} |A_{i_1}\cap A_{i_2}\cap A_{i_3}|\\
&.\\
&+ (-1)^{k-1} \sum _{i_1\neq\ldots\neq i_k}|A_{i_1}\cap\ldots \cap A_{i_k}|\\
&.\\
&+(-1)^{n-1} |A_1\cap\ldots \cap A_n|.
\end{align*}
\end{proposition}
\end{frame}

\begin{frame}
\frametitle{Inclusion exclusion - proof}
\begin{itemize}
\item The general version can be proved by induction on $n$.
\item The base case is easy. For the inductive step we start by noticing that 
\begin{align*}
&|A_1\cup\ldots\cup A_n|\\ 
=& |(A_1\cup\ldots\cup A_{n-1})\cup A_n| \\
=& |A_1\cup\ldots\cup A_{n-1}| +|A_n| - |(A_1\cup\ldots\cup A_{n-1})\cap A_n |\\
=& |A_1\cup\ldots\cup A_{n-1}| +|A_n| - |(A_1\cap A_n)\cup\ldots\cup (A_{n-1}\cap A_n)|. 
\end{align*}
\item By the inductive hypothesis the claimed formula works for $|A_1\cup\ldots\cup A_{n-1}|$ and $|(A_1\cap A_n)\cup\ldots\cup (A_{n-1}\cap A_n)|$. 
\item We can check that applying the formula for these expressions ans adding $|A_n|$ gives us what we want.
\end{itemize}
\end{frame}

\begin{frame}
\frametitle{Permutations and combinations}
\begin{proposition}
Let $k\leq n\in\bN$. Then:
\begin{enumerate}
\item The number of ways we can select $k$ objects from a set of $n$ objects, where the order of selection is important, is given by the formula
\[P(n,k) = \frac{n!}{(n-k)!}.\]
\item The number of ways we can select $k$ objects from a set of $n$ objects, where the order of selection is \emph{not} important, is given by the formula
\[C(n,k) = {n \choose k} = \frac{n!}{(n-k)! k!}.\]
\end{enumerate}
\end{proposition}
\end{frame}

\begin{frame}
\frametitle{Permutations and combinations - proof}
\begin{itemize}
\item $\frac{n!}{(n-k)!}$ because we have $n$ possibilities for the first selection, $n-1$ for the second etc. 
\item For the $k$th selection when we have $(n-(k-1))$ possibilities. 
\item Thus we have $n\times (n-1)\times\ldots\times (n-(k-1)) = \frac{n!}{(n-k)!}$ total.\vspace{0.5cm}

\item $\frac{n!}{(n-k)! k!}$ because if we don't care about the order, an ordered selection of $k$ objects is equivalent to all the other selections of the same objects but in a different order. 
\item There are $k!$ different ways to order a collection of $k$ elements. 
\item So we get the formula for $C(n,k)$ by dividing the formula for $P(n,k)$ by $k!$.  
\end{itemize}
\end{frame}

\begin{frame}
\frametitle{Pigeon hole principle}
\begin{lemma}
If $k < n$ and you have $n$ balls in $k$ bags, there must be at least one bag containing at least two balls. More precisely, there must be at least one bag containing at least $\lceil \frac{n}{k}\rceil$ balls.
\end{lemma}
\begin{itemize}
\item This lemma gets its name from the fact that it is often stated in terms of pigeons and pigeon holes, rather than balls and bags. 
\item The following is a restatement of the pigeon hole principle that can be more useful in some situations.
\end{itemize}
\begin{lemma}\label{L:Dij}
In any finite collection of natural numbers, the maximum must be at least as large as the mean, and the minimum must be at most as large as the mean.
\end{lemma}
\end{frame}

\begin{frame}
\frametitle{Pigeon hole principle examples}
\begin{example}
If you choose five distinct numbers between 1 and 8, then two of those numbers must sum to 9.
\end{example}
\begin{proof}
\begin{itemize}
\item The four sets $\{1,8\}$, $\{2,7\}$, $\{3,6\}$, $\{4,5\}$ partition $\{1,\ldots,8\}$. \vspace{0.1cm}
\item Each one of our five numbers must be in one of these sets. \vspace{0.3cm}
\item So there must be one set containing two, and thus two elements that sum to 9.
\end{itemize}
\end{proof}
\end{frame}

\begin{frame}
\frametitle{Pigeon hole principle examples}
\begin{example}
In a city of 200,000 people, at least 547 people will have the same birthday. 
\end{example}
\begin{proof}
\begin{itemize}
\item There are 366 possible birthdays (including leap years). \vspace{0.3cm}
\item Since there are 200,000 people, the average number of people born on a day will be $\frac{200,000}{366}= 546.45$. \vspace{0.3cm}
\item By lemma \ref{L:Dij}, the day that has the most birthdays must have a larger number of birthdays than this, so at least 547.
\end{itemize}
\end{proof}
\end{frame}

\begin{frame}
\frametitle{Pigeon hole principle examples}
\begin{example}
For every integer $n$ there is a multiple of $n$ that has only $0$s and $1s$ in its decimal expansion.
\end{example}
\begin{proof}
\begin{itemize}
\item Consider the numbers $x_1,x_2,\ldots,x_{n}$, where $x_1=1$, $x_2=11$, and $x_k$ is $1$ repeated $k$ times. \vspace{0.2cm}
\item There are $n-1$ non-zero values in $\bZ_n$, so either $n|x_k$ for some $k$ (in which case we are done), or there are $i<j\leq n$ such that the value of $x_i \mod n$ is the same as the value of $x_j \mod n$. \vspace{0.2cm}
\item But then $n|(x_j-x_i)$, and $x_j-x_i$ has the required form, so the proof is complete.
\end{itemize}
\end{proof}
\end{frame}

\begin{frame}
\frametitle{Pigeon hole principle examples}
\begin{example}
A baseball team plays every day for 30 days. They can play more than once each day, but they play at most 45 games in total. There is some period of consecutive days where they play exactly 14 games.
\end{example}
\end{frame}

\begin{frame}
\frametitle{Pigeon hole principle examples - proof for example 8}
\begin{itemize}
\item Let $a_j$ be the number of games played up to and including the $j$th day. 
\item Then $a_1,a_2,\ldots,a_{30}$ is strictly increasing with max 45. 
\item $a_1 + 14,a_2+14,\ldots, a_{30}+14$ is strictly increasing, max 59. 
\item Combining these two sequences gives us 60 elements, each with values between 1 and 59. 
\item By the pigeon hole principle there must be two terms in the sequence with the same value.
\item Since the team plays everyday, one term must be from the 1st half, and the other from the 2nd half. 
\item I.e. there there must be $i < j$ with $a_j=a_i+14$. 
\item But this just means that exactly 14 games are played between the $i$th day and the $j$th day, which is what we want to prove. 
\end{itemize}
\end{frame}

\begin{frame}
\frametitle{Pigeon hole principle examples}
\begin{example}
If we have $n+1$ positive integers, each less than or equal to $2n$, there must be one number that divides another one.
\end{example}
\begin{proof}
\begin{itemize}
\item Every positive integer can be written as $q2^k$, where $q$ is an odd number and $k$ is some natural number. 
\begin{itemize}
\item (Induction) This is obviously true when $n=1$, so let $n>1$.  
\item If $n$ is odd there is nothing to prove, so suppose $n = 2l$ for some $l$. 
\item Then $l=q2^k$ by the inductive hypothesis, and so $n=q2^{k+1}$. 
\end{itemize}
\item Now, there are only $n$ odd numbers less than or equal to $2n$. 
\item So, given a list of $n+1$ numbers there must be numbers $a\neq b$ in the list with $a=q2^{k_1}$, and $b= q2^{k_2}$ for the same $q$. 
\item If $k_1< k_2$ then $a|b$, otherwise $b|a$. 
\end{itemize}
\end{proof}
\end{frame}

\begin{frame}
\frametitle{Pigeon hole principle examples}
\begin{example}
In any group of more than 2 people, at least two people must have the same number of friends (assuming friendship is symmetric). 
\end{example}
\begin{proof}
Suppose there are $n$ people, and $n\geq 2$. Then each person can have between 0 and $n-1$ friends. There are two cases.
\begin{enumerate}
\item Everyone has at least one friend. In this case each person has between 1 and $n-1$ friends, so there are $n$ people and $n-1$ possibilities, so at least two people must have the same number of friends.
\item Someone has no friends. In this case each person has between 0 and $n-2$ friends, so there are again $n$ people and $n-1$ possibilities.
\end{enumerate}
\end{proof}
\end{frame}

\begin{frame}
\frametitle{Pigeon hole principle examples}
\begin{example}
In any sequence of $n^2+1$ distinct real numbers, there must either be a strictly increasing subsequence of size $n+1$, or a strictly decreasing subsequence of size $n+1$.
\end{example}
\end{frame}

\begin{frame}
\frametitle{Pigeon hole principle examples - proof for example 11}
\begin{itemize}
\item Suppose our set of numbers is $(a_0,a_1,\ldots,a_{n^2})$. 
\item For each $k\in\{0,\ldots,n^2\}$ define the pair $(i_k,d_k)$. 
\begin{itemize}
\item $i_k$ is the length of the longest strictly increasing subsequence starting at $a_k$.
\item $d_k$ is the length of the longest strictly decreasing subsequence starting at $a_k$. 
\end{itemize}
\item Suppose there are no strictly increasing or decreasing subsequences of size $n+1$. 
\item Then $i_k$ and $d_k$ are both less than or equal to $n$ for all $k$. 
\item Since the minimum possible value for $i_k$ and $d_k$ is 1, this means there are $n^2$ possible distinct values for $(i_k,d_k)$. 
\item But there are $n^2+1$ terms in the sequence, so there must be $l<k\in\{0,\ldots,n^2\}$ with $(i_l,d_l)=(i_k,d_k)$. 
\item But this is impossible, because if $a_l<a_k$ we must have $i_l>i_k$, and if $a_l > a_k$ we must have $d_l > d_k$. 
\end{itemize}
\end{frame}

\begin{frame}
\frametitle{Friends and enemies}
\begin{proposition}\label{P:Ramsey}
Suppose two people can either be friends or enemies. In any group of 6 people, either there are three mutual friends, or three mutual enemies.
\end{proposition}
\begin{proof}
\begin{itemize}
\item Let $x$ be some member of the group. 
\item Out of the five remaining people, there must either be three who are friends with $x$, or three who are not. 
\item Suppose wlog there are three people who are friends with $x$. 
\item If any two of them are friends with each other then this provides a group of three mutual friends. 
\item If no two of them are friends then they are a group of three mutual enemies. 
\item In either case, we are done.
\end{itemize} 
\end{proof}
\end{frame}

\begin{frame}
\frametitle{Ramsey numbers}
\begin{definition}
Let $m$ and $n$ be natural numbers greater than or equal to 2. We define the \emph{Ramsey number} $R(m,n)$ to be the minimum number of people at a party so that there are either $m$ mutual friends, or $n$ mutual enemies.  
\end{definition}
\begin{itemize}
\item It's obvious that $R(m,n)=R(n,m)$, for all $m$ and $n$. 
\item By proposition \ref{P:Ramsey}, we know $R(3,3)\leq 6$ 
\item We can find a group of 5 where there are neither three mutual friends, nor three mutual enemies, so $R(3,3)=6$
\item In general, it is very difficult to find Ramsey numbers.
\item For example $R(4,4) = 18$, but $R(5,5)$ is only known to lie somewhere in the range 43-48.
\item $R(10,10)$ is only known to be between 798 and 23556. 
\end{itemize}
\end{frame}

\begin{frame}
\frametitle{Combinations with repetition}
\begin{theorem}\label{T:balls}
Suppose we have an infinite supply of balls in $n$ different colours. Suppose we choose $k$ balls, and the only distinguishing feature of the balls is their colour. Then there are ${n+k-1 \choose k}$ different possible outcomes if we don't care about the order the balls are chosen.
\end{theorem}
\end{frame}

\begin{frame}
\frametitle{Combinations with repetition - proof}
\begin{itemize}
\item We use a trick. 
\item Choosing $k$ balls in $n$ different colours is like putting $k$ different balls into $n$ different boxes. 
\item We will represent this graphically using $*$ to represent balls, and $|$ to represent the boundaries of the boxes. E.g.
\[**|*|***||*\]

\item Strings of $k$ stars and $n-1$ lines correspond exactly to possible choices.
\item Same number of choices as strings with $k$ stars and $n-1$ lines. 
\item We can think of this as starting with $n+k-1$ vertical lines, then choosing $k$ of them to change to stars. 
\item But this is just ${n+k -1 \choose k}$, which is what we aimed to prove.
\end{itemize}
\end{frame}



\end{document}