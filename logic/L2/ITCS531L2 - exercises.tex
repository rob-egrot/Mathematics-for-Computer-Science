\documentclass{article}

\usepackage{amsmath, mathrsfs, amssymb, stmaryrd, cancel, hyperref, relsize,tikz,amsthm,comment,bussproofs,xcolor}
\usepackage{graphicx}
\usepackage{xfrac}
\hypersetup{pdfstartview={XYZ null null 1.25}}
\usepackage[all]{xy}
\usepackage[normalem]{ulem}
\usepackage{tikz-cd}


\theoremstyle{plain}
\newtheorem{theorem}{Theorem}{\bfseries}{\itshape}
\newtheorem{proposition}[theorem]{Proposition}{\bfseries}{\itshape}
\newtheorem{definition}[theorem]{Definition}{\bfseries}{\upshape}
\newtheorem{lemma}[theorem]{Lemma}{\bfseries}{\upshape}
\newtheorem{example}[theorem]{Example}{\bfseries}{\upshape}
\newtheorem{corollary}[theorem]{Corollary}{\bfseries}{\upshape}
\newtheorem{remark}[theorem]{Remark}{\bfseries}{\upshape}
\newtheorem{fact}[theorem]{Fact}{\bfseries}{\upshape}
\newtheorem{Q}[theorem]{Exercise}{\bfseries}{\upshape}

\newtheorem*{theorem*}{Theorem}

% Uncomment below to create stand-alone file
% \newcommand*{\prefix}{}

\newcommand{\bN}{\mathbb{N}}
\newcommand{\bZ}{\mathbb{Z}}
\newcommand{\bQ}{\mathbb{Q}}
\newcommand{\bR}{\mathbb{R}}
\newcommand{\bP}{\mathbb{P}}
\newcommand{\HCF}{\mathbf{HCF}}
\newcommand{\lequiv}{\models\text{\reflectbox{$\models$}}}
\newcommand{\lra}{\leftrightarrow}
\newcommand{\ra}{\rightarrow}
\newcommand{\la}{\leftarrow}

\title{ITCS 531 \\Logic 2: Deduction rules for propositional logic exercises}
\author{Rob Egrot}
\date{}

\begin{document}
\maketitle
%\includecomment{comment}

\begin{Q}
The following deduction tree proves that $\phi\rightarrow\psi$ can be deduced from $\neg\phi\vee\psi$ in intuitionistic propositional logic. Add labels indicating the rules used at each stage.
\begin{prooftree}

\AxiomC{$\neg\phi\vee\psi$}

\AxiomC{$[\neg\phi]_1$}
\AxiomC{$[\phi]_2$}
\BinaryInfC{$\bot$}
\UnaryInfC{$\psi$}
\UnaryInfC{$\phi\rightarrow\psi$}

\AxiomC{$[\psi]_1$}
\UnaryInfC{$\psi$}
\AxiomC{$[\phi]_3$}
\BinaryInfC{$\phi\rightarrow\psi$}

\TrinaryInfC{$\phi\rightarrow\psi$}

\end{prooftree}
\end{Q}
\begin{comment}
\textbf{Solution:}
\begin{prooftree}

\AxiomC{$\neg\phi\vee\psi$}

\AxiomC{$[\neg\phi]_1$}
\AxiomC{$[\phi]_2$}
\LeftLabel{ $\color{red}(\neg_E)$}
\BinaryInfC{$\bot$}
\LeftLabel{ $\color{red}(\bot_E)$}
\UnaryInfC{$\psi$}
\LeftLabel{ $\color{red}(\rightarrow_I)$}
\UnaryInfC{$\phi\rightarrow\psi$}

\AxiomC{$[\psi]_1$}
\UnaryInfC{$\psi$}
\AxiomC{$[\phi]_3$}
\RightLabel{ $\color{red}(\rightarrow_I)$}
\BinaryInfC{$\phi\rightarrow\psi$}
\LeftLabel{ $\color{red}(\vee_E)$}
\TrinaryInfC{$\phi\rightarrow\psi$}

\end{prooftree}
\end{comment}

\begin{Q}
What is being proved in the following deduction tree? Add labels indicating the rules at each stage.
\begin{prooftree}
\AxiomC{$[\neg(\neg\phi\vee\psi)]_1$}
\AxiomC{$\phi\rightarrow\psi$}
\AxiomC{$[\phi]_2$}
\BinaryInfC{$\psi$}
\UnaryInfC{$\neg\phi\vee\psi$}
\BinaryInfC{$\bot$}
\UnaryInfC{$\neg\phi$}
\UnaryInfC{$\neg\phi\vee\psi$}
\AxiomC{$[\neg(\neg\phi\vee\psi)]_1$}
\UnaryInfC{$\neg(\neg\phi\vee\psi)$}
\BinaryInfC{$\bot$}
\UnaryInfC{$\neg\neg(\neg\phi\vee\psi)$}
\UnaryInfC{$\neg\phi\vee\psi$}
\end{prooftree}
\end{Q}
\begin{comment}
\textbf{Solution:}
This tree proves that $\neg\phi\vee\psi$ can be deduced from $\phi\rightarrow\psi$ in \emph{classical} propositional logic. 
\begin{prooftree}
\AxiomC{$[\neg(\neg\phi\vee\psi)]_1$}
\AxiomC{$\phi\rightarrow\psi$}
\AxiomC{$[\phi]_2$}
\RightLabel{ $\color{red}(\rightarrow_E)$}
\BinaryInfC{$\psi$}
\RightLabel{ $\color{red}(\vee_{I_r})$}
\UnaryInfC{$\neg\phi\vee\psi$}
\LeftLabel{ $\color{red}(\neg_E)$}
\BinaryInfC{$\bot$}
\LeftLabel{ $\color{red}(\neg_I)$}
\UnaryInfC{$\neg\phi$}
\LeftLabel{ $\color{red}(\vee_{I_l})$}
\UnaryInfC{$\neg\phi\vee\psi$}
\AxiomC{$[\neg(\neg\phi\vee\psi)]_1$}
\UnaryInfC{$\neg(\neg\phi\vee\psi)$}
\RightLabel{ $\color{red}(\neg_E)$}
\BinaryInfC{$\bot$}
\RightLabel{ $\color{red}(\neg_I)$}
\UnaryInfC{$\neg\neg(\neg\phi\vee\psi)$}
\RightLabel{ $\color{red}(\neg\neg_E)$}
\UnaryInfC{$\neg\phi\vee\psi$}
\end{prooftree}
\end{comment}

\begin{Q}
Show that $(\phi\wedge\psi)\rightarrow(\psi\wedge\phi)$ can be deduced from an empty set of axioms.
\end{Q}
\begin{comment}
\textbf{Solution:}
\begin{prooftree}
\AxiomC{$[\phi \wedge \psi]_1$}
\UnaryInfC{$\phi\wedge\psi$}
\LeftLabel{ $\color{red}(\wedge_{E_r})$}
\UnaryInfC{$\psi$}
\AxiomC{$[\phi \wedge \psi]_1$}
\UnaryInfC{$\phi\wedge\psi$}
\RightLabel{ $\color{red}(\wedge_{E_l})$}
\UnaryInfC{$\phi$}
\LeftLabel{ $\color{red}(\wedge_{I})$}
\BinaryInfC{$\psi\wedge\phi$}
\LeftLabel{ $\color{red}(\rightarrow_{I})$}
\UnaryInfC{$(\phi \wedge \psi)\rightarrow(\psi\wedge\phi)$}
\end{prooftree}
\end{comment}



\begin{Q}
Show that we can deduce $\phi\wedge(\psi\vee\chi)$ if we start with $(\phi\wedge \psi)\vee (\phi\wedge \chi)$.
\end{Q}
\begin{comment}
\textbf{Solution:}
To save space let $\theta =(\phi\wedge \psi)\vee (\phi\wedge \chi)$.
\begin{prooftree}
\AxiomC{$\theta$}
\AxiomC{$[\phi\wedge\psi]$}
\LeftLabel{ $\color{red}(\wedge_{E_l})$}
\UnaryInfC{$\phi$}
\AxiomC{$[\phi\wedge\chi]$}
\RightLabel{ $\color{red}(\wedge_{E_l})$}
\UnaryInfC{$\phi$}
\LeftLabel{ $\color{red}(\vee_E)$}
\TrinaryInfC{$\phi$}

\AxiomC{$\theta$}
\AxiomC{$[\phi\wedge\psi]$}
\LeftLabel{ $\color{red}(\wedge_{E_r})$}
\UnaryInfC{$\psi$}
\LeftLabel{ $\color{red}(\vee_{I_l})$}
\UnaryInfC{$\psi\vee\chi$}
\AxiomC{$[\phi\wedge\chi]$}
\RightLabel{ $\color{red}(\wedge_{E_r})$}
\UnaryInfC{$\chi$}
\RightLabel{ $\color{red}(\vee_{I_r})$}
\UnaryInfC{$\psi\vee\chi$}
\RightLabel{ $\color{red}(\vee_E)$}
\TrinaryInfC{$\psi\vee\chi$}
\LeftLabel{ $\color{red}(\wedge_I)$}
\BinaryInfC{$\phi\wedge(\psi\vee\chi)$}
\end{prooftree}
\end{comment}

\begin{Q}
Show that we can deduce $(\phi\wedge \psi)\vee (\phi\wedge \chi)$ if we start with $\phi\wedge(\psi\vee\chi)$.
\end{Q}
\begin{comment}
\textbf{Solution:}
\begin{prooftree}

\AxiomC{$\phi\wedge (\psi\vee \chi)$}
\LeftLabel{ $\color{red}(\wedge_{E_r})$}
\UnaryInfC{$\psi\vee \chi$}

\AxiomC{$\phi\wedge (\psi\vee \chi)$}
\LeftLabel{ $\color{red}(\wedge_{E_l})$}
\UnaryInfC{$\phi$}

\AxiomC{$[\psi]_1$}
\UnaryInfC{$\psi$}
\LeftLabel{ $\color{red}(\wedge_I)$}
\BinaryInfC{$\phi\wedge\psi$}
\LeftLabel{ $\color{red}(\vee_{I_l})$}
\UnaryInfC{$(\phi\wedge\psi)\vee(\phi\wedge\chi)$}

\AxiomC{$\phi\wedge (\psi\vee \chi)$}
\RightLabel{ $\color{red}(\wedge_{E_l})$}
\UnaryInfC{$\phi$}

\AxiomC{$[\chi]_1$}
\UnaryInfC{$\chi$}
\RightLabel{ $\color{red}(\wedge_I)$}
\BinaryInfC{$\phi\wedge\chi$}
\RightLabel{ $\color{red}(\vee_{I_r})$}
\UnaryInfC{$(\phi\wedge\psi)\vee(\phi\wedge\chi)$}


\LeftLabel{ $\color{red}(\vee_E)$}
\TrinaryInfC{$(\phi\wedge\psi)\vee(\phi\wedge\chi)$}

\end{prooftree}
\end{comment}

\end{document}