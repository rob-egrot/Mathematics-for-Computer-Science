\documentclass{article}

\usepackage{amsmath, mathrsfs, amssymb, stmaryrd, cancel, hyperref, relsize,tikz,amsthm,comment,enumerate}
\usepackage{graphicx}
\usepackage{xfrac}
\hypersetup{pdfstartview={XYZ null null 1.25}}
\usepackage[all]{xy}
\usepackage[normalem]{ulem}
\usepackage{tikz-cd}


\theoremstyle{plain}
\newtheorem{theorem}{Theorem}{\bfseries}{\itshape}
\newtheorem{proposition}{Proposition}{\bfseries}{\itshape}
\newtheorem{definition}{Definition}{\bfseries}{\upshape}
\newtheorem{lemma}{Lemma}{\bfseries}{\upshape}
\newtheorem{example}{Example}{\bfseries}{\upshape}
\newtheorem{corollary}{Corollary}{\bfseries}{\upshape}
\newtheorem{remark}{Remark}{\bfseries}{\upshape}
\newtheorem{fact}{Fact}{\bfseries}{\upshape}
\newtheorem{Q}{Exercise}{\bfseries}{\upshape}

\newtheorem*{theorem*}{Theorem}

\newcommand{\bN}{\mathbb{N}}
\newcommand{\bZ}{\mathbb{Z}}
\newcommand{\bQ}{\mathbb{Q}}
\newcommand{\bR}{\mathbb{R}}
\newcommand{\bP}{\mathbb{P}}
\newcommand{\HCF}{\mathbf{HCF}}
\DeclareMathOperator{\ISp}{\downarrow \mathit{p}}
\DeclareMathOperator{\ISq}{\downarrow \mathit{q}}



\title{ITCS 531 \\Counting 3: Ordinal numbers exercises}
\author{Rob Egrot}
\date{}

\begin{document}
\maketitle
%\includecomment{comment}

\begin{Q}\label{Q:injective}
Prove that an order embedding must be 1-1. Give a simple example of an order preserving 1-1 function between posets that is not an order embedding.
\end{Q} 
\begin{comment}
\textbf{Solution:}

\end{comment} 

\begin{Q}\label{Q:init}
Let $P$ be a well ordered set and let $p\in P$. Suppose $f:P\to \ISp$ is an order preserving map.
\begin{enumerate}[(a)]
\item Let $S=\{q\in P: f(q)< q\}$. Prove that $S$ has a smallest element.
\item Let $q'$ be the smallest element of $S$. Prove that $f(f(q')) = f(q')$.
\item Explain why this proves lemma 1.11.
\end{enumerate} 
\end{Q}
\begin{comment}
\textbf{Solution:}
Let $f:P\to \ISp$ be an order preserving map. Let $S=\{q\in P: f(q)< q\}$. Then $S$ is non-empty as it must contain $p$, so, as $P$ is well ordered, $S$ has a least element, $q'$ say. Since $f(q')<q$, we have $f(q')\notin S$, and so $f(f(q'))\not< f(q')$. But, as $f$ is order preserving we have $f(f(q'))\leq f(q')$, and thus $f(f(q'))= f(q')$. Thus $f$ is not 1-1, and so cannot be an order embedding.
\end{comment}

\begin{Q}\label{Q:successor}
Let $P$ and $Q$ be well orders, let $p\in P$, and suppose $h:P\to Q$ is an order isomorphism. Suppose $p^+$ exists in $P$.
\begin{enumerate}[(a)]
\item Suppose there is $q\in Q$ with $h(p)<q<h(p^+)$. Prove there must be $p'\in P$ with $p<p'<p^+$.
\item Explain why this proves lemma 1.12.
\end{enumerate}
\end{Q}
\begin{comment}
\textbf{Solution:}
Suppose $p$ is not the greatest element of $P$. Then $p^+$ exists. Suppose there is $q\in Q$ with $h(p)< q < h(p^+)$. Then there is $p'\in P$ with $h(p')= q$. But as $h(p)< q$ we must have $p < p'$, and as $q < h(p^+)$ we must have $p' < p^+$. So $p< p' < p^+$. But this contradicts the definition of $p^+$. 
\end{comment}

\begin{Q}\label{Q:isom}
Let $P$ and $Q$ be well orders, let $p\in P$, let $q\in Q$, and suppose $h:\ISp \to \ISq$ is an order isomorphism. Let $p'< p$.
\begin{enumerate}[(a)]
\item Prove that $h$ restricts to an order isomorphism from $\ISp'$ to $\downarrow h(p')$.
\item Let $g$ be an order isomorphism from $\ISp'$ to $\downarrow h(p')$. Define $S = \{x\in \ISp' : g(x)\neq h(x)\}$, and suppose $S$ is non-empty. Let $x'$ be the minimal element of $S$.
\begin{enumerate}[(i)]
\item Suppose $h(x')< g(x')$. Show that there is $y\in \ISp'$ with $x' < y$ and $h(y) = g(x')$. 
\item Using this value $y$, prove that the assumption that $h(x')< g(x')$ leads to a contradiction. 
\end{enumerate}
\item Using the results of the previous parts of this exercise, prove lemma 1.13.
\end{enumerate}
\end{Q}
\begin{comment}
\textbf{Solution:}
Clearly $h$ restricts to an order isomorphism as required, so all we must do is prove that this is only such order isomorphism. So let $g$ be an order isomorphism from $\ISp'$ to $\downarrow h(p')$, and suppose $\{x\in \ISp' : g(x)\neq h(x)\}\neq \emptyset$. Then, by the well ordering property, there is a least $x\in \ISp'$ such that $h(x)\neq g(x)$. Suppose $h(x)< g(x)$. Then, as $g$ is an order isomorphism, there must be $y\in P$ with $g(y) = h(x)$, and $y< x$. But then, by minimality of $x$, we must have $g(y)=h(y)\neq h(x)$, as $h$ is also an order isomorphism. This is, of course, a contradiction. If $g(x)< h(x)$ then we can make a similar argument.  
\end{comment}

\begin{Q}\label{Q:c}
Prove corollary 1.15.
\end{Q}
\begin{comment}
\textbf{Solution:}

\end{comment}

\end{document}