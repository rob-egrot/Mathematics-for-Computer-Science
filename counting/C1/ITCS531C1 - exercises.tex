\documentclass{article}

\usepackage{amsmath, mathrsfs, amssymb, stmaryrd, cancel, hyperref, relsize,tikz,amsthm,comment,enumerate}
\usepackage{graphicx}
\usepackage{xfrac}
\hypersetup{pdfstartview={XYZ null null 1.25}}
\usepackage[all]{xy}
\usepackage[normalem]{ulem}
\usepackage{tikz-cd}


\theoremstyle{plain}
\newtheorem{theorem}{Theorem}{\bfseries}{\itshape}
\newtheorem{proposition}{Proposition}{\bfseries}{\itshape}
\newtheorem{definition}{Definition}{\bfseries}{\upshape}
\newtheorem{lemma}{Lemma}{\bfseries}{\upshape}
\newtheorem{example}{Example}{\bfseries}{\upshape}
\newtheorem{corollary}{Corollary}{\bfseries}{\upshape}
\newtheorem{remark}{Remark}{\bfseries}{\upshape}
\newtheorem{fact}{Fact}{\bfseries}{\upshape}
\newtheorem{Q}{Exercise}{\bfseries}{\upshape}

\newtheorem*{theorem*}{Theorem}

\newcommand{\bN}{\mathbb{N}}
\newcommand{\bZ}{\mathbb{Z}}
\newcommand{\bQ}{\mathbb{Q}}
\newcommand{\bR}{\mathbb{R}}
\newcommand{\bP}{\mathbb{P}}
\newcommand{\HCF}{\mathbf{HCF}}

% Uncomment below to create stand-alone file
% \newcommand*{\prefix}{}



\title{ITCS 531 \\Counting 1: Cardinal numbers exercises}
\author{Rob Egrot}
\date{}

\begin{document}
\maketitle
%\includecomment{comment}

\begin{Q}
Prove that if $X$ and $Y$ are disjoint finite sets, then the cardinal arithmetic operations agree with the usual arithmetic operations on $|X|$ and $|Y|$. In other words, prove that $|X\cup Y|$ is equal to the result of adding $|X|$ and $|Y|$ as normal, and do similar for the other two arithmetic operations we have defined.
\end{Q}
\begin{comment}
\textbf{Solution:} \begin{enumerate}[]
\item $|X|+|Y|$: Since $X$ and $Y$ are disjoint, if $z\in X\cup Y$ then either $z\in X$, or $z\in Y$, but not both. So, if $|X| = m$ and $|Y|=n$, we have $|X\cup Y| = m+n = |X|+|Y|$ as required.
\item $|X|\times|Y|$: $X\times Y = \{(x,y):x\in X \text{ and } y\in Y\}$. If $|X| = m$ and $|Y|=n$ then $|X\times Y| = m\times n = |X|\times|Y|$.
\item $|X|^{|Y|}$: $X^Y$ is the set of all functions from $Y$ to $X$. How many functions are there? Well, such a function must map each element of $Y$ to exactly one element of $X$. So, for each $y\in Y$ there are exactly $|X|$ choices. So, if $|X| = m$ and $|Y|=n$, we get $m^n$ different functions. So $|X^Y|=m^n$ as required.
\end{enumerate}
\end{comment}

\begin{Q}
Let $X_i$ be countable for all $i\in \bN$, and suppose $X_i\cap X_j=\emptyset$ for all $i\neq j\in \bN$. Prove that $\bigcup_{i\in\bN} X_i$ is countable. HINT: We don't need the condition that $X_i\cap X_j=\emptyset$, but it makes the notation slightly simpler. Think about the proof that $|\bN|=|\bN\times\bN|$. 
\end{Q}
\begin{comment}
\textbf{Solution:} 
We need to find an injective function from $\bigcup_{i\in\bN} X_i$ to $\bN$. Since $\bN\times\bN$ is countable, there is an injective $f:\bN\times\bN\to \bN$. Since each $X_i$ is countable, there are injective functions $g_i:X_i\to \bN$ for all $i\in\bN$. Define a function $g:\bigcup_{i\in\bN} X_i\to \bN\times\bN$ by $g(x) = (i,g_i(x))$, where $x\in X_i$. This is well defined because $X_i\cap X_j = \emptyset$ for all $i\neq j$. Then $g$ is injective, because, given $x_1\in X_i$ and $x_2\in X_j$ with $x_1\neq x_2$, if $i\neq j$ then $(i,g_i(x_1))\neq (j,g_j(x_2))$, as $i\neq j$, and if $i=j$ then $g_i(x_1)\neq g_i(x_2)$, as $g_i$ is injective. So $f\circ g:\bigcup_{i\in\bN} X_i\to \bN$ is the composition of two injective functions, and so is injective. This gives us the injective function we need.   
\end{comment}

\begin{Q}
Let $X$ be a countable set. Prove that the set of all finite subsets of $X$ is countable. 
\end{Q}
\begin{comment}
\textbf{Solution:} 
Let $f:X\to \bN$ be injective. Arrange the prime numbers in a list as $p_0,p_1,\ldots$. We know from number theory that the set of primes is infinite, so this is a countably infinite list. Given $S=\{x_1,\ldots,x_n\}\subseteq X$, define $g(S)= p_{f(x_0)}\times p_{f(x_1)}\times\ldots\times p_{f(x_n)}$. Then $g$ is a function from the set of all finite subsets of $X$ to $\bN$. Moreover, $g$ is injective, because if $S_1\neq S_2$ then $g(S_1)$ and $g(S_2)$ will have different prime factorizations, and so we know from the Fundamental Theorem of Arithmetic that this means they must be different numbers.
\end{comment}

\begin{Q}\label{\prefix E:power}
Let $X$ be a set, let $\wp(X)$ be the powerset of $X$.
\begin{enumerate}
\item[a)] Define a simple injective function from $X$ to $\wp(X)$.
\item[b)] Prove that there is no surjective function from $X$ to $\wp(X)$. HINT: Suppose such a function exists. Can you derive a contradiction? The argument used here is similar to the one used in Russell's paradox. 
\item[c)] What does this tell us about the relationship between $|X|$ and $|\wp(X)|$?
\end{enumerate}
\end{Q}
\begin{comment}
\textbf{Solution:}
\begin{enumerate}[a)]
\item Use e.g. $x\mapsto \{x\}$.
\item Suppose $f:X\to \wp(X)$ is surjective. Let $Z=\{x\in X: x\notin f(x)\}$. Then, as $f$ is surjective, there is $z\in X$ with $f(z) = Z$. Suppose $z\in Z$. Then, by definition of $Z$ and $z$, we must have $z\notin Z$. On the other hand, if $z\notin Z$, then $z\in f(z)=Z$, which is also a contradiction. We conclude there can be no such surjective function.
\item We see that $|X|<|\wp(X)|$. In other words, $X$ is strictly smaller than $\wp(X)$.
\end{enumerate} 
\end{comment}


\end{document}