\documentclass{article}
\usepackage{amsmath, mathrsfs, amsthm, lmodern, amssymb,float,fancyhdr, color,enumerate} 
\usepackage{graphicx}
\usepackage{tabularx}
\usepackage[margin=1.0in]{geometry}
\definecolor{mygrey}{gray}{0.3}
\usepackage[all]{xy}

\begin{document}
\title{ITCS 531 part A test 2021}
\author{}
\date{}
\theoremstyle{plain}
\newtheorem{Q}{Question}
\maketitle
Answer all questions. In all answers include your working.
\begin{Q}
\begin{enumerate}[a)]
\item Find the highest common factor of 236 and 122 using the Euclidean algorithm.
\item Calculate $(99^2 \mod 32)^3 \mod 15$.
\item How many positive integers smaller than 2000 are there that are divisible by 13 and not divisible by 11?
\item Goldbach's conjecture, which is currently unproven but generally believed to be true, states that:
\begin{itemize}
\item[$(\dagger)$] If $n$ is an even natural number greater than 2, then $n$ is a sum of two prime numbers (e.g. 4 = 2+2, 8 = 3+5 etc.).
\end{itemize}
Golbach's original formulation of the conjecture was:
\begin{itemize}
\item[$(\ddagger)$] If $n$ is any natural number greater than 5, then $n$ is a sum of three primes.
\end{itemize}
Prove that $(\dagger)$ and $(\ddagger)$ are equivalent.

\end{enumerate}
\end{Q}

\begin{Q}
\begin{enumerate}[a)]
\item There is a city where there are only two kinds of people, monks and thieves. Monks \emph{always} tell the truth, and thieves \emph{always} lie. Traveling in this city you meet two people, $A$ and $B$. Person $A$ tells you that both $A$ and $B$ are monks, but person $B$ tells you that $A$ is a thief. Say with justification whether each of $A$ and $B$ is a monk or a thief.
\item Let $\phi$ be the propositional formula $\neg((p\wedge q)\rightarrow (p\vee q))$. Is $\phi$ satisfiable? Justify your answer.
\item Let $\mathscr L = \{0,1,+,\times, \leq\}$ be a signature in first-order logic, and suppose $0$ and $1$ are constants, that $+$ and $\times$ are binary functions, and that $\leq$ is a binary relation. Let $\mathbb Z$ be an $\mathscr L$-structure by interpreting the non-logical symbols of $\mathscr L$ with their usual meanings. For each of the following $\mathscr L$-sentences, say if it is true or false in $\mathbb Z$. Explain your reasoning.
\begin{enumerate}[i)]
\item $\forall x\forall y( (x \leq x\times y) \vee (y\leq x\times y))$.
\item $\forall x\exists y(x + y \approx 0)$.
\item $\exists y \forall x(x + y \approx 0)$.
\end{enumerate}
\item With $\mathscr L$ as in part c), write an $\mathscr L$-formula $\phi(x)$ with a single free variable $x$ such that $\mathbb Z,v\models \phi(x)$ if and only if $|v(x)|$ is prime.
\end{enumerate}
\end{Q}



\begin{Q}
\begin{enumerate}[a)]
 \item Let $n$ be a natural number with $n >2$. Prove by induction that $2^n > 2n$.
\item Arrange the following sets in order of size. 
\[\mathbb N, \mathbb R, \mathbb Q, \mathbb Z, \mathbb C, \mathbb R^2, \{1,2,3,4\}\]
If two or more sets are the same size then say so. You don't have to provide proofs in this question.
\end{enumerate}
\end{Q}
\end{document}