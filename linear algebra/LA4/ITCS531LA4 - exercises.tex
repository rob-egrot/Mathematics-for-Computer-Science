\documentclass{article}

\usepackage{amsmath, mathrsfs, amssymb, stmaryrd, cancel, hyperref, relsize,tikz,amsthm,comment,enumerate}
\usepackage{graphicx}
\usepackage{xfrac}
\hypersetup{pdfstartview={XYZ null null 1.25}}
\usepackage[all]{xy}
\usepackage[normalem]{ulem}
\usepackage{tikz-cd}


\theoremstyle{plain}
\newtheorem{theorem}{Theorem}{\bfseries}{\itshape}
\newtheorem{proposition}{Proposition}{\bfseries}{\itshape}
\newtheorem{definition}{Definition}{\bfseries}{\upshape}
\newtheorem{lemma}{Lemma}{\bfseries}{\upshape}
\newtheorem{example}{Example}{\bfseries}{\upshape}
\newtheorem{corollary}{Corollary}{\bfseries}{\upshape}
\newtheorem{remark}{Remark}{\bfseries}{\upshape}
\newtheorem{fact}{Fact}{\bfseries}{\upshape}
\newtheorem{Q}{Exercise}{\bfseries}{\upshape}

\newtheorem*{theorem*}{Theorem}

% Uncomment below to create stand-alone file
% \newcommand*{\prefix}{}

\newcommand{\bN}{\mathbb{N}}
\newcommand{\bZ}{\mathbb{Z}}
\newcommand{\bQ}{\mathbb{Q}}
\newcommand{\bR}{\mathbb{R}}
\newcommand{\bP}{\mathbb{P}}
\newcommand{\bF}{\mathbb{F}}
\newcommand{\spa}{\mathrm{span}}
\newcommand{\cL}{\mathcal{L}}
\DeclareMathOperator{\ran}{\mathrm{ran}}


\title{ITCS 531 \\Linear Algebra 4:  Inner products on real vector spaces}
\author{Rob Egrot}
\date{}

\begin{document}
\maketitle
%\includecomment{comment}

\begin{Q}
Let $a,b,c,d$ be positive real numbers. Use Cauchy-Schwarz (theorem \ref{T:LA4CS}) to prove that 
\[16 \leq (a+b+c+d)(\frac{1}{a} + \frac{1}{b} + \frac{1}{c} + \frac{1}{d}).\]
\end{Q}
\begin{comment}
\textbf{Solution:}
Let $u = (\sqrt{a} , \sqrt b, \sqrt{c}, \sqrt{d})$, and let $v = (\frac{1}{\sqrt a},\frac{1}{\sqrt b}\frac{1}{\sqrt c},\frac{1}{\sqrt d})$. Then $\langle u, v \rangle^2 = (1+1+1+1)^2 = 16$. Also, $\| u\|^2 = \langle u,u\rangle = a+b+c+d$, and $\| v\|^2 = \langle v,v\rangle = \frac{1}{a}+\frac{1}{b}+\frac{1}{c}+\frac{1}{d}$. By Cauchy-Schwarz we have $\langle u, v \rangle^2\leq\| u\|^2\| v\|^2$. I.e. $16\leq (a+b+c+d)(\frac{1}{a}+\frac{1}{b}+\frac{1}{c}+\frac{1}{d})$, which is what we want to prove.   
\end{comment}

\begin{Q}
Let $x_1,\ldots ,x_n\in\bR$. Prove that $(x_1+\ldots +x_n)^2\leq n(x_1^2+\ldots +x_n^2)$. 
\end{Q}
\begin{comment}
\textbf{Solution:}
Let $u = (x_1,\ldots,x_n)$, and let $v = (1,\ldots,1)$. So $\langle u, v\rangle^2 = (x_1+\ldots+x_n)^2$. Also, $\|u\|^2 = x_1^2+\ldots+ x_n^2$, and $\|v\|^2 = 1+1+\ldots+1 = n$. So, by Cauchy-Schwarz, we have $(x_1+\ldots+x_n)^2\leq n( x_1^2+\ldots+ x_n^2)$ as claimed. 
\end{comment}

\begin{Q}
Is there an inner product on $\bR^2$ such that the associated norm is given by $\|(x,y)\| = \max\{x, y\}$? Provide a proof for your answer.
\end{Q}
\begin{comment}
\textbf{Solution:}
No. For example, think about when both $x$ and $y$ are negative. Then $\max\{x, y\}$ is also negative, but norms are never negative. 
\end{comment}

\begin{Q}
Let $V$ be a real inner product space.
\begin{enumerate}[(a)]
\item Prove that $\langle u+v, u-v\rangle = \|u\|^2 -\|v\|^2$ for all $u,v\in V$.
\item A rhombus is a parallelogram whose four sides all have equal length. Prove that the diagonals of a rhombus are orthogonal to each other.
\item  Prove that
\[\langle u, v\rangle = \frac{\|u+v\|^2-\|u-v\|^2}{4}\]
for all $u,v\in V$.
\end{enumerate}
\end{Q}
\begin{comment}
\textbf{Solution:}
\begin{enumerate}[(a)]
\item \begin{align*}
\langle u+v,u-v\rangle &= \langle u, u-v\rangle + \langle v,u-v \rangle\\
&=\langle u, u \rangle - \langle u,v \rangle + \langle v, u \rangle - \langle v, v \rangle\\
&= \|u\|^2 - \|v\|^2.
\end{align*}
\item Think about this picture:
\[\xymatrix{ & &\ar[rrr]^{u}\ar@{..>}[ddr] & & &\\
\\
\ar@{..>}[uurrrrr]\ar[uurr]^v\ar[rrr]_u & && \ar[uurr]_{v} 
}\]
This is a rhombus. The up-right diagonal is given by $u+v$, and the down-right diagonal is given by $u-v$. By part (a) we have $\langle u+v, u-v\rangle = \|u\|^2 -\|v\|^2$, and by the assumption that $u$ and $v$ are the same length we have $\|u\|^2 =\|v\|^2$. So $\langle u+v, u-v\rangle = 0$, which means the diagonals are orthogonal to each other.
\item We could work through this calculation directly, but with a little cleverness we can use a trick and save some effort. Set $x = \frac{u+v}{2}$, and set $y = \frac{u-v}{2}$. Then $\langle u, v\rangle  = \langle x+y, x-y \rangle$, and by part (a) we have 
\[\langle x+y, x-y \rangle = \|x\|^2 -\|y\|^2 = \frac{\|u+v\|^2}{4} - \frac{\|u-v\|^2}{4}.\] 
\end{enumerate}
\end{comment}

\end{document}